%% Generated by Sphinx.
\def\sphinxdocclass{jupyterBook}
\documentclass[letterpaper,10pt,english]{jupyterBook}
\ifdefined\pdfpxdimen
   \let\sphinxpxdimen\pdfpxdimen\else\newdimen\sphinxpxdimen
\fi \sphinxpxdimen=.75bp\relax
%% turn off hyperref patch of \index as sphinx.xdy xindy module takes care of
%% suitable \hyperpage mark-up, working around hyperref-xindy incompatibility
\PassOptionsToPackage{hyperindex=false}{hyperref}
%% memoir class requires extra handling
\makeatletter\@ifclassloaded{memoir}
{\ifdefined\memhyperindexfalse\memhyperindexfalse\fi}{}\makeatother

\PassOptionsToPackage{warn}{textcomp}

\catcode`^^^^00a0\active\protected\def^^^^00a0{\leavevmode\nobreak\ }
\usepackage{cmap}
\usepackage{fontspec}
\defaultfontfeatures[\rmfamily,\sffamily,\ttfamily]{}
\usepackage{amsmath,amssymb,amstext}
\usepackage{polyglossia}
\setmainlanguage{english}



\setmainfont{FreeSerif}[
  Extension      = .otf,
  UprightFont    = *,
  ItalicFont     = *Italic,
  BoldFont       = *Bold,
  BoldItalicFont = *BoldItalic
]
\setsansfont{FreeSans}[
  Extension      = .otf,
  UprightFont    = *,
  ItalicFont     = *Oblique,
  BoldFont       = *Bold,
  BoldItalicFont = *BoldOblique,
]
\setmonofont{FreeMono}[
  Extension      = .otf,
  UprightFont    = *,
  ItalicFont     = *Oblique,
  BoldFont       = *Bold,
  BoldItalicFont = *BoldOblique,
]


\usepackage[Bjarne]{fncychap}
\usepackage[,numfigreset=1,mathnumfig]{sphinx}

\fvset{fontsize=\small}
\usepackage{geometry}


% Include hyperref last.
\usepackage{hyperref}
% Fix anchor placement for figures with captions.
\usepackage{hypcap}% it must be loaded after hyperref.
% Set up styles of URL: it should be placed after hyperref.
\urlstyle{same}

\addto\captionsenglish{\renewcommand{\contentsname}{Front matter}}

\usepackage{sphinxmessages}



         \usepackage[Latin,Greek]{ucharclasses}
        \usepackage{unicode-math}
        % fixing title of the toc
        \addto\captionsenglish{\renewcommand{\contentsname}{Contents}}
        

\title{What kind of thing is a planet?}
\date{May 16, 2021}
\release{}
\author{Rohan S.\@{} Byrne}
\newcommand{\sphinxlogo}{\vbox{}}
\renewcommand{\releasename}{}
\makeindex
\begin{document}

\pagestyle{empty}
\sphinxmaketitle
\pagestyle{plain}
\sphinxtableofcontents
\pagestyle{normal}
\phantomsection\label{\detokenize{index::doc}}


\sphinxAtStartPar
\sphinxhref{https://zenodo.org/badge/latestdoi/339255399}{\sphinxincludegraphics{{339255399}.png}}


\chapter{Declaration}
\label{\detokenize{frontmatter/declaration:declaration}}\label{\detokenize{frontmatter/declaration::doc}}
\sphinxAtStartPar
The thesis comprises only original work towards the PhD, except where indicated in the preface. Due acknowledgement has been made in the text to all other material used. The thesis is shorter than the maximum word limit in length, exclusive of tables, maps, bibliography and appendices.
Rohan Byrne
May 2021


\chapter{Preamble}
\label{\detokenize{frontmatter/preamble:preamble}}\label{\detokenize{frontmatter/preamble::doc}}

\chapter{Abstract}
\label{\detokenize{frontmatter/abstract:abstract}}\label{\detokenize{frontmatter/abstract::doc}}

\chapter{Introduction}
\label{\detokenize{frontmatter/introduction:introduction}}\label{\detokenize{frontmatter/introduction::doc}}
\sphinxAtStartPar
The first\sphinxhyphen{}order geological feature of our planet must unarguably be plate tectonics. It is responsible for the partitioning of the Earth into terrestrial and marine environments; it drives the periodic gathering and scattering of the continents; it erects and tears down the great mountain ranges; it recycles and perpetually renews the chemical inventory of the surface world. Plate tectonics has even been implicated as a prerequisite {[}\sphinxcite{references:id416}, \sphinxcite{references:id417}{]} or co\sphinxhyphen{}requisite {[}\sphinxcite{references:id2}, \sphinxcite{references:id435}{]} for the emergence of life on our planet. Alongside the theory of evolution, the theory of plate tectonics underpins the deepest meaningful chapters of the human story.

\sphinxAtStartPar
It is provocative, then, to note that, sixty years since its advent, there remain deep unsolved questions regarding the theory of plate tectonics. It is not yet known how or when it originated {[}\sphinxcite{references:id350}, \sphinxcite{references:id305}, \sphinxcite{references:id722}{]}, nor when and under what conditions it might cease {[}\sphinxcite{references:id644}, \sphinxcite{references:id498}, \sphinxcite{references:id647}{]}. It is not yet clear why it operates on Earth and not, or no longer, on any other comparable planet {[}\sphinxcite{references:id338}, \sphinxcite{references:id518}, \sphinxcite{references:id349}{]}. Most significantly, there is as yet no single framework that can successfully unify the kinematic and dynamic aspects of plate tectonics {[}\sphinxcite{references:id331}, \sphinxcite{references:id461}, \sphinxcite{references:id409}, \sphinxcite{references:id631}{]}.

\sphinxAtStartPar
The need for a satisfactory solution to the conundrum of plate tectonics has been sharpened of late by the advent of large\sphinxhyphen{}scale exoplanet astronomy. As of writing, NASA’s exoplanet database enumerates some four thousand confirmed planet sightings beyond our solar system {[}\sphinxcite{references:id462}{]}, of which more than fifty could be considered ‘Earth\sphinxhyphen{}like’ even under the strictest definition of the term {[}\sphinxcite{references:id296}{]}. Factoring in observational biases toward smaller stars and larger, closer\sphinxhyphen{}orbiting planets, the total number of so\sphinxhyphen{}called ‘Earth\sphinxhyphen{}likes’ within fifty light\sphinxhyphen{}years alone could number in the thousands {[}\sphinxcite{references:id464}{]}. There is an increasingly strong case to be made that such planets may ultimately prove to make up the numerical bulk of all planets in our universe {[}\sphinxcite{references:id463}{]}.

\sphinxAtStartPar
Such buoyant results suggest the Earth is likely to find itself in good company in the cosmos. However, a ‘comparative planetology’ approach to our own solar system supports the opposite conclusion. Here, at least four planetary bodies qualify as ‘Earth\sphinxhyphen{}like’ according to the most common astronomical definitions (CITATION); yet, of these, only one has a youthful and dynamic surface capable of supporting the lengthy chemical reaction chains that sustain and constitute life as we know it {[}\sphinxcite{references:id417}{]}. Meanwhile, some planetary bodies formerly viewed as too small or too cold to host life have proved to be tantalisingly dynamic, with Saturn’s moon Titan {[}\sphinxcite{references:id414}{]}, Jupiter’s moon Europa {[}\sphinxcite{references:id741}{]}, and even the spurned ‘dwarf’ planet Pluto {[}\sphinxcite{references:id299}{]} exhibiting clear signs of recent geological activity.

\sphinxAtStartPar
Clearly the view that planetary behaviours are predominantly a function of bulk planetary parameters wants nuance. The situation immediately suggests two hypotheses, both as plausible as they are unalike: either planets are deterministic heat engines driven rigidly and precisely by their bulk parameters \sphinxhyphen{} henceforth the ‘Sensitivity Hypothesis’ {[}\sphinxcite{references:id498}, \sphinxcite{references:id426}, \sphinxcite{references:id744}{]}; or they are fundamentally non\sphinxhyphen{}linear systems governed by chaotic attractors and the exigencies of chance and history \sphinxhyphen{} the ‘Stochastic Hypothesis’ {[}\sphinxcite{references:id439}, \sphinxcite{references:id435}, \sphinxcite{references:id513}{]}. The contest of these two ideas goes to the heart of planetary science.

\sphinxAtStartPar
This thesis sets out the reasoning, methodology, and outcomes of a massive suite\sphinxhyphen{}modelling program designed to test this question at a scale never previously attempted. First, a rigorous literature review and theoretical discussion is accompanied by the results of preliminary one\sphinxhyphen{}dimensional modelling, designed to illustrate the fundamental issue and provide supporting quantitative constraints for further analysis (Chapter 1). Next, we describe the novel methodologies and new tools devised for this program, together with a brief account of the technical and theoretical challenges involved in numerical modelling on the scale required for this work (Chapter 2). In subsequent chapters we lay out the surprising results of a comprehensive benchmarking study (Chapter 3), thoroughly replicate and extend an iconic viscoplastic rheology paper (Chapter 4), and explore the impact of material heterogeneities on a planet’s choice of tectonic mode (Chapter 5). In the final results chapter (Chapter 6), we apply our model system to the Early Earth to critically evaluate the role that oceanic plateaux could have played in the rise of plate tectonics on our planet. This is followed by two discussion chapters in which we review the theoretical and methodological learnings of the study as a whole (Chapter 7), and endeavour to situate the work in a ‘comparative planetology’ framework, with implications for the projected likelihood of finding substantially Earth\sphinxhyphen{}like planets beyond our own (Chapter 8). With the essential business of our study complete, we take an opportunity to consider the status and future of our discipline as a whole from the point of view of the history and philosophy of science (Chapter 9), before bringing matters to a close with some very brief final remarks (Conclusion).

\sphinxAtStartPar
Naturally, references and certain supplementary materials may be found in their proper place (Bibliography and Appendices); in addition to these we also direct the interested reader to the website for our modelling tool, PlanetEngine (planetengine.info), to the interactive digital version of our thesis (https://rsbyrne.github.io/thesis), and to our popular science blog, which navigates related terrain (buildmeaplanet.com). A detailed point\sphinxhyphen{}for\sphinxhyphen{}point overview of the entire thesis is also provided immediately following this introduction (Summary).

\sphinxAtStartPar
It has been the ambition of this author to attempt to make a significant, and indeed, foundational contribution to the young discipline of fundamental planetary science: what we argue will in the fullness of time be called ‘planetology’ (Conclusion). Such efforts are almost always doomed, and it will be the judgement of the reader whether this attempt falls into that category; but, for the progress of science as a whole, we must on occasion invest in a little fancy, whose unbidden fruit is so often the sweetest. Along this lofty and perhaps foolhardy road, the author has begged and received the indulgence, aid, favour, and patience of many kind and wise people. Though we ask no forgiveness for our impossible dream, we do extend our gratitude to those without whom we could not even have made the attempt. The rather lengthy list concludes the prefatory section of this thesis (Acknowledgements).


\chapter{Acknowledgements}
\label{\detokenize{frontmatter/acknowledgements:acknowledgements}}\label{\detokenize{frontmatter/acknowledgements::doc}}

\chapter{Review and Theory}
\label{\detokenize{content/chapter_01_background/main:review-and-theory}}\label{\detokenize{content/chapter_01_background/main::doc}}
\sphinxAtStartPar
The purpose of this study is to deduce what underlying parameters most strongly control a planet’s choice of tectonic mode. The fundamental question regards the existence or otherwise of a well\sphinxhyphen{}defined ‘tectonic phase diagram’ in which primarily temperature defines the proper global geodynamic regime at any given time. Phase spaces are defined by their phase transitions, and in the field of planetary geodynamics, only one such transition is well\sphinxhyphen{}attested and amenable to study: the emergence on Earth of plate tectonics from an earlier pre\sphinxhyphen{}tectonic state of disputed character. The Early Earth is therefore key to the larger question of tectonic modality.

\sphinxAtStartPar
In this review chapter, we will present a brief history of scholarship on the topic, summarise the chief lines of evidence, and discuss the leading hypotheses regarding the divergent fates of Earth and our most familiar planetary cousins. Our line of inquiry will be illustrated with some simple one\sphinxhyphen{}dimensional modelling. We will not discuss in detail the analytical and numerical intricacies of the problem at hand, which will instead be related as appropriate in the prefacing remarks of the relevant results chapters (Chapters 3, 4, 5, \& 6); nor will we provide much background or context for our methodological approach, which is treated separately (Chapter 2 \& Chapter 7).


\section{Planets from first principles}
\label{\detokenize{content/chapter_01_background/main:planets-from-first-principles}}
\sphinxAtStartPar
Our subject is planets \sphinxhyphen{} but what is a planet?

\sphinxAtStartPar
According to the International Astronomical Union highly\sphinxhyphen{}publicised ruling of 2006 {[}\sphinxcite{references:id472}{]}, “a planet is a celestial body that; 1) is in orbit around a sun, 2) has sufficient mass to have a nearly round shape, 3) has “cleared the neighborhood” around its orbit.” Under this rubric, Pluto, Charon, Eris, and other icy worlds of the outermost solar system are disqualified. Indeed, the rationale for the IAU’s opinion was that any broader definition would result in a categorical calamity: potentially hundreds of new worlds would achieve the mantle of ‘planet’, including many that at the time seemed unlikely to be of much significance or interest {[}\sphinxcite{references:id126}{]}, creating needless public and academic ‘annoyance’ {[}\sphinxcite{references:id118}{]}.

\sphinxAtStartPar
The IAU’s ruling sparked an impressive global backlash, including an attempt by lawmakers to quash the revision {[}\sphinxcite{references:id124}{]} and various pleas for celebrity astronomers to intercede {[}\sphinxcite{references:id123}{]}. The case has even become a niche topic of sociological literature, with papers discussing its effect on children {[}\sphinxcite{references:id122}, \sphinxcite{references:id128}{]} and its impact on popular perceptions of the credibility of astronomy {[}\sphinxcite{references:id121}, \sphinxcite{references:id131}{]}.

\sphinxAtStartPar
In the immediate aftermath, a petition of over 300 signatures from attendees at the IAU meeting itself protested the new classification, which had been decided by a partial convention of the union rather than a general colloquium, perhaps to avoid protest {[}\sphinxcite{references:id127}{]}. Alan Stern, the director of the New Horizons mission to the outer solar system, quickly became one of the highest\sphinxhyphen{}profile advocates for revising or ignoring the IAU opinion {[}\sphinxcite{references:id132}{]}. He and his colleagues have since developed a so\sphinxhyphen{}called ‘geophysical definition’ which is now the standard at NASA {[}\sphinxcite{references:id119}{]}: “A planet is a sub\sphinxhyphen{}stellar mass body that has never undergone nuclear fusion and that has sufficient self\sphinxhyphen{}gravitation to assume a spheroidal shape adequately described by a triaxial ellipsoid regardless of its orbital parameters.”

\sphinxAtStartPar
Stern’s definition is a profound revision of the cosmos, but one which is well\sphinxhyphen{}argued and well\sphinxhyphen{}suited to the purposes of planetary scientists. Under Stern’s mandate, not only Pluto and Eris, but also the asteroid belt objects Ceres and Vesta, as well as various ‘satellite planets’ \sphinxhyphen{} Earth’s moon, Pluto’s partner Charon, and most of the satellites of the outer gas giants \sphinxhyphen{} all would be considered equally as planets. Such an expansion puts the geophysical, atmospheric, and climatic aspects of planets front and centre, above and beyond their orbital or formational histories. It sets the stage epistemologically for a fundamental deepening of our understanding of the unique processes that govern sub\sphinxhyphen{}stellar objects in our universe, and \sphinxhyphen{} as we will discuss later (Chapter 8) \sphinxhyphen{} may be regarded in retrospect as a milestone event in the birth of a new field of focussed, integrated, holistic planetary science.


\section{Heat and geometry: planetary basics}
\label{\detokenize{content/chapter_01_background/main:heat-and-geometry-planetary-basics}}
\sphinxAtStartPar
The geophysical definition of a planet provides some clarification on what should and what shouldn’t be included as essential considerations of our topic. A planet is a gravitationally\sphinxhyphen{}bound ellipsoidal object, not currently or formerly made of plasma, surrounded by mostly empty space. Such a definition has some immediate implications. The chemical and gravitational prescriptions of Stern’s definition provide an upper and lower spatial scale respectively: anywhere from the radius of Mimas, at 200 km the smallest rounded body in the solar system {[}\sphinxcite{references:id473}{]}, to the radius of Jupiter, about 70,000 km, above which any additional mass tends to be compressed without increasing planet size {[}\sphinxcite{references:id118}{]}. It implies a temporal scale too: between the atmospheric (hours) and geological (gigayear) timescales.

\sphinxAtStartPar
Other implicit constraints are more subtle. Because a planet is qualified by its shape \sphinxhyphen{} a function of mass and matter \sphinxhyphen{} it is parsimonious to stipulate that these quantities are connate to each planet and cannot be changed. Of course, in reality, such changes do occur: consider the formation of Earth’s moon from a putative impact with a Mars\sphinxhyphen{}sized planetoid in the Hadean {[}\sphinxcite{references:id117}{]}, or the hypothesised delivery to Earth of a significant mass of volatiles by icy planetoids {[}\sphinxcite{references:id116}{]}. In the system outlined here, however, such events would be considered ‘catastrophic’ \sphinxhyphen{} outside of the prescribed spatiotemporal scales of the subject, and violating the prescribed conservative quantities of the problem; in essence, such a calamity would be considered the ‘death’ of the original planet and the ‘birth’ of a new one.

\sphinxAtStartPar
There are also more sustained violations: erosion of planetary atmospheres into space, for example {[}\sphinxcite{references:id112}, \sphinxcite{references:id113}{]}, or the capture of solar protons in the aurorae {[}\sphinxcite{references:id114}{]}, or the ongoing low\sphinxhyphen{}level exchange of meteoroids between Earth and Mars {[}\sphinxcite{references:id115}{]}. Such endemic exceptions, however, plague all systems; hence we will dub them ‘pathological’, requiring careful handling but not deference.

\sphinxAtStartPar
Planets, then, can be said to be spheroidal, sub\sphinxhyphen{}stellar objects of a fixed mass and bulk chemistry. Within these parameters, however, there are considerable degrees of freedom:
\begin{itemize}
\item {} 
\sphinxAtStartPar
The geometric characters of the spheroid can change \sphinxhyphen{} eccentricity, radius, and obliquity \sphinxhyphen{} with consequences for the centre of mass, the gravitational field, and the tidal coupling of the planet with any extra\sphinxhyphen{}planetary systems.

\item {} 
\sphinxAtStartPar
The distribution of mass can change, anywhere between the extremes of an arbitrarily thin hollow shell and an arbitrarily dense core, with consequences for the moment of inertia and rate of rotation as well as the gravitational field and the gravitational binding energy. The total mass, of course, cannot change \sphinxhyphen{} this would be ‘catastrophic’ \sphinxhyphen{} hence the gravitational pull of a planet at a sufficient distance above the surface is fixed.

\item {} 
\sphinxAtStartPar
The angular velocity of the planet can change due to internal forces and geometric transformations. Angular momentum can also change, though necessarily only by interactions with extra\sphinxhyphen{}planetary bodies: when these interactions are periodic, systematic, and persistent over geological time, they may be parameterised as an intrinsic forcing within the planet; otherwise, they may be considered ‘pathological’ (if frequent) or ‘catastrophic’ (if infrequent) exceptions, outside the intrinsic scope of the problem.

\item {} 
\sphinxAtStartPar
The chemical makeup of the planet can change. Indeed, all the degrees of freedom within any constituent molecules of a planet are available as degrees of freedom of the planet as a whole, as are all the possibilities for the breaking of old bonds and the forging of new ones.

\item {} 
\sphinxAtStartPar
The isotopic makeup of a planet can change, albeit largely without consequence for bulk chemistry \sphinxhyphen{} hence their enormous analytical utility as tracers and monitors for Earth sciences. Earth is overwhelmingly composed of less than 300 stable nuclides {[}\sphinxcite{references:id111}{]}.

\item {} 
\sphinxAtStartPar
The elemental inventory of a planet can change; here, however, we must be careful. The decay of short\sphinxhyphen{}lived radioactive nuclides like Uranium\sphinxhyphen{}235 and Thorium\sphinxhyphen{}232 into daughter elements are significant to the system; the decays of extremely long\sphinxhyphen{}lived nuclides like Uranium\sphinxhyphen{}238 and Calcium\sphinxhyphen{}48 are not significant over ‘planetary’ timescales as stipulated above (“Nudat 2” n.d.). Decay of non\sphinxhyphen{}radioactive nuclides due to the finite half\sphinxhyphen{}lives of subatomic particles are not significant on any reasonable timescale and are not available as degrees of freedom for planetary processes as we choose to consider them.

\item {} 
\sphinxAtStartPar
The spatial distribution of chemical and elemental species can change: to wit, the differentiation of the felsic crust, mafic mantle, and metallic core due to gravitational forcing; but also, the partitioning of rare\sphinxhyphen{}earth elements and radiogenics into the continental crust due to purely chemical factors.

\end{itemize}

\sphinxAtStartPar
Each of these degrees of freedom entail certain endothermic or exothermic reactions, and these naturally chain one to another in complex ways. The net result, however, is the exhaustion of all possible gradients and the conversion of all potential energy into diffuse kinetic energy, i.e. heat. But what is the total internal energy of a planet \sphinxhyphen{} its enthalpy?

\sphinxAtStartPar
Because there is no such thing as ‘absolute’ enthalpy, it is necessary to first define a natural point of reference for the system against which changes in the energy budget may be compared. Thankfully, the stipulations we have already made regarding the conserved quantities of planets, as well as the limitations we have provided on relevant spatiotemporal scales and the resultant pruning of available degrees of freedom, imply one fairly reliable, though somewhat counter\sphinxhyphen{}intuitive, reference point: not the planet’s initial condition, but rather its ‘rest state’ after a sufficient span of time.

\sphinxAtStartPar
Consider a situation where the Earth as a whole exchanges no heat with its environment. In fact, this is a surprisingly credible simplification: the global geothermal flux is estimated at only 47 terawatts {[}\sphinxcite{references:id108}, \sphinxcite{references:id762}{]}, predominantly along the mid\sphinxhyphen{}ocean ridge {[}\sphinxcite{references:id430}{]}. This is insignificant compared to the 100,000 terawatts processed by the solar/atmospheric/oceanic system and comparable to the over 16 terawatts consumed by human civilisation in the modern era {[}\sphinxcite{references:id109}{]}; a small\sphinxhyphen{}enough flux that, if held constant and ignoring all internal heat production, a billion years would only see a 232 K drop in planetary average temperatures {[}\sphinxcite{references:id498}{]}. Indeed, looking backward, we find Earth’s core has only lost in the order of 50 K per billion years since the planet was formed {[}\sphinxcite{references:id107}{]}, confirming our intuition.

\sphinxAtStartPar
What is the ‘ultimate’ temperature of an isenthalpic planet such as we have described? This will be the temperature obtained at an arbitrarily remote time in the future \sphinxhyphen{} long enough for all permissible degrees of freedom to be exploited. At this time, only heat remains: a planet has no more tricks to perform. It transpires that this number is not all that difficult to derive, nor so uncertain as we might assume \sphinxhyphen{} at least in comparison to the inverse problem, which is fraught with controversy.

\sphinxAtStartPar
On ‘Ultimate Earth’, the radiogenic nuclides have completely decayed into heat and stable daughters, while the stable nuclides have sorted by density from top to bottom \sphinxhyphen{} a crude but serviceable assumption. The Earth is now a perfect sphere of the smallest possible size, and its rotation has completely ceased due to tidal interactions with other cosmic masses.

\sphinxAtStartPar
We attempted to calculate the temperature of Ultimate Earth based off order\sphinxhyphen{}of\sphinxhyphen{}magnitude estimates, published scaling laws, and reference data {[}\sphinxcite{references:id474}, \sphinxcite{references:id106}, \sphinxcite{references:id457}, \sphinxcite{references:id473}, \sphinxcite{references:id498}, \sphinxcite{references:id105}{]}. When calculated from present day observations, we derive a temperature of between 4,000 and 5,000 K, or about  2.92 * 10\textasciicircum{}31 joules: comparable to the temperature of the Earth’s core, where most of today’s heat is stored {[}\sphinxcite{references:id104}{]}. When we instead venture to calculate the temperature starting from the presumed initial condition of the Earth, we obtain an Ultimate Earth temperature an order of magnitude higher \sphinxhyphen{} 40,000 to 50,000 K, about 2.75 * 10\textasciicircum{}32 joules. In other words, the Earth’s heat has fallen by a factor of ten over its 4.5 billion year history.

\sphinxAtStartPar
The reason we can be fairly confident about these figures is that they are dominated by gravitational potential energy and the heat of accretion, which can both be trivially derived analytically from Earth’s mass; cumulatively, these factors account for about 10\textasciicircum{}32 joules of Ultimate Earth’s heat budget, about 93\%. Some 88\% of this energy is converted instantly into heat upon accretion; another 9\% is released through differentiation, of which 8.7\% is due to the core and 0.27\% is due to the crust. Only 3\% of the Earth’s original gravitational binding energy remains as potential energy which could feasibly be released as work and heat. The remaining degrees of freedom \sphinxhyphen{} orbital, chemical, and nuclear \sphinxhyphen{} add up to a little over 10\textasciicircum{}31 joules, but together make up the majority of the remaining energy available to be released. Interestingly, the amount of heat that could theoretically be liberated by gravitational sorting of nuclides \sphinxhyphen{} i.e. the formation of continents \sphinxhyphen{} is comparable to the remaining radiogenic budget of the Earth. If as much continental volume were to be produced in the next 5 Gy as in the previous, it would represent a significant proportion of geothermal flux.

\sphinxAtStartPar
In reality, the Earth is not isenthalpic: it receives solar heat and sheds geothermal heat into space. We observe here that some 89\% of the Earth’s heat has been lost to space since formation. This number is very close to the 88\% of heat that would have been released geologically instantaneously at accretion. As Lord Kelvin famously calculated {[}\sphinxcite{references:id103}{]}, the maximum timescale to reach a present\sphinxhyphen{}day geotherm from a wholly molten state is a mere tens of millions of years, and much, much shorter than that if more active thermal processing is at work. Hence it is conceivable, though unverifiable, that almost all of Earth’s accretionary heat was lost shortly after formation, with freezing of the core and radiogenic decay since accounting for most if not all of the observed geothermal fluxes at magnitudes similar to today’s.

\sphinxAtStartPar
We have laid out what is admittedly a fairly rudimentary construct to commence our thesis. However, we argue it furnishes the appropriate ‘ballpark’ figures and situates the intuition in the proper place for a truly planetary way of thinking about thermodynamics and flow. The essential lesson of the Ultimate Earth concept is that the overall thermal regime of a planet is in no hurry to reach equilibrium with space; that in fact, over the lifetime of many planets \sphinxhyphen{} curtailed by the circumstances of their host stars to mere billions, rather than tens of billions of years \sphinxhyphen{} no state resembling Ultimate Earth will ever be achieved by star\sphinxhyphen{}bound worlds. In short, heat is not destiny for a planet.


\section{From planetary to exoplanetary: an expanding scope of inquiry}
\label{\detokenize{content/chapter_01_background/main:from-planetary-to-exoplanetary-an-expanding-scope-of-inquiry}}
\sphinxAtStartPar
It is incredible to consider that, within the lifetime of a single PhD student, a family of nine known planets has become a universe of thousands.

\sphinxAtStartPar
The expansion in scope of planetary sciences has occurred so rapidly it is easy to lose track of what has changed. Discoveries, in our solar system and beyond, are moving faster than theory, and it seems a major revolution in our understanding of planets waits to be realised. An atmosphere of genuine hope and excitement surrounds the field \sphinxhyphen{} both within the academy and beyond.

\sphinxAtStartPar
Here we present a brief overview of relevant recent discoveries in our home solar system and beyond.


\subsection{Local planets}
\label{\detokenize{content/chapter_01_background/main:local-planets}}
\sphinxAtStartPar
Unmanned exploration of our local system is entering a renewed era of great activity, emboldened by growing public interest and increasing private cash {[}\sphinxcite{references:id198}{]}. This exploration is yielding provocative evidence that challenges our understanding of planetary dynamics.

\sphinxAtStartPar
While much of note has been uncovered on the rocky planets of our inner solar system, especially Mars, Venus, and the Moon, here we will discuss more recent discoveries whose import has arguably not yet been fully recognised in the broader geodynamics literature \sphinxhyphen{} planetary bodies that were supposed to be inert, which have proved to be full of activity. On the moons of Jupiter and Saturn, and far out beyond the orbit of Neptune, young, dynamic surfaces are being revealed, speaking to active internal processes that may support life \sphinxhyphen{} and hold invaluable clues to untangling the history of Earth and the deeper questions of tectonics.


\subsubsection{Europa}
\label{\detokenize{content/chapter_01_background/main:europa}}
\sphinxAtStartPar
Jupiter’s moon Europa has proved to be surprisingly analogous to Earth: its surface ice has been shown to exhibit plate\sphinxhyphen{}like tectonics {[}\sphinxcite{references:id549}{]}, with familiar fault patterns {[}\sphinxcite{references:id617}{]}, deformation at fault tips {[}\sphinxcite{references:id586}{]}, cryovolcanic ‘igneous’ terrains {[}\sphinxcite{references:id524}{]}, strikingly Earth\sphinxhyphen{}like ridge flexures, forebulges, and other collisional features {[}\sphinxcite{references:id560}{]}, and some very recent evidence of ongoing subduction {[}\sphinxcite{references:id277}{]}.

\sphinxAtStartPar
Europa is heated internally by tides {[}\sphinxcite{references:id583}{]}, but beyond that, little is known of its thermal state. Some have speculated the existence of deep sea vents which could be excellent candidates for hosting life {[}\sphinxcite{references:id555}{]}. Whether Europa proves fertile or not, it remains the only known example of active horizontal tectonics beyond Earth, and hence is of tremendous importance to the broader question of what makes a ‘living’ planet.


\subsubsection{Io}
\label{\detokenize{content/chapter_01_background/main:io}}
\sphinxAtStartPar
If Europa provides some context for the Earth of today, another of Jupiter’s moons, Io, may hold the key to its past. Io, which is comparable in size and makeup to our Moon {[}\sphinxcite{references:id193}{]}, is the only known body beyond Earth proven to exhibit non\sphinxhyphen{}icy volcanism in the present day {[}\sphinxcite{references:id194}{]}. The scale of volcanism on Io is extreme and unmatched in the solar system {[}\sphinxcite{references:id195}{]}; first observed during the Voyager {[}\sphinxcite{references:id191}{]} and Galileo {[}\sphinxcite{references:id187}{]} missions, the New Horizons flyby in 2007 detected no fewer than eleven active ejecta plumes, some jetting higher than 350 km {[}\sphinxcite{references:id196}{]}. Io’s tremendous activity is related to its great internal heat, derived from intense tidal forces {[}\sphinxcite{references:id591}{]}, which may even be sufficient to generate subsurface magma ocean {[}\sphinxcite{references:id190}{]}.

\sphinxAtStartPar
Though initially suspected to be relatively low\sphinxhyphen{}temperature due to sulphur content {[}\sphinxcite{references:id189}{]}, Io’s lavas are now thought to be mostly ultramafic and extremely hot {[}\sphinxcite{references:id188}, \sphinxcite{references:id187}{]}, fuelling analogies with the Archaean komatiite lavas on Earth {[}\sphinxcite{references:id184}, \sphinxcite{references:id185}{]}. For these reasons, Io \sphinxhyphen{} and the supposed ‘heat pipe’ tectonic mode that operates there (discussed later) \sphinxhyphen{} may provide a critical window into the early Earth {[}\sphinxcite{references:id183}, \sphinxcite{references:id186}, \sphinxcite{references:id391}, \sphinxcite{references:id767}{]}.


\subsubsection{Enceladus}
\label{\detokenize{content/chapter_01_background/main:enceladus}}
\sphinxAtStartPar
Saturn’s moon Enceladus may be second only to Earth in geological activity {[}\sphinxcite{references:id159}, \sphinxcite{references:id168}{]}, making it an indispensable reference for comparative planetology.

\sphinxAtStartPar
The dynamic nature of Enceladus first became known with the Voyager missions, when its surface was found have large, crater\sphinxhyphen{}free terrains {[}\sphinxcite{references:id178}, \sphinxcite{references:id179}{]} interpreted as evidence of cryovolcanism {[}\sphinxcite{references:id170}{]}, as well as some apparently tectonic features such as grabens and fault banding {[}\sphinxcite{references:id169}{]}, arguing for the existence of ongoing, active resurfacing on this small, frigid moon {[}\sphinxcite{references:id180}{]} where none had been considered possible before {[}\sphinxcite{references:id171}{]}. Enceladus was a chief target of the later Cassini mission, which captured high\sphinxhyphen{}resolution photos of the moon’s young and fault\sphinxhyphen{}marked surface {[}\sphinxcite{references:id175}{]}, acquired magnetometric evidence of a volcanically\sphinxhyphen{}replenished water\sphinxhyphen{}derived atmosphere {[}\sphinxcite{references:id173}{]}, and, most impressively, observed a large cryovolcanic plume {[}\sphinxcite{references:id542}{]} composed of almost pure water {[}\sphinxcite{references:id177}{]} jetting from a tectonically active region around the south pole {[}\sphinxcite{references:id172}{]}.

\sphinxAtStartPar
Enceladus is an ‘ice moon’, with a silicate core overlain by a mostly watery mantle and a water ice outer crust {[}\sphinxcite{references:id164}{]}. Its tiny size \sphinxhyphen{} less than 500 km in diameter \sphinxhyphen{} makes its evident activity a conundrum {[}\sphinxcite{references:id168}{]}. Like Europa and Io, Enceladus’ internal heat had been presumed to be predominantly tidal {[}\sphinxcite{references:id176}{]}, dissipated almost exclusively throughout the icy mantle {[}\sphinxcite{references:id165}{]}; however, measured heat fluxes and dynamic modelling demand alternative sources, thought to be supplied by significant radiogenic heating from the silicate core {[}\sphinxcite{references:id163}{]}. Consequently, Enceladus \sphinxhyphen{} like Earth \sphinxhyphen{} has both volumetric and basal heating components, and hence a Urey number, though no published estimates for this quantity yet exist. Also, on Enceladus, like on Earth, a feedback exists between the two heat sources: while Earth core heat plays a role in redistributing and partitioning heat\sphinxhyphen{}producing elements in the silicate mantle, Enceladus’ radiogenic heat may shift and enhance the tidal dissipation forces in its icy mantle \sphinxhyphen{} which may explain why this particular moon is full of activity, while the icier but otherwise similar Mimas is inert {[}\sphinxcite{references:id163}{]}.

\sphinxAtStartPar
Enceladus’ tectonics have been variously characterised as Earth\sphinxhyphen{}like plate tectonics {[}\sphinxcite{references:id167}{]}, Venusian episodic overturn {[}\sphinxcite{references:id160}, \sphinxcite{references:id166}{]}, and mobile\sphinxhyphen{}lid convection {[}\sphinxcite{references:id158}{]}. Particular analogues aside, what Enceladus argues above all is that the laws of geodynamics are not intrinsic to any particular chemistry or any particular spatiotemporal scale.


\subsubsection{Pluto}
\label{\detokenize{content/chapter_01_background/main:pluto}}
\sphinxAtStartPar
Pluto, controversially relegated to ‘dwarf’ status by the International Astronomical Union {[}\sphinxcite{references:id148}{]}, has since become the latest overlooked body to exhibit surprising geological activity {[}\sphinxcite{references:id141}{]}. Far from being a glorified comet, the New Horizons flyby in 2015 {[}\sphinxcite{references:id154}{]} showed Pluto to be a living world {[}\sphinxcite{references:id152}{]}, with dunes {[}\sphinxcite{references:id151}{]}, troughs {[}\sphinxcite{references:id149}{]}, six kilometre\sphinxhyphen{}tall mountains and plateaux {[}\sphinxcite{references:id134}{]}, active glaciology {[}\sphinxcite{references:id147}{]}, cryovolcanoes {[}\sphinxcite{references:id133}{]} and volcanic resurfacing {[}\sphinxcite{references:id140}, \sphinxcite{references:id299}, \sphinxcite{references:id146}{]}, and a thin but influential atmosphere {[}\sphinxcite{references:id150}{]}: all in all, a range of features that collectively “rival Mars in richness” according to one research team {[}\sphinxcite{references:id153}{]}.

\sphinxAtStartPar
Most significantly, it now seems Pluto \sphinxhyphen{} like the other icy bodies here discussed \sphinxhyphen{} may possess a subsurface water ocean of its own {[}\sphinxcite{references:id145}{]}; seasonal partial freezing of that ocean might explain the extensional features on the surface {[}\sphinxcite{references:id155}{]} and drive some of the observed cryovolcanism {[}\sphinxcite{references:id299}{]}. However \sphinxhyphen{} unlike Europa, Io, or Enceladus \sphinxhyphen{} tidal heat cannot be a factor on Pluto. This has led to speculation that the icy moon is anomalously well\sphinxhyphen{}insulated by clathrates {[}\sphinxcite{references:id142}{]}. In any case, the vitality of Pluto’s active tectonics despite the moon’s small size and presumably meagre thermal inventory underscores that tectonics can achieve a lot with a little over geological timescales.


\subsection{Exoplanets}
\label{\detokenize{content/chapter_01_background/main:exoplanets}}
\sphinxAtStartPar
Since the first confirmed sightings of planets beyond our solar system in the 1990s {[}\sphinxcite{references:id156}, \sphinxcite{references:id469}{]} the number of known exoplanets has climbed well over 4,000 {[}\sphinxcite{references:id462}{]}. Most of this extraordinary success is owed to Kepler mission, now terminated {[}\sphinxcite{references:id212}{]}. Some of these worlds have local analogues; many do not {[}\sphinxcite{references:id474}{]}. Planets have been observed around stars of all sizes, ages, and temperatures {[}\sphinxcite{references:id215}{]}, and hundreds of multiple\sphinxhyphen{}planet systems have been identified, including some with known populations comparable to our own solar system {[}\sphinxcite{references:id466}, \sphinxcite{references:id465}{]}. Statistical analyses which cross\sphinxhyphen{}correlate hit\sphinxhyphen{}rate with observation probability have concluded that planets are at least as abundant stars, and likely much more abundant, putting their number in the order of hundreds of billions to tens of trillions in our galaxy alone {[}\sphinxcite{references:id467}{]}.

\sphinxAtStartPar
Though unavoidable detection biases have skewed the dataset somewhat, it is today possible to obtain reasonable quantitative estimates of our galaxy’s planetary demography. The result has been a complete overturning of traditional planetary formation theory {[}\sphinxcite{references:id201}{]}. Far from being a universe dominated by cumbersome giants, it is now apparent that small rocky or icy planets in fact predominate {[}\sphinxcite{references:id471}{]}. Furthermore, the most common class of planets appears to comprise those which are larger in radius than Earth but smaller than Neptune {[}\sphinxcite{references:id210}{]}. A good example of this very popular new planet class is the system of three temperate, watery sub\sphinxhyphen{}Neptunes recently discovered in orbit around a nearby M\sphinxhyphen{}dwarf {[}\sphinxcite{references:id206}{]}. Our local solar system has no representative of this population and it has been speculated that, if one did exist, it may have been destroyed {[}\sphinxcite{references:id454}{]} or ejected {[}\sphinxcite{references:id205}{]} at some point long ago. These exciting new worlds are often close\sphinxhyphen{}orbiting and are tantalising candidates for life {[}\sphinxcite{references:id199}{]}.

\sphinxAtStartPar
Determining the interior structures, or even the density, of exoplanets discovered using radius\sphinxhyphen{}based methods can be difficult. Very large\sphinxhyphen{}radius bodies may be assumed to be Jupiter\sphinxhyphen{}like gas giants purely by the requirement that, if they were any denser, they would collapse. The density of smaller planets is more indeterminate; although there is some direct evidence from transit spectroscopy {[}\sphinxcite{references:id207}{]} and ideally more in the near future {[}\sphinxcite{references:id213}{]}, typically we are forced to rely on a combination of statistical inference and broad mass\sphinxhyphen{}radius scaling {[}\sphinxcite{references:id421}, \sphinxcite{references:id450}{]}. Such methods suggest that planets within the highly populous 1 \sphinxhyphen{} 4 Earth radius interval fall into two broad camps: ‘super\sphinxhyphen{}Earths’ which are rocky but potentially much more massive than our planet, and the smallest of the sub\sphinxhyphen{}Neptunes, implicitly volatile\sphinxhyphen{}rich planets with only slightly greater masses but significantly larger radii {[}\sphinxcite{references:id420}{]}. Between these camps exists an apparent ‘radius gap’ between 1.5 to 2 Earth radii {[}\sphinxcite{references:id214}{]}; this has been taken to imply that an evolutionary tipping point exists above which runaway gas accumulation invariably turns rocky planets into icy ones. Modelling suggests the bifurcation is not a sharp one; some oversized rocky worlds and undersized gassy ones are expected {[}\sphinxcite{references:id208}{]}. Nevertheless, the ‘Neptune transition’ implies that the scope of ‘Earth\sphinxhyphen{}like’ planetary science can practically be neatly constrained to the range of half to double Earth\sphinxhyphen{}radius. Fortunately, there is a great abundance of such planets: thousands, in fact, within 50 lightyears alone {[}\sphinxcite{references:id464}, \sphinxcite{references:id462}{]}.

\sphinxAtStartPar
Perhaps because we have no local examples, super\sphinxhyphen{}Earths have invited particularly intense speculation. Much debate surrounds the question of whether such worlds should be expected to have plate tectonics or not. The debate is pertinent to purely Earth\sphinxhyphen{}based studies because it goes to the question of what plate tectonics actually requires in order to commence and continue. Indeed, the super\sphinxhyphen{}Earth scholarship is almost evenly split between those who claim plate tectonics is more likely on larger worlds {[}\sphinxcite{references:id202}, \sphinxcite{references:id431}, \sphinxcite{references:id643}, \sphinxcite{references:id426}, \sphinxcite{references:id744}{]}, those who claim it is less likely {[}\sphinxcite{references:id204}, \sphinxcite{references:id733}, \sphinxcite{references:id433}, \sphinxcite{references:id747}{]}, and those again who argue that size has an inherently ambiguous effect {[}\sphinxcite{references:id312}, \sphinxcite{references:id735}, \sphinxcite{references:id434}{]}. The disagreement does not break down simply along methodological lines, with numerical modelling, analytical arguments, and geochemical considerations variously employed in support of all three contentions.

\sphinxAtStartPar
The super\sphinxhyphen{}Earth debate arguably generates more heat than light, but it is nonetheless illuminating in that it reveals the true uncertainties of geodynamics where we are unable to rely on the arrogance of fact, as we can on Earth. It prompts consideration of whether we would even have a notion of plate tectonics if we did not observe it in action at home, and invites speculation about what completely alien solutions may exist to the fundamental question of planets.


\section{Geodynamics: an evolving discipline}
\label{\detokenize{content/chapter_01_background/main:geodynamics-an-evolving-discipline}}
\sphinxAtStartPar
Tectonics, from the Greek tectonicus, ‘pertaining to building’, is a body of theory which endeavours to characterise the manner in which the spatial configuration of the landmasses changes over time. It is one of the foundational theories of natural history and arguably comprises the backbone of modern geosciences. However, tectonic theory is not, per se, a ‘dynamic’ theory with the power to explain how, or indeed, why such reconfiguration occurs. It is in this distinction that the field of geodynamics arises.

\sphinxAtStartPar
Geodynamics is to the theory of plate tectonics as genetics is to the theory of evolution. It seeks to interface the enormously successful narrative power of tectonics with the explanatory and predictive power of the underlying laws of physics \sphinxhyphen{} that is, classical mechanics and thermodynamics. Geodynamics defined in this way is necessarily younger than tectonic theory, which is itself only two generations old; as a discipline, it is arguably yet to move beyond descriptive science and acquire true predictive and prescriptive powers.


\subsection{Empirical foundations of geodynamic thought}
\label{\detokenize{content/chapter_01_background/main:empirical-foundations-of-geodynamic-thought}}
\sphinxAtStartPar
The world that geodynamics sets out to describe is characterised by several axiomatic properties. Some have been apparent since ancient times; others only much more recently. What is important to note is how powerfully suggestive they are of a unifying underlying phenomenon when assembled side by side, as follows:
\begin{itemize}
\item {} 
\sphinxAtStartPar
The Earth’s upper crust is partitioned chemically between felsic (continental) and mafic (oceanic) terrains {[}\sphinxcite{references:id282}{]}.

\item {} 
\sphinxAtStartPar
The Earth’s crust is partitioned texturally between extensive smooth regions (plains and plateaux) and localised, predominantly arcuate rough regions (hills and mountains) {[}\sphinxcite{references:id281}{]}.

\item {} 
\sphinxAtStartPar
The Earth’s surface is partitioned topographically between dominant low\sphinxhyphen{}elevation regions (ocean floors and continental shelves) and scattered high\sphinxhyphen{}elevation regions (continental cores) {[}\sphinxcite{references:id474}{]}.

\item {} 
\sphinxAtStartPar
The Earth’s upper crust is partitioned mechanically between broad areas of low or moderate strain rate (plate interiors) and narrow, globally interlaced bands of high strain rate (plate boundaries) {[}\sphinxcite{references:id284}{]}.

\item {} 
\sphinxAtStartPar
The Earth’s lithosphere can be shown to be clearly partitioned into zones of common velocity when the vectors are constructed around Euler poles {[}\sphinxcite{references:id474}{]}.

\item {} 
\sphinxAtStartPar
The Earth’s surface is partitioned chronologically between temporally heterogeneous zones of intermixed extremely young and extremely old rocks (continents) and temporally gradated zones made of rocks no older than 300 million years (ocean floors) {[}\sphinxcite{references:id279}{]}.

\item {} 
\sphinxAtStartPar
The landmasses of the Earth are variously scattered into fragments or consolidated into supercontinents throughout deep time {[}\sphinxcite{references:id280}{]}.

\item {} 
\sphinxAtStartPar
The bulk Earth is partitioned chemically and gravitationally between volatiles (atmosphere and hydrosphere), silicates (crust and mantle), and core (heavy metals) {[}\sphinxcite{references:id473}{]}.

\item {} 
\sphinxAtStartPar
The bulk Earth is heated basally by conduction across the core\sphinxhyphen{}mantle boundary as well as volumetrically (although not necessarily homogeneously) by radiogenic decay {[}\sphinxcite{references:id88}{]}.

\item {} 
\sphinxAtStartPar
The Earth, it is fairly well established, is round {[}\sphinxcite{references:id285}{]}.

\end{itemize}

\sphinxAtStartPar
Such a bill of facts, once collected, testifies strongly to the existence of a global assortative process with both toroidal and poloidal components. Because of the apparent thermal and compositional gradients involved, the immediate sensible process thus implicated is convection, i.e. transport by thermal buoyancy. Convection implies zones of upwelling and downwelling: rheological features which can be identified geologically with ocean ridges {[}\sphinxcite{references:id278}{]} and trenches {[}\sphinxcite{references:id277}{]} respectively. Better still, the pattern of geothermal fluxes aligns almost exactly with what would be expected if Earth’s surface plates represented the upper limbs of mantle\sphinxhyphen{}scale convection cells {[}\sphinxcite{references:id276}{]}. All these virtues explain why, from the mid\sphinxhyphen{}20th century onward, the model of global plate movement driven by mantle convection has served as a unifying explanatory paradigm for all geoscience.


\subsection{Challenges to the conventional model of plate tectonics}
\label{\detokenize{content/chapter_01_background/main:challenges-to-the-conventional-model-of-plate-tectonics}}
\sphinxAtStartPar
While the essential logic of convection\sphinxhyphen{}driven plate motion as a model for the Earth is sufficiently robust as to be beyond serious doubt, the scope of uncertainty at finer levels is daunting.


\subsubsection{The Naive Theory of plate tectonics}
\label{\detokenize{content/chapter_01_background/main:the-naive-theory-of-plate-tectonics}}
\sphinxAtStartPar
To illustrate how our understanding of global geodynamics has evolved, it is helpful to clearly articulate a ‘Naive Theory’ of plate tectonics to serve as a null hypothesis for the discussion. According to the Naive Theory:
\begin{itemize}
\item {} 
\sphinxAtStartPar
Earth is in a state of whole\sphinxhyphen{}mantle convection driven mainly by accretionary heat escaping from the core.

\item {} 
\sphinxAtStartPar
Material and thermal heterogeneities, while certainly present, do not determine the global convective planform.

\item {} 
\sphinxAtStartPar
Global convection is powered by the thermal gradient between the deep Earth and outer space, and should be expected to dwindle as this heat reservoir is depleted. Hence, we would expect elevated global geothermal fluxes in the past relative to the present.

\item {} 
\sphinxAtStartPar
Plate motions represent the upper limb of these whole\sphinxhyphen{}mantle cells, with the plates themselves in some sense propelled by the overall mantle current.

\item {} 
\sphinxAtStartPar
Mantle\sphinxhyphen{}scale upwellings should be expected to map to mid\sphinxhyphen{}ocean ridges and downwellings to subduction zones. Implicitly, plate scale forces such as ridge push, basal traction, and slab pull should cooperate to transport the plates according to whole\sphinxhyphen{}mantle circulation, and ought to be of similar magnitudes.

\end{itemize}

\sphinxAtStartPar
In general, the ‘Naive Theory’ is how the solid Earth would be expected to behave if it was dominantly basally heated, subject to linear viscosity laws, and chemically homogeneous to a first\sphinxhyphen{}order estimation. Such a rheology is powerfully amenable to numerical analysis and has been extensively studied as a common end\sphinxhyphen{}member behaviour for all sorts of systems in a state of thermally\sphinxhyphen{}driven flow (this thesis treats with similar rheologies in Chapter 3 and Chapter 4). Indeed, the ‘Naive Theory’ serves as an excellent approximation for systems ranging from Earth’s atmosphere to Campbell’s tomato soup.


\subsubsection{Insufficiencies of the Naive Theory}
\label{\detokenize{content/chapter_01_background/main:insufficiencies-of-the-naive-theory}}
\sphinxAtStartPar
However, when applied to the bulk Earth, the Naive Theory fails in nearly every respect, even as a generalisation. For one, it massively overestimates the contribution of ridge push to plate motions: slab pull is acknowledged as by far the dominant factor {[}\sphinxcite{references:id272}{]}. It also understates the apparently very substantial role played by chemical heterogeneities, most obviously vis a vis continental crust and the supercontinent cycle {[}\sphinxcite{references:id257}, \sphinxcite{references:id629}{]}, but also, increasingly, in respect to the hypothesised basal chemical heterogeneities at the core\sphinxhyphen{}mantle boundary {[}\sphinxcite{references:id270}, \sphinxcite{references:id438}{]}, which some suspect to be integral to the global convective planform {[}\sphinxcite{references:id365}, \sphinxcite{references:id265}{]}.

\sphinxAtStartPar
More generally, upwellings and downwellings do not always correspond with plate boundaries, are not in general symmetric, and can range in size from mantle\sphinxhyphen{}scale to lithospheric scales. With respect to downwellings, these can be plane\sphinxhyphen{}symmetric and aligned with (collisional) plate boundaries, as seen in subduction zones {[}\sphinxcite{references:id266}{]}, but may or may not be mantle scale depending on plate strength and upper mantle temperatures {[}\sphinxcite{references:id274}, \sphinxcite{references:id273}{]}; there is also a significant role played by delamination and dripping, which can occur on a range of scales and almost always remote from plate boundaries {[}\sphinxcite{references:id269}{]}. Upwellings, too, can be both mantle\sphinxhyphen{}scale and smaller\sphinxhyphen{}scale. ‘Hotspots’, as seen beneath Hawaii and East Africa, may fall into either category, depending on the interpretation of the tomographic and geochemical evidence. When major upwelling zones coincide with plate boundaries, it is strictly in the form of plumes, as seen in Iceland and East Africa {[}\sphinxcite{references:id271}{]}, with any planiform upwelling beneath the mid\sphinxhyphen{}ocean ridge being a strictly plate\sphinxhyphen{}instigated, shallow phenomenon {[}\sphinxcite{references:id268}{]}.


\subsubsection{Naive Theory and the Urey Paradox}
\label{\detokenize{content/chapter_01_background/main:naive-theory-and-the-urey-paradox}}
\sphinxAtStartPar
One sense in which the Naive Theory is surprisingly apt is in its minimisation of the role of radiogenic heating relative to secular cooling as a driver of convection in the mantle. The proportion of surface heat flux accounted for by radiogenic decay in the mantle is called the Urey Ratio {[}\sphinxcite{references:id88}{]} and it was originally supposed to be close to unity: that is, the Earth releases its heat as rapidly as it is produced {[}\sphinxcite{references:id399}{]}. This assumption was required to circumvent the so\sphinxhyphen{}called ‘Urey paradox’, in which unrealistic temperatures are obtained in the early Earth if today’s geothermal flux is mostly accounted for by secular cooling. However, multiple lines of evidence \sphinxhyphen{} including estimates of Earth’s original radiogenic inventory based on chondritic compositions {[}\sphinxcite{references:id457}{]}, as well as state\sphinxhyphen{}of\sphinxhyphen{}the\sphinxhyphen{}art geoneutrino surveys {[}\sphinxcite{references:id264}, \sphinxcite{references:id386}, \sphinxcite{references:id768}, \sphinxcite{references:id401}{]} \sphinxhyphen{} now agree on a much lower Urey ratio, somewhere in the neighbourhood of one third {[}\sphinxcite{references:id402}, \sphinxcite{references:id401}{]}.

\sphinxAtStartPar
In one sense, a low Urey ratio is favourable for the Naive Theory, as it implies that basal heating dominates volumetric heating: a state of affairs that promotes vigorous long\sphinxhyphen{}wavelength convection {[}\sphinxcite{references:id291}, \sphinxcite{references:id339}{]}. However, to avoid the Urey paradox, such a state of affairs would require an inverse relationship between global geothermal flux and net mantle temperatures \sphinxhyphen{} in other words, it seems that the Earth sheds its heat more rapidly the cooler it gets {[}\sphinxcite{references:id457}{]}. This counter\sphinxhyphen{}intuitive behaviour is hard to evoke with the kinds of isoviscous or exponentially temperature\sphinxhyphen{}dependent rheologies that the Naive Theory supposes.


\subsubsection{Going beyond the Naive Theory}
\label{\detokenize{content/chapter_01_background/main:going-beyond-the-naive-theory}}
\sphinxAtStartPar
We have employed the Naive Theory here mostly as a discursive device. However, though no theory of this form has ever been seriously advocated by qualified geodynamicists, the Naive Theory is a surprisingly close representation of common opinion regarding solid Earth dynamics, both among the lay public and in the non\sphinxhyphen{}specialist planetary and geoscientific literature. This can be seen, for instance, in the generally accepted supposition that Mars and the Moon lack active tectonics due to their lesser size, and hence \sphinxhyphen{} presumably \sphinxhyphen{} their more rapid rate of cooling. The resilience of Naive\sphinxhyphen{}like assumptions about the Earth is likely due in some degree to the fact that, as misguided as the theory is, its elegant formulation and superficially potent explanatory qualities still recommend it over alternative models, the underlying causal mechanisms of which remain elusive.

\sphinxAtStartPar
So, what might an improved, unified geodynamic model of the Earth entail? We can draw up a decent catalogue of what such a model would require by merely inverting the assumptions of our by now well\sphinxhyphen{}falsified null hypothesis:
\begin{itemize}
\item {} 
\sphinxAtStartPar
Earth is in a state of mixed convection on multiple interconnected wavelengths, with the dominant driver being negative buoyancy of the lithosphere.

\item {} 
\sphinxAtStartPar
Chemical and thermal heterogeneities are significant and have a first\sphinxhyphen{}order effect on global convective planform.

\item {} 
\sphinxAtStartPar
Instead of convective vigour being determined by global temperatures, temperatures are in fact determined by convective vigour, which has a range of complicated dependencies. Global heat flux can go inversely with global temperature and potentially exhibit many other counter\sphinxhyphen{}intuitive feedbacks.

\item {} 
\sphinxAtStartPar
Plate distributions powerfully condition global flow, and are themselves a function more of historical and local contingencies than of deep perturbations.

\item {} 
\sphinxAtStartPar
Mantle\sphinxhyphen{}driven upwellings and downwellings are to some degree separate from surface\sphinxhyphen{}driven movements, with the two systems at times reinforcing and at times disrupting one another.

\end{itemize}

\sphinxAtStartPar
What conceivable system, governed by what conceivable laws, could provide for these behaviours? Over the past two decades, significant progress has been made from various quarters to provide a new paradigm that satisfies these requirements without stipulating causes beyond the essentials of thermally driven flow. Two aspects of this paradigm will now be discussed: the discovery of tectonic modes, and the special role of continents.


\subsection{Continents: a decisive control in the tectonic system}
\label{\detokenize{content/chapter_01_background/main:continents-a-decisive-control-in-the-tectonic-system}}
\sphinxAtStartPar
Much of the progress of recent years is owed to the development of robust and versatile computational numerical modelling tools, which have proved to be particularly efficacious for investigating non\sphinxhyphen{}linear, and indeed, highly counter\sphinxhyphen{}intuitive systems {[}\sphinxcite{references:id383}{]}. The story they tell is one of a global flow regime controlled by chaotic interactions between high Rayleigh number instabilities and pronounced material heterogeneities: in particular, the strong, cold, low\sphinxhyphen{}density materials that make up the continents.


\subsubsection{Thermal effects of continental crust}
\label{\detokenize{content/chapter_01_background/main:thermal-effects-of-continental-crust}}
\sphinxAtStartPar
One major observation from recent geodynamic modelling is that a planet’s global flow regime needn’t necessarily be that which maximises heat flux {[}\sphinxcite{references:id459}{]}. A good example is the effect of continental crust on mantle temperatures. Intuition holds that the presence of thick continental lithosphere should insulate the mantle, leading to reduced net surface heat flux {[}\sphinxcite{references:id261}, \sphinxcite{references:id260}, \sphinxcite{references:id292}{]}. This feedback is often given a role to play in the periodic assembly and breakup of supercontinents, by triggering the formation of large sub\sphinxhyphen{}continental mantle plumes {[}\sphinxcite{references:id258}, \sphinxcite{references:id256}{]}.

\sphinxAtStartPar
Modelling, however, no longer supports this. Even in purely thermal models, supercontinentality does not necessarily lead to increased sub\sphinxhyphen{}continental temperatures {[}\sphinxcite{references:id263}{]}. In dynamic models, only fairly sluggish rheologies experience any meaningful sub\sphinxhyphen{}continental warming from insulation {[}\sphinxcite{references:id259}, \sphinxcite{references:id257}{]}; when geochemistry is considered, the effect of the partitioning of heat\sphinxhyphen{}producing elements into continents can be shown to substantially outweigh any insulation effect {[}\sphinxcite{references:id262}{]}. In fact, under more sophisticated rheologies, increased vigour due to transient warming {[}\sphinxcite{references:id676}{]} and geometric flow conditioning imposed by supercontinentality {[}\sphinxcite{references:id438}{]} may in fact boost the global surface flux. In this way, the apparent regularity of the supercontinent cycle \sphinxhyphen{} the first\sphinxhyphen{}order feature of Earth’s Phanerozoic history \sphinxhyphen{} could be accounted for purely by surface\sphinxhyphen{}driven forcings of the underlying mantle, without the need for catastrophic interventions like massive mantle plumes.


\subsubsection{Continents as a resolution of the Urey paradox}
\label{\detokenize{content/chapter_01_background/main:continents-as-a-resolution-of-the-urey-paradox}}
\sphinxAtStartPar
Continents may also provide a partial resolution to the Urey paradox. With some exceptions, estimations of the global heat budget over time have not explicitly taken into account the role of continents {[}\sphinxcite{references:id255}, \sphinxcite{references:id292}{]}. Numerical modelling suggests that mantle heat flow, to a first order approximation, is unaffected by historically typical degrees of continental coverage {[}\sphinxcite{references:id400}{]}. As higher continental abundances imply lesser radiogenic forcing in the mantle due to partitioning, the net result is a lower Urey ratio for much of Earth’s history. Feedbacks of this sort resolve the Urey paradox without the need for untestable heterogeneous mantle models {[}\sphinxcite{references:id402}{]}.


\subsubsection{Continent\sphinxhyphen{}climate feedbacks}
\label{\detokenize{content/chapter_01_background/main:continent-climate-feedbacks}}
\sphinxAtStartPar
Another under\sphinxhyphen{}appreciated role of continents in mantle convection comes indirectly by way of the atmosphere. Through albedo {[}\sphinxcite{references:id249}{]}, silicate weathering {[}\sphinxcite{references:id252}, \sphinxcite{references:id253}{]}, carbon storage {[}\sphinxcite{references:id251}{]}, and topographic effects {[}\sphinxcite{references:id254}, \sphinxcite{references:id248}{]}, continents have long been acknowledged as a potent factor in long\sphinxhyphen{}term global climate.

\sphinxAtStartPar
However, it is only recently that climate forcings on mantle flow have begun to be recognised. Numerical suite modelling shows that mantle flow can be suppressed as surface temperature rises {[}\sphinxcite{references:id513}{]}; conversely, modelling of subduction zones suggests the opposite coupling, as global warming enhances sediment runoff, lubricating subduction and possibly accelerating mantle overturn {[}\sphinxcite{references:id250}, \sphinxcite{references:id298}{]}. The net effect of climate may be so profound as to induce a complete switch in the style of tectonics {[}\sphinxcite{references:id367}, \sphinxcite{references:id674}{]}; the picture only becomes more complicated when the variegated feedbacks between continental drift, evolution, and bioclimatic controls are also taken into account {[}\sphinxcite{references:id342}{]}. Whether climate is a stabilising or destabilising agent in solid Earth circulation remains in question, as does the significance of tectonic\sphinxhyphen{}climate feedbacks to the maintenance of planetary habitability {[}\sphinxcite{references:id416}{]}.


\subsection{Tectonic modes: beyond plate tectonics}
\label{\detokenize{content/chapter_01_background/main:tectonic-modes-beyond-plate-tectonics}}
\sphinxAtStartPar
In the 1990s, the failure of linear mantle models to capture the major features of plate tectonics, even at extreme Rayleigh values {[}\sphinxcite{references:id381}{]}, motivated the development of non\sphinxhyphen{}linear, stress\sphinxhyphen{}dependent rheologies {[}\sphinxcite{references:id247}, \sphinxcite{references:id408}{]}. These models had the virtue of exhibiting strain\sphinxhyphen{}localisation, wherein the upper boundary layer separates into discrete units that can be generalised as plates {[}\sphinxcite{references:id246}, \sphinxcite{references:id385}{]}. The new models supported many interesting discoveries; most notably for our purposes, the discovery of the first non\sphinxhyphen{}trivial ‘tectonic mode’ to be observed outside of nature, the ‘episodic overturn’ mode {[}\sphinxcite{references:id759}{]}, now widely believed to be operative on Venus {[}\sphinxcite{references:id685}, \sphinxcite{references:id639}{]}. The discovery of episodic overturn gave credence to the importance of ‘phases’ in rheological parameter space; as applied to planets, these phases became known as ‘tectonic modes’ and are today a major topic of study. While some are fairly trivial, like the ‘stagnant’ regime, others are more surprising.


\subsubsection{Mobile lid}
\label{\detokenize{content/chapter_01_background/main:mobile-lid}}
\sphinxAtStartPar
When the convective vigour is much higher than the yield stress, a state is obtained where part or all of the lithosphere is failing at any given time. The ‘mobile lid’ category as it is presently defined is generous: it encompasses plate tectonics as well modes exhibiting very broad, or even uniform, lithospheric strains patterns.

\sphinxAtStartPar
Under an isoviscous rheology, a mobile tectonic mode persists in all but the most extreme cases {[}\sphinxcite{references:id392}{]}; nevertheless, there are subtleties to the mobile lid that are presented in depth in Chapter 3.


\subsubsection{Stagnant lid}
\label{\detokenize{content/chapter_01_background/main:stagnant-lid}}
\sphinxAtStartPar
When the yield stress is very high compared to the convective vigour of the mantle, a very strong, thick, static lithosphere may be formed. In a continuum mechanics sense, a stagnant lid could be characterised as any flow regime in which the maximum strain rate at the upper boundary layer is close to zero. More recently, authors have distinguished between a hot stagnant lid, as seen perhaps on the very early Earth {[}\sphinxcite{references:id717}, \sphinxcite{references:id441}{]}, and a cold stagnant lid, represented by the present\sphinxhyphen{}day Moon {[}\sphinxcite{references:id513}{]}.

\sphinxAtStartPar
The stagnant lid mode is a typical consequence of depth\sphinxhyphen{} or temperature\sphinxhyphen{}dependent rheologies such as Arrhenius\sphinxhyphen{}type viscosity {[}\sphinxcite{references:id380}{]}, which is the partial subject of Chapter 3.


\subsubsection{Episodic overturn}
\label{\detokenize{content/chapter_01_background/main:episodic-overturn}}
\sphinxAtStartPar
When the yield stress is nominal relative to convective vigour, a temporally unstable regime ensues in which long periods of lid stagnation are interspersed with abrupt failure events. Episodic overturn has a special place in tectonic mode theory: in models that reproduce a brittle lithosphere {[}\sphinxcite{references:id759}{]} in which a parameter implementing mechanical strength is the control variable, the episodic regime can be shown to occupy an intermediary position between stagnant and mobile lids. This observation demonstrates the feasibility of a ‘phase diagram’ type conception of planetary tectonic mode: testing the applicability of this paradigm is an ongoing and active field of research {[}\sphinxcite{references:id718}, \sphinxcite{references:id513}{]}. Venus is commonly considered a representative of this class based on the uniformity of its surface ages, around 300 My {[}\sphinxcite{references:id639}{]}.

\sphinxAtStartPar
Episodic overturn is the main subject of Chapter 4 of this thesis.


\subsubsection{Magma ocean}
\label{\detokenize{content/chapter_01_background/main:magma-ocean}}
\sphinxAtStartPar
In a magma ocean, the upper lithosphere is by definition in a state of global, continuous, contiguous failure: this is the definition of a fluid. Most magma ocean models have only a shallow ocean, and most are short\sphinxhyphen{}lived \sphinxhyphen{} indeed, so short as to be below the characteristic thermal timescale {[}\sphinxcite{references:id334}{]}. Hence, characterising the magma ocean as a tectonic mode is somewhat fraught. The Hadean eon of the Earth is so\sphinxhyphen{}called because of the assumption that a sustained magma ocean existed at this time {[}\sphinxcite{references:id696}{]}, and it is traditionally assumed that this is so for all the rocky planets {[}\sphinxcite{references:id354}{]}, though mounting evidence of a ‘cool early Earth’ increasingly call this into question {[}\sphinxcite{references:id244}, \sphinxcite{references:id245}{]}. The Moon and other small rocky bodies preserve the most unambiguous evidence of magma ocean tectonics, with the Moon’s highland\sphinxhyphen{}lowland dichotomy serving as the classic fossil example {[}\sphinxcite{references:id242}, \sphinxcite{references:id241}, \sphinxcite{references:id243}, \sphinxcite{references:id311}, \sphinxcite{references:id240}{]}.

\sphinxAtStartPar
The magma ocean, and the manner and timing of its demise, is an important consideration for models of the early Earth (Chapters 5 \& 6).


\subsubsection{Sluggish lid}
\label{\detokenize{content/chapter_01_background/main:sluggish-lid}}
\sphinxAtStartPar
If viscosity is only moderately temperature\sphinxhyphen{}dependent, a ‘sluggish’ regime can develop {[}\sphinxcite{references:id408}{]}, with large aspect\sphinxhyphen{}ratio convection cells leading to diffuse deformation zones in the upper boundary layer. Although the lid is technically mobile in this scenario, the motion is driven by mantle currents rather than vice versa, resulting in relatively tepid convection {[}\sphinxcite{references:id456}{]} and limited or strictly sub\sphinxhyphen{}lithospheric recycling {[}\sphinxcite{references:id269}{]}. While Earth is sometimes considered to have passed through a sluggish lid phase {[}\sphinxcite{references:id348}{]}, Mars is the oft\sphinxhyphen{}cited bellwether for this mode, based on its extensive deformed but ancient terrains {[}\sphinxcite{references:id237}{]}, isotopic analysis of Martian meteorites {[}\sphinxcite{references:id239}{]}, and numerical modelling {[}\sphinxcite{references:id238}{]}. A sluggish lid can be recognised by the existence of substantial toroidal flow in the upper boundary layer in the absence of significant poloidal flow.

\sphinxAtStartPar
The possibility of a sluggish lid on the early Earth is contemplated in Chapters 5 and 6 of this thesis.


\subsubsection{Plume tectonics}
\label{\detokenize{content/chapter_01_background/main:plume-tectonics}}
\sphinxAtStartPar
If the defining characteristics of plate tectonics are horizontal motion, arcuate deformation, and surface\sphinxhyphen{}driven convection, plume tectonics represents the opposite: vertical motion, bulbous deformation, and basally\sphinxhyphen{}driven convection. Plume tectonics, also called ‘vertical tectonics’, preceded plate tectonics as the popular explanation for the global distribution and structure of major landforms {[}\sphinxcite{references:id230}{]}, and it remains popular with those unconvinced by a dominantly surface\sphinxhyphen{}driven global circulation regime {[}\sphinxcite{references:id231}, \sphinxcite{references:id232}, \sphinxcite{references:id501}{]}. The picture is complicated by the fact that the two substantially independent systems {[}\sphinxcite{references:id228}{]} nevertheless seem to not only co\sphinxhyphen{}exist but cooperate on the Earth, with plumes inducing, or being induced by, the formation of triple junctions at the surface {[}\sphinxcite{references:id235}{]}, a process which apparently goes back to the Archaean {[}\sphinxcite{references:id234}{]}. Plume tectonics is also connected with plate tectonics in that it is potentially the source for much of Earth’s early continental crust {[}\sphinxcite{references:id227}, \sphinxcite{references:id429}{]} via a process which may be ongoing today; consider the oceanic plateaux, such as the Kerguelen and Azores, which arguably represent Phanerozoic proto\sphinxhyphen{}continents {[}\sphinxcite{references:id490}, \sphinxcite{references:id705}, \sphinxcite{references:id706}{]}. It has lately been hypothesised that plume tectonics not only preceded plate tectonics on the Earth {[}\sphinxcite{references:id743}{]} but actively gave rise to it {[}\sphinxcite{references:id233}{]}, by means of any one of numerous modelled mechanisms {[}\sphinxcite{references:id581}, \sphinxcite{references:id366}{]}.

\sphinxAtStartPar
Controversies aside, plume or vertical tectonics is a vital part of the emerging tectonic mode paradigm. Beyond Earth, plume tectonics is the preferred explanation for the Tharsis terrain on Mars {[}\sphinxcite{references:id220}, \sphinxcite{references:id219}, \sphinxcite{references:id226}, \sphinxcite{references:id218}, \sphinxcite{references:id236}{]}, where it appears to have been localised, massive, and extremely long\sphinxhyphen{}lived {[}\sphinxcite{references:id504}{]}; it is also implicated by various features on the surface of Venus including coronae and nova {[}\sphinxcite{references:id703}, \sphinxcite{references:id222}, \sphinxcite{references:id223}, \sphinxcite{references:id221}, \sphinxcite{references:id224}, \sphinxcite{references:id225}{]}.

\sphinxAtStartPar
The relationship between plume tectonics and plate tectonics, and the possibility and mechanisms of a plume\sphinxhyphen{}to\sphinxhyphen{}plate transition, are the main subject of Chapters 5 and 6 of this thesis.


\subsubsection{Heat pipe}
\label{\detokenize{content/chapter_01_background/main:heat-pipe}}
\sphinxAtStartPar
The heat pipe model was first postulated to explain certain observations of Jupiter’s moon Io {[}\sphinxcite{references:id407}{]}, but it wasn’t until more recently that it attained a rigorous thermodynamic treatment {[}\sphinxcite{references:id181}{]}. Today it is recognised as a distinct tectonic mode \sphinxhyphen{} arguably the second truly non\sphinxhyphen{}trivial mode to be discovered after episodic overturn {[}\sphinxcite{references:id647}{]}. In this model, a very strong, thick, cold lithosphere overlies a hot interior; long\sphinxhyphen{}lived plumes generate persistent volcanism on the surface, which continually re\sphinxhyphen{}buries the vacuum\sphinxhyphen{}chilled planet surface and so cools the interior by a process of pressure\sphinxhyphen{}driven downward cold advection {[}\sphinxcite{references:id391}{]}. Heat pipe tectonics is even more efficient at disbursing internal heat than plate tectonics \sphinxhyphen{} Io’s surface heat flux is estimated at an incredible 40 times that of Earth {[}\sphinxcite{references:id505}{]}.

\sphinxAtStartPar
The heat pipe mode is one resolution to the problematic jigsaw puzzle of the early Earth heat budget {[}\sphinxcite{references:id404}, \sphinxcite{references:id767}{]}; though it has yet to be convincingly replicated in numerical modelling, it remains a hot topic in the literature.


\subsubsection{Null mode}
\label{\detokenize{content/chapter_01_background/main:null-mode}}
\sphinxAtStartPar
In the interest of completeness, the case of a planet thoroughly at equilibrium with the temperature of free space should be mooted. Such a mode, here called the ‘null mode’, is a perfect attractor in parameter space upon which all planets must eventually converge if left to their own devices. Dwarf planets such as Ceres likely approximate the null mode in our solar system; if the Earth were set adrift in interstellar space, it would reach such a state on the order of 100 Gy from now {[}\sphinxcite{references:id474}{]}.


\subsubsection{The future of tectonic modes}
\label{\detokenize{content/chapter_01_background/main:the-future-of-tectonic-modes}}
\sphinxAtStartPar
The discovery of the episodic overturn and heat pipe modes makes it almost certain that more unseen modes remain to be uncovered. While limitless novelty can be generated with more and more complex material configurations, the real test of the paradigm will be to show that single\sphinxhyphen{}material, mathematically parsimonious laws, like the viscoplastic formulation that enables episodic behaviour, can continue to generate substantially original behaviours that can be called truly new tectonic modes.


\subsection{Tectonic mode transitions and planetary memory}
\label{\detokenize{content/chapter_01_background/main:tectonic-mode-transitions-and-planetary-memory}}
\sphinxAtStartPar
So far, we have encountered mode transitions as diagnostic or taxonomic classifications of viable systems for processing geothermal heat. This is a geophysical analogue of how weather and climate may be understood fundamentally as systems for processing solar heat. According to this manner of thinking, planets should be expected to ‘select’ the most efficient system for processing their heat that is made possible by their circumstances at any given time {[}\sphinxcite{references:id625}{]}.

\sphinxAtStartPar
We have already seen that a planet’s geothermal flux must not be naively expected to decrease as a function of time, or even as a function of temperature: both may indeed have the opposite effect. However, that does not invalidate the null hypothesis that the flux should be the highest possible.

\sphinxAtStartPar
Consider a schematic tectonic history of the Earth {[}\sphinxcite{references:id349}{]}: first, a magma ocean blasted by accretionary heat, core separation, and short\sphinxhyphen{}lived radiogenics; then, as surface cooling drove the magma to depth, an Io\sphinxhyphen{}like heat\sphinxhyphen{}pipe phase; third, a sluggish plume\sphinxhyphen{}tectonic phase akin to Venus or perhaps Enceladus; and finally, with sufficient cooling yielding a sufficiently brittle lithosphere, the breakup of one plate into multiple plates and the dawn of plate tectonics. According to this history, geothermal flux starts high when global temperatures are high, dips through the single\sphinxhyphen{}plate era, then rises with the advent of subduction; at each stage, the tectonic mode is a direct and unitary consequence of the state variables of global relict and radiogenic heat. This is what we will call the ‘phase diagram’ model of tectonic mode transitions, wherein a planet tracks a path across a multidimensional tectonic phase space {[}\sphinxcite{references:id675}, \sphinxcite{references:id102}, \sphinxcite{references:id498}{]} just as a block of water ice is compelled by its state variables of temperature and pressure to cross the liquid domain on its way to being vapour.

\sphinxAtStartPar
This way of viewing tectonic modes is increasingly challenged by geodynamic numerical modelling. A number of groups have recently furnished evidence that a planet’s choice of tectonic mode is inherently multistable {[}\sphinxcite{references:id94}, \sphinxcite{references:id93}, \sphinxcite{references:id513}{]}, indeterminate {[}\sphinxcite{references:id435}{]}, remote from equilibrium {[}\sphinxcite{references:id513}{]}, and heavily influenced by stochastic {[}\sphinxcite{references:id367}{]}, contingent {[}\sphinxcite{references:id100}, \sphinxcite{references:id644}{]}, historical {[}\sphinxcite{references:id96}{]}, and circumstantial factors {[}\sphinxcite{references:id99}, \sphinxcite{references:id101}, \sphinxcite{references:id346}{]}.

\sphinxAtStartPar
These lines of evidence suggest a new model: one in which tectonic mode is treated not as a state variable or a mere product of state variables, but rather as a process variable \sphinxhyphen{} a configuration that flows through the system, like heat itself {[}\sphinxcite{references:id351}{]}. Under such a model, multiple tectonic modes could be seen as coequal simultaneous possible configurations for each planet, each manifesting certain exigencies and instabilities that could spontaneously and unpredictably precipitate a transition into an alternative mode. Thus two planets identical in every way may, in sufficient time, evolve to be more and more unalike, until the ultimate logic of heating and cooling forces both to converge on stagnation and eventual, frigid death.

\sphinxAtStartPar
Though the new model is not yet well quantified, there are many facts we have already seen that should predispose us to suspect it may be true. Consider again the analogy with weather and climate. Non\sphinxhyphen{}linearities and chaotic behaviour are as influential there as it is here claimed they are in the solid Earth. However, there are two key differences. Firstly, planetary tectonic mode not only processes its heat source, but feeds back into it by determining the rate at which it is conveyed to space. On Earth, we see evidence of this effect most definitively in the lower\sphinxhyphen{}than\sphinxhyphen{}expected modern Urey ratio, which implies that Phanerozoic surface processes have in fact created a much more efficient channel for extracting heat from the core than existed prior. Secondly, planets have long\sphinxhyphen{}term memories. While the oldest climate information with direct relevance to the modern atmosphere is \textasciitilde{}200 My old carbon, Earth’s tectonic processes can be affected by circumstances billions of years old: for instance, when a buoyant Archaean craton impacts a subduction zone and forces a reconfiguration of plate stresses or even plate boundaries {[}\sphinxcite{references:id581}{]}.

\sphinxAtStartPar
While ultimately it may be tempting to cut through the uncertainty and hysteresis and return to the basic intuition that the Earth is becoming thermally ‘older’ with time, with less net heat to go around and hence less prospect for activity, it must again be stressed that the actual rates of flux, even in our elevated present regime, are very small with respect to tectonic timescales. Even a seemingly irreversible process like the formation of cratons on Earth is conceptually not undoable: as we calculated earlier, sufficient potential energy remains inside the Earth to effect the ‘rewinding’ of that particular degree of freedom. In principle, it might even be possible to establish a circle \sphinxhyphen{} or, more properly, a very shallow spiral \sphinxhyphen{} of alternating tectonic modes, which, if sufficiently regular, might be construable as a distinctive tectonic mode of its own: consider episodic overturn {[}\sphinxcite{references:id759}{]}. Such a cycle, despite the inevitable ratcheting of entropy, could nonetheless conceivably be stable even on timescales comparable to the present age of our universe.


\section{Review}
\label{\detokenize{content/chapter_01_background/main:review}}
\sphinxAtStartPar
The emerging model of tectonic modes will demand novel and imaginative approaches. This study’s particular methodology will be discussed forthwith (Chapter 2); for now, we must conclude with a reaffirmation of the essential thesis. The age of planets as dumb rocks mechanistically unspooling into space has come to an end. Felled by a torrent of empirical evidence at home and far beyond, that conception is now being replaced with a new one: broader, richer, more ripe with possibilities than what came before; but also challenging, subtle, and bewilderingly vast. It is to be hoped that this new paradigm will in time yield new tools; tools which, if wielded with boldness and ingenuity, may open up the tomb of Earth’s ancient past, and lay bare the pathways of life.


\chapter{Tools and Methods}
\label{\detokenize{content/chapter_02_methods/intro:tools-and-methods}}\label{\detokenize{content/chapter_02_methods/intro::doc}}
\sphinxAtStartPar
Under the operating paradigm of our research program, the phenomena of tectonics as we know them are held to be the surface expressions of a global system of solid\sphinxhyphen{}state convection. Tectonics \sphinxhyphen{} literally the ‘construction’ of Earth and other planets’ surface geology on a global scale \sphinxhyphen{} will thus be studied by proxy of the mantle\sphinxhyphen{}scale circulations which are inferred to motivate it. Because the theory justifying this substitution mandates certain controls, assumptions, and simplifications which must be accommodated by our modelling methodology, we will discuss these matters a little here.


\section{The analytical toolkit}
\label{\detokenize{content/chapter_02_methods/section1:the-analytical-toolkit}}\label{\detokenize{content/chapter_02_methods/section1::doc}}
\sphinxAtStartPar
Tectonics is known to us through its sensible processes of orogeny, seismicity, and volcanism. The energy available to carry out these permutations ultimately derives from the depletion of the thermal gradient of the Earth’s hot interior with space, mitigated to an uncertain degree by internal heat production via radiogenics, core despinning, and other means. Estimates of global heat flow vary from around \(42\) terawatts {[}\sphinxcite{references:id85}{]} to upwards of \(47\) terawatts {[}\sphinxcite{references:id108}{]}. Of this power, a mere \(1\%\) is thought to be necessary to account for all the geological activity witnessed on Earth {[}\sphinxcite{references:id88}{]}; if our Earth is a heat engine, it is a weak one.


\subsection{The Nusselt number}
\label{\detokenize{content/chapter_02_methods/section1:the-nusselt-number}}
\sphinxAtStartPar
A geodynamically rigid planet with Earth’s interior temperature would not be able to access even these modest energies: it would be trapped by its flat, linear conductive geotherm. That the planetary geotherm is evidently much greater than this is evidence that more kinetic processes are at work. The dimensionless temperature gradient is related to the Nusselt number or \(Nu\), the ratio of the measured temperature gradient to the reference gradient, which is the purely conductive geotherm. It can be given in terms of the rate of change of the dimensionless potential temperature \(\theta^*\) with respect to dimensionless depth \(y^*\) {[}\sphinxcite{references:id89}{]}:
\begin{equation*}
\begin{split} Nu = 1 + \left| \frac{\partial \theta^*}{\partial y^*} \right| _S \end{split}
\end{equation*}
\sphinxAtStartPar
Where \(|x|_S\) indicates the average value across a surface. The asterisks indicate a non\sphinxhyphen{}dimensionalised quantity: this is a convention throughout the literature.

\sphinxAtStartPar
When dimensionless parameters are used \sphinxhyphen{} unit mantle thickness and unit temperature range \sphinxhyphen{} the conductive geotherm for a non\sphinxhyphen{}curved domain is exactly one. Hence, for square geometries, the dimensionless temperature gradient \(Nu\) does double\sphinxhyphen{}duty as the heat transfer efficiency: it is the factor by which heat transfer is greater than it would be under a scenario of pure conduction. For curved domains, where the outer length is greater than the inner length, the conductive geotherm is proportionately lesser as it is in a sense ‘stretched out’ across the circumference; letting \(f\) be the ratio of inner to outer lengths (either circumferential or radial), \(Nu\) in these cases diverges from the dimensionless temperature gradient by a factor of \(f\) for cylinders and \(f^2\) for shells. Though harder to measure in practice than in theory, it is implicit that Earth’s Nusselt number must be much greater than one; it is sometimes cited in the order of \(10\) {[}\sphinxcite{references:id84}{]}, which is characteristic of laminar (sub\sphinxhyphen{}turbulent) flow {[}\sphinxcite{references:id83}{]}.


\subsection{The Prandtl, Grashof, Reynolds, and Rayleigh numbers}
\label{\detokenize{content/chapter_02_methods/section1:the-prandtl-grashof-reynolds-and-rayleigh-numbers}}
\sphinxAtStartPar
If conduction is insufficient to explain Earth’s geotherm, another process is implicated, and that is free convection \sphinxhyphen{} buoyancy\sphinxhyphen{}driven advection of heat. The relative effectiveness of convection is a product of two further dimensionless quantities. The \sphinxstyleemphasis{Prandtl} number \(Pr\) takes the ratio of momentum diffusivity and thermal diffusivity:
\begin{equation*}
\begin{split} Pr \equiv \frac{\nu_r}{\kappa_r} = \frac{\frac{\mu_r}{\rho_r}}{\frac{k_r}{\rho_r c_{p_r}}} \end{split}
\end{equation*}
\sphinxAtStartPar
Where \(\mu\) is viscosity, \(\rho\) is density, \(k\) is thermal conductivity, \(c_p\) is specific heat, and the \(r\) suffix indicates a choice of reference value for variable quantities. The \sphinxstyleemphasis{Grashof} number \(Gr\), meanwhile, concerns the forces involved: it is the ratio of buoyancy to viscous drag.

\sphinxAtStartPar
Without a sufficient \sphinxstyleemphasis{Prandtl} number, heat will escape from each parcel faster than the parcel itself can be transported by buoyancy, while a low \sphinxstyleemphasis{Grashof} number would imply that the drag of the medium on each parcel is too great for buoyancy to overcome.

\sphinxAtStartPar
These co\sphinxhyphen{}equal terms multiplied give us a third and final dimensionless quantity: the \sphinxstyleemphasis{Rayleigh} number \(Ra\) or ‘convective vigour’, which is more strictly interpreted as the ratio of the diffusive and convective time scales in the medium; i.e. \(Ra\) serves as the \sphinxstyleemphasis{Peclet} number for heat. For high values of \(Ra\), convection is much more efficient than conduction for transporting heat, leading to high fluid velocities and flow regimes grading from sluggish to laminar to turbulent. For low \(Ra\), conduction dominates, and the material is largely or totally quiescent. Separating these two domains is an often empirically\sphinxhyphen{}obtained value, the Critical Rayleigh Number or \(Ra_{cr}\), which is innate to each fluid; \(Ra\) is sometimes given in terms of \(Ra_{cr}\) as \(r = \frac{Ra}{Ra_{cr}}\). Values of \(Ra\) in most applications can be quite high, and so are usually represented in decimal orders of magnitude; for mantle materials as modelled hereafter, for example, the critical \(Ra\) can be shown to be somewhere between \(10^3-10^4\), with ‘Earthlike’ behaviour scarcely manifest anywhere below \(10^7\) (Chapter 3). Although \(Ra_{cr}\) is often obtained empirically, it can be derived from first principles for certain simple cases, as will be shown.

\sphinxAtStartPar
The \sphinxstyleemphasis{Rayleigh} number is a powerful tool for interrogating the behaviours of convecting fluids; however, the correct parameterisation of such a heavily compound term is a nuanced affair. Several assumptions are commonly made in the context of mantle circulation which simplify matters at the cost of limiting the scope of validity.

\sphinxAtStartPar
The ‘infinite \sphinxstyleemphasis{Prandtl}’ assumption asserts that momentum diffusivity is incomparably greater than thermal diffusivity; i.e.:
\begin{equation*}
\begin{split} \nu_r >> \kappa_r \end{split}
\end{equation*}
\sphinxAtStartPar
This is a defensible assumption for the Earth, where the estimated value of the Prandtl number is in fact around \(10^{23}\) {[}\sphinxcite{references:id89}{]}. Implied by the above, but worth stating clearly, is that the \sphinxstyleemphasis{Reynolds} number of the system \sphinxhyphen{} the ratio of inertial to viscous forces \sphinxhyphen{} approaches zero: i.e. inertia is negligible, present velocity is independent of previous velocity, and turbulence is consequently impossible. (This follows because the thermal \sphinxstyleemphasis{Peclet} number, which is \(Ra\), must be a product of \sphinxstyleemphasis{Pr} and \sphinxstyleemphasis{Re}; hence, to be finite, its expression in terms of \sphinxstyleemphasis{Pr} and \sphinxstyleemphasis{Re} must cancel at the limit.)

\sphinxAtStartPar
The infinite \sphinxstyleemphasis{Prandtl} statement is often taken in tandem with the \sphinxstyleemphasis{Boussinesq} approximation, which neutralises all density\sphinxhyphen{}driven force terms which are not coefficients of gravity; in other words, the fluid is held to be incompressible:
\begin{equation*}
\begin{split} \frac{\partial u}{\partial x} + \frac{\partial v}{\partial y} = 0 \end{split}
\end{equation*}
\sphinxAtStartPar
Where \(u\) and \(v\) connote horizontal and vertical velocity components respectively. The incompressibility assumption in two dimensions allows us to define a stream function \(\Psi(x, y)\):
\begin{equation*}
\begin{split} u = \frac{\partial \Psi}{\partial y}, \quad v = -\frac{\partial \Psi}{\partial x} \end{split}
\end{equation*}\begin{equation*}
\begin{split} \overline{u} = \nabla \Psi \end{split}
\end{equation*}
\sphinxAtStartPar
Where \(u\) is the velocity vector and \(\nabla\) is the familiar vector differential operator \(nabla\) or \(del\).

\sphinxAtStartPar
The stream function has many useful properties: lines of constant \(\Psi\) are called streamlines and are everywhere parallel to the velocity vector at that point, and a difference in value between any two points defines the volumetric flux across a line connecting those points, or equivalently the advective flux when multiplied by density \(\rho\). (The absolute value of \(\Psi\), however, is arbitrary.)

\sphinxAtStartPar
In addition to the \sphinxstyleemphasis{Boussinesq} and infinite \sphinxstyleemphasis{Prandtl} assumptions, we may further assert that the gravity is always radial and varies only with depth, and also that the fluid is inelastic, i.e. it has no stress memory. Together these several approximations hold wherever a dense, viscous fluid is subject to extreme pressures over relatively large spatio\sphinxhyphen{}temporal scales; hence they are held to be broadly appropriate for mantle problems, with some caveats.

\sphinxAtStartPar
With the aid of this toolkit of assumptions, together with the constitutive equations for conservation of mass and energy, it is possible to obtain velocity and pressure solutions for the otherwise insoluble \sphinxstyleemphasis{Navier\sphinxhyphen{}Stokes} equations: the conservation of momentum equations for viscous fluids. The derivation for mantle problems is canonical but lengthy; details can be found in the universally cited textbook literature on the topic {[}\sphinxcite{references:id89}, \sphinxcite{references:id88}{]}. One extremely useful product, however, is a family of robust parameterisations of the \sphinxstyleemphasis{Rayleigh} number for mantle convection. For a basally\sphinxhyphen{}heated system, the following holds:
\begin{equation*}
\begin{split} Ra \equiv \frac{\alpha_r \Delta T_r \rho_r^2 g_r b^3 c_{p_r}}{\mu_r k_r} \end{split}
\end{equation*}
\sphinxAtStartPar
Where \(\alpha\) is the coefficient of thermal expansion and \(T\) is the temperature drop across the layer thickness \(b\). It may be convenient instead to take the dynamic viscosity \(\nu=\frac{\mu}{\rho}\) instead, in which case:
\begin{equation*}
\begin{split} Ra = \frac{\alpha \Delta T \rho g b^3 c_p}{\nu k} \end{split}
\end{equation*}
\sphinxAtStartPar
If the system is internally rather than basally heated \sphinxhyphen{} e.g. as a result of radiogenic heating from uniformly distributed isotopes across the mantle \sphinxhyphen{} the following instead obtains:
\begin{equation*}
\begin{split} Ra_H \equiv \frac{\alpha_r \rho_r^3 g_r b^5 c_{p_r} H_r}{\mu_r k_r^2} \end{split}
\end{equation*}
\sphinxAtStartPar
Or, again, in terms of dynamic viscosity:
\begin{equation*}
\begin{split} Ra_H = \frac{\alpha g \rho H b^5}{k \kappa \nu} \end{split}
\end{equation*}
\sphinxAtStartPar
Where \(H\) is the heating in terms of power per mass. (The expression for \(Ra_H\) relates to that for \(Ra\) by a factor of \(\frac{b^2 H \rho}{k}\), for reasons that will later become clear.)

\sphinxAtStartPar
Of course, in reality, convection in the mantle is driven by both volumetric and basal heating. Unfortunately, there is not yet a universally accepted derivation for such a ‘mixed\sphinxhyphen{}heating \sphinxstyleemphasis{Rayleigh} number’, but one approach which has the virtue of simplicity takes the basally\sphinxhyphen{}heated \(Ra\) expression and adds a coefficient which may be interpreted as a non\sphinxhyphen{}dimensional \(H\) term:
\begin{equation*}
\begin{split} H = \frac{\rho H^* b^2}{k \Delta T^*} \end{split}
\end{equation*}
\sphinxAtStartPar
Where H* is the specific internal heating rate in W/kg and T* is the dimensionless temperature drop. Because the expression is derived as the quotient of RaH and \(Ra\), the new H term may simply be provided as a coefficient of the basally\sphinxhyphen{}heated \(Ra\) derivation {[}\sphinxcite{references:id81}, \sphinxcite{references:id89}{]}.

\sphinxAtStartPar
Because the \sphinxstyleemphasis{Rayleigh} number so expressed is equivalent to the coefficient of the buoyancy term, it should now be clear why it is often simply dubbed ‘convective vigour’, as that is its primary effect. By parameterising the system in this way, the behaviour of seemingly distinct scenarios can be seen to be related through their common \sphinxstyleemphasis{Rayleigh} number; what’s more, a dimensionless treatment of the problem can be readily converted to a dimensionalised one by expanding the terms of \(Ra\) with their empirical or inferred values.


\subsection{Linear stability analysis and the critical \sphinxstyleemphasis{Rayleigh} number}
\label{\detokenize{content/chapter_02_methods/section1:linear-stability-analysis-and-the-critical-rayleigh-number}}
\sphinxAtStartPar
It was hitherto given that the critical Rayleigh number, below which convection is not possible, is typically obtained empirically. In fact, for simple cases such as this of planar basally\sphinxhyphen{}heated isoviscous flow, an expression for Racr due to arbitrary perturbations can be derived from the assumptions already held using linear stability analysis. First consider the state of a purely conducting system at thermal equilibrium:
\begin{equation*}
\begin{split} T_c^* = \frac{T_0}{T_1 - T_0} + y^* \end{split}
\end{equation*}
\sphinxAtStartPar
Where \(T_c^*\) is the dimensionless conductive temperature at dimensionless depth \(y^*\); i.e. there is a linear dependency of temperature and depth.

\sphinxAtStartPar
Let us now impose a thermal anomaly \(\theta^{'*}\), uncertain in wavelength and infinitesimal in amplitude:
\begin{equation*}
\begin{split} \theta^{'*} \equiv T^{'*} - T_c^* \end{split}
\end{equation*}
\sphinxAtStartPar
Where the starred notation indicates a non\sphinxhyphen{}dimensionalised parameter and the prime notation, here and henceforth, identifies a perturbation. The choice of \(\theta\) here relates to potential temperature, the quantity conserved along adiabats, which is what this perturbation will ultimately induce.

\sphinxAtStartPar
Before perturbation, the pressure gradient forces were defined solely by the hydrostatic pressure \(p_c\) \sphinxhyphen{} the pressure field which is purely sufficient to counteract the force of gravity. After the introduction of the perturbation, but before the resultant perturbed state is realised, the pressure field is modified in two ways: by the buoyancy anomaly of the perturbation, but also by the contribution of the modified density of the parcel to pre\sphinxhyphen{}perturbative hydrostatic pressure. Taking this into account, we can define a true perturbation pressure \(\Pi^{'*}\) as:
\begin{equation*}
\begin{split} \Pi^{'*} \equiv p^{'*} - p_c^* \end{split}
\end{equation*}
\sphinxAtStartPar
Where \(p^{'*}\) is the pressure deviation relative to the hydrostatic pressure.

\sphinxAtStartPar
What determines if this seed of chaos shall grow? Equivalently, we may ask which is faster \sphinxhyphen{} the growth of the anomaly, or the ambient restoring forces. The answer depends in part on the wavelength of the perturbation and in part on the overall convective vigour of the system; a very lengthy expansion {[}\sphinxcite{references:id89}{]} reaches the sixth derivative before delivering the following relation:
\begin{equation*}
\begin{split} Ra_{cr} = \frac{\pi^4}{4\lambda^{*4}} \cdot \left( 4 + \lambda^{*2} \right) ^3 \end{split}
\end{equation*}
\sphinxAtStartPar
Where \(\lambda^*=\frac{\lambda}{b}\), the wavelength of perturbation equivalent to the original anomaly \(\theta^{'*}\) in the horizontal coordinate, expressed as a ratio of the layer thickness \(b\), and \(Ra_{cr}\) is what we came for: the ‘critical’ \sphinxstyleemphasis{Rayleigh} number above which perturbations of a given wavelength will grow more rapidly than they are diffused. The expression defines a curve through the space of \(Ra_{cr}\) vs dimensionless wavenumber which has a single minimum: this is \(\lambda^*_{cr}\), the wavelength of perturbation at which \(Ra_{cr}\) is at its lowest. Perturbations near this critical wavelength will tend to grow the fastest, since, as it were, they experience the highest ‘local’ \sphinxstyleemphasis{Rayleigh} number. As it happens, this wavelength, and the minimum \(Ra\) it requires to grow, come to:
\begin{equation*}
\begin{split} Ra_{cr, \min} = \frac{27\pi^4}{4} \approx 6.57.5 \end{split}
\end{equation*}\begin{equation*}
\begin{split} \lambda_{cr}^* = 2 \sqrt{2} \approx 2.828 \end{split}
\end{equation*}
\sphinxAtStartPar
At first glance it might seem that we have not truly answered the question of what defines the critical Rayleigh number for a convecting system as a whole, but rather only a contingent answer depending on wavelengths of perturbation. Consider, though, the significance of driving the Rayleigh number below the minimum critical value. This is equivalent to stating that no perturbations at all \sphinxhyphen{} not even the least stable ones \sphinxhyphen{} are able to grow quicker than the diffusive timescale. At the minimum critical value itself, it follows that only perturbations of \(\sqrt{2}\) scale will grow; this value nonetheless serves adequately as the \(Ra_{cr}\) of the entire fluid, since a perturbation of such a wavelength can always be discovered in any real system \sphinxhyphen{} if geometry permits.

\sphinxAtStartPar
Having determined the conditions under which the conductive planform becomes unstable, it behooves us to establish what the new stability criterion must be which now the system seeks. Assuming that the fastest\sphinxhyphen{}growing perturbation will ultimately come to dominate all others, what we need is an expression for the velocity field in terms of \(\lambda\) that we can solve for the critical wavelength \(\lambda_{cr}\) {[}\sphinxcite{references:id76}{]}.

\sphinxAtStartPar
First let us find the infinitesimal thermal anomaly in terms of perturbation wavelength, which must be a sinusoidal function in both \(y\) and \(x\):
\begin{equation*}
\begin{split} \theta^{'*} = \widehat{\theta}_0^{'*} \sin \left( \pi y^* \right) \sin \left( \frac{2 \pi x^*}{\lambda^*} \right) \end{split}
\end{equation*}
\sphinxAtStartPar
Where \(\widehat{\theta}_0^{'*}\) is the first term of the Fourier expansion of \(\theta^{'*}\) and provides the wave amplitude, which is arbitrary.

\sphinxAtStartPar
We can now take the stream function \(\Psi\) in terms of \(\theta^{'}\) and substitute:
\begin{equation*}
\begin{split} \Psi^* = - \left( \frac{\lambda^*}{2} \right) \left( \frac{4\pi^2}{\lambda^{*2}} + \pi^2 \right) \widehat{\theta}_0^{'*} \sin \left( \pi y^* \right) \cos \left( \frac{2 \pi x^*}{\lambda^*} \right) \end{split}
\end{equation*}
\sphinxAtStartPar
The contours of the stream function give the geometry of convection, which, for the critical \(\lambda\) in two dimensions, takes the form of pairs of counter\sphinxhyphen{}rotating half\sphinxhyphen{}cells of aspect \(\frac{\lambda_{cr}^*}{2}=\sqrt{2}\); in other words, the planform of convection at steady state for any basally\sphinxhyphen{}heated planar isoviscous system will tend to approach an aspect ratio with the approximate dimensions, in landscape, of the page this sentence is written on.


\subsection{Boundary layer theory and the \protect\(Ra-Nu\protect\) scaling}
\label{\detokenize{content/chapter_02_methods/section1:boundary-layer-theory-and-the-ra-nu-scaling}}
\sphinxAtStartPar
The \sphinxstyleemphasis{Rayleigh} number by itself is a powerful tool for controlling and dissecting mantle convection models \sphinxhyphen{} but it would be better still if \(Ra\), the chief input parameter of the convection, could be analytically related to the \sphinxstyleemphasis{Nusselt} number, the most important output parameter:
\begin{equation*}
\begin{split} Nu \propto f \circ Ra \end{split}
\end{equation*}
\sphinxAtStartPar
Because \(Nu\) uniquely measures what \(Ra\) uniquely shapes \sphinxhyphen{} the convective planform \sphinxhyphen{} we know that some function must connect these quantities. To characterise this mystery function, it will be necessary to take what we have learned and delve into the uncertain world of boundary layers.

\sphinxAtStartPar
We have already established that when the \sphinxstyleemphasis{Rayleigh} number is supercritical, heat may be more rapidly transported by advection than by diffusion. The effect of the ensuing convection is to deflect this conductive geotherm \(T_c^*\) towards the steeper ‘adiabatic geotherm’: the path in temperature\sphinxhyphen{}pressure space along which the potential temperature \(\theta\) \sphinxhyphen{} that which a parcel would achieve if brought to a standard reference pressure without gaining or losing any heat \sphinxhyphen{} is effectively constant. As the abiabat approaches the system boundaries, a point of diminishing returns is reached, and conductive processes take precedence once more. These regions of conductivity are the convecting system’s boundary layers.

\sphinxAtStartPar
It is possible to obtain an expression for the thickness of these boundary layers by considering the linear stability of just the layers themselves. First, we must determine the rate at which the conductive layer expands. This is complicated in the first instance by the fact that the actual layer thickness itself is hard to define in a continuum. Traditionally, however, it has sufficed to define it as the domain across which the first ten percent of temperature is gained or lost. Hence:
\begin{equation*}
\begin{split} y_T = 2 \eta_T \sqrt{\kappa t} \approx 2.32 \sqrt{\kappa t} \end{split}
\end{equation*}
\sphinxAtStartPar
Where \(y_T\) is the boundary layer thickness, \(\sqrt{\kappa t}\) is interpreted as the characteristic length scale of thermal diffusion \(\kappa\), and \(\eta_T\) is the inverse error function of \(0.1\), a constant term approximately equal to \(1.16\).

\sphinxAtStartPar
As the boundary grows, so do the thermal buoyancy forces. The relevant \sphinxstyleemphasis{Rayleigh} number to parameterise the vigour of the incipient convection is taken over the boundary layer thickness itself, and hence grows as the layer grows:
\begin{equation*}
\begin{split} Ra_{y_T} = \frac{\alpha \Delta T g y_T*3}{\nu \kappa} \end{split}
\end{equation*}
\sphinxAtStartPar
Where \(\alpha\) is the thermal expansivity and \(\nu\) is the dynamic viscosity \(\frac{\nu}{\rho}\).

\sphinxAtStartPar
Now what we are interested in is what the thickness of the boundary layer will be when the \sphinxstyleemphasis{Rayleigh} number defined over it, \(Ra_{y_T}\), is at its critical value, \(Ra_{y_T,crit}\). Below this value, convective disruption of the layer will not be possible, as any perturbations within the layer will be thermally diffused before they can grow; while above this value, convection is inevitable and the conductive profile of the layer cannot be sustained. The expression for \(Ra_{y_T,crit}\) is the same as that for \(Ra_{y_T}\), except that the temperature contrast \(T\) is half that of the system as a whole; this is because the dimensionless temperature change across either boundary layer goes from zero or unit at the outer edge to exactly \(0.5\) at the inner edge, where the layers face the tepid conditions of the intracellular fluid; so we write:
\begin{equation*}
\begin{split} Ra_{y_T,crit} = \frac{Ra_{y_T}}{2} \end{split}
\end{equation*}
\sphinxAtStartPar
And accordingly:
\begin{equation*}
\begin{split} y_T = \left\{ \frac{2 Ra_F \nu \kappa}{\alpha g \Delta T} \right\} ^{\frac{1}{3}} \end{split}
\end{equation*}
\sphinxAtStartPar
Where \(Ra_F\) is coined to refer to the minimum critical \sphinxstyleemphasis{Rayleigh} number across the layer as defined when that layer is at the brink of collapse.

\sphinxAtStartPar
At this point we might be tempted to define a general critical \sphinxstyleemphasis{Rayleigh} number for the layer by the same means we deduced one for the system as a whole previously. Unfortunately, the dynamic quality of the layer thickness \(y_T\) poses one unknown too many. For a boundary layer that is developing through time, it is not guaranteed that an appropriate perturbation of the appropriate scale will emerge at the appropriate moment, nor even that the geometry of the layer will ever be sufficient to permit such a perturbation in the first place. We have come as far as analytical methods can take us; to close the loop, it is necessary to obtain RaF empirically:
\begin{equation*}
\begin{split} y_T = \left\{ \frac{807 \nu \kappa}{\alpha g \Delta T} \right\} ^{\frac{1}{3}} \end{split}
\end{equation*}
\sphinxAtStartPar
Where the value \(807\) is the experimentally determined \(Ra_F\) for a free\sphinxhyphen{}slip surface {[}\sphinxcite{references:id78}{]}.

\sphinxAtStartPar
Now, because the thickness of a conductive layer is directly related to the thermal gradient across it, and thence to the Nusselt number \(Nu\), while the right side contains the coefficients of the global Rayleigh number \(Ra\), it finally becomes apparent what form the relationship between \(Nu\) and \(Ra\) should take:
\begin{equation*}
\begin{split} Nu = 0.112 Ra^{\frac{1}{3}}, \quad Ra_F = 807 \end{split}
\end{equation*}
\sphinxAtStartPar
Or more generally:
\begin{equation*}
\begin{split} Nu \propto Ra^\beta, \quad \beta \approx \frac{1}{3} \end{split}
\end{equation*}
\sphinxAtStartPar
That a scaling law of this form would obtain for two dimensionless flow constants such as these is not surprising; empirically, just such a relationship is in fact very widely attested {[}\sphinxcite{references:id80}, \sphinxcite{references:id408}, \sphinxcite{references:id79}{]}. Authors have differed, however, on the proper value of \(beta\). Though the canonicity of the analytically\sphinxhyphen{}derived value of one third is beyond dispute, it is clear from the divergent results of numerous studies that, in any real scenario, many more variables than we have accounted for must enter the equation. Time\sphinxhyphen{}dependence, long\sphinxhyphen{}lived thermal heterogeneities, aspect ratio, internal heating, and countless other factors all have a part to play. Obtaining robust scaling laws that account for all these factors is the vexing business of this thesis.


\subsection{Critical values for the internally\sphinxhyphen{}heated case}
\label{\detokenize{content/chapter_02_methods/section1:critical-values-for-the-internally-heated-case}}
\sphinxAtStartPar
What we have deduced so far is valid only for planar domains with basal heating. This will not suffice if our subject is the real Earth, which is both basally heated from the core and volumetrically heated throughout by radioactive decay.

\sphinxAtStartPar
Consider a convecting system with constant and uniform internal heating. Basal heat will be disregarded. For this analysis it will be necessary to prescribe that the basal boundary is insulating; in other words, while the upper boundary retains a \sphinxstyleemphasis{Dirichlet}\sphinxhyphen{}type fixed temperature condition, the lower boundary must be a \sphinxstyleemphasis{Neumann}\sphinxhyphen{}type condition of heat flux zero. In such a system, we cannot rely on the difference of basal and surface temperature for our linear stability analysis. Instead:
\begin{equation*}
\begin{split} \Delta T_r = \frac{b^2 H \rho}{k} \end{split}
\end{equation*}
\sphinxAtStartPar
Where, again, \(b\) is the layer thickness, \(H\) is the heating per mass, \(\rho\) is density, and \(k\) is conductivity. The new temperature scale is thus the factor by which temperature must be non\sphinxhyphen{}dimensionalised in this treatment. The conducting geotherm must take this into account, and can no longer be expected to be linear:
\begin{equation*}
\begin{split} T_c^* = \frac{T_0}{\Delta T_r} + y^* - \frac{y^{*2}}{2} \end{split}
\end{equation*}
\sphinxAtStartPar
We now recall the Rayleigh number for internally heated convection, as given previously:
\begin{equation*}
\begin{split} Ra_H = \frac{\alpha g \rho H b^5}{k \kappa \nu} \end{split}
\end{equation*}
\sphinxAtStartPar
Which, together with the conductive geotherm \(T_c^*\) provides:
\begin{equation*}
\begin{split} \frac{d p_c^*}{d y^*} = -Ra_H T_c^* \end{split}
\end{equation*}
\sphinxAtStartPar
I.e. the rate of change of the hydrostatic pressure with respect to depth. Unlike in the basally\sphinxhyphen{}heated case, the pressure here is given as dependent on the conductive temperature profile; previously, both temperature and hydrostatic pressure were necessarily linear with depth. From here the analysis proceeds much as in the basally\sphinxhyphen{}heated case, only to culminate in an insoluble ordinary differential equation {[}\sphinxcite{references:id89}{]} from which only empirical data can recover us:
\begin{equation*}
\begin{split} Ra_{H, cr, \min} = 867.8, \quad \lambda_{cr}^* = 3.51 \end{split}
\end{equation*}
\sphinxAtStartPar
{[}\sphinxcite{references:id77}{]}

\sphinxAtStartPar
In other words, for the onset of convection in an internally\sphinxhyphen{}heated system with basal\sphinxhyphen{}insulating, surface\sphinxhyphen{}isothermal, free\sphinxhyphen{}slip boundaries, the critical \sphinxstyleemphasis{Rayleigh} number and characteristic wavelength are both a little more than one quarter greater than for the equivalent basally\sphinxhyphen{}heated case.


\subsection{Chaos and attraction: approximate solutions to insoluble equations}
\label{\detokenize{content/chapter_02_methods/section1:chaos-and-attraction-approximate-solutions-to-insoluble-equations}}
\sphinxAtStartPar
The nature of convecting systems in practice ensures that even the simplest problems can be effectively or absolutely insoluble by analytical means. Though we will shortly outline methods for meeting these challenges experimentally, it is always unwise to go too far empirically whither mathematics cannot follow.

\sphinxAtStartPar
One means of probing beyond the insolubility barrier is to take an eigenmode expansion of the equations of state and discard all but the fewest number of terms which still support nonlinear interactions:
\begin{equation*}
\begin{split} \Psi^* = \frac{4 + \lambda^{*2}}{\sqrt{2}} A(\tau) \sin \left( \frac{2 \pi x^*}{\lambda^*} \right) \sin \left( \lambda y^* \right) \end{split}
\end{equation*}\begin{equation*}
\begin{split} \theta^* = \frac{1}{\pi r} \left[ C(\tau) \sin \left( 2 \pi y^* \right) - \sqrt{2} B(\tau) \cos \left( \frac{2 \pi x^*}{\lambda^*} \right) \sin \left( \pi y^* \right) \right] \end{split}
\end{equation*}
\sphinxAtStartPar
Where \(\Psi\) is again the stream function, \(x\) and \(y\) are coordinates, \(\lambda\) is the featural wavelength, \(r\) is the \sphinxstyleemphasis{Rayleigh} number as a proportion of the critical value \(r = \frac{Ra}{Ra_{cr}}\), and \(A(\tau)\), \(B(\tau)\), and \(C(\tau)\) are time\sphinxhyphen{}dependent coefficients which are functions of \(\tau\), time non\sphinxhyphen{}dimensionalised by wavelength:
\begin{equation*}
\begin{split} \tau = \pi^2 \left[ 1 + \frac{2}{\lambda^*} \sin \left( \pi y^* \right) \right] t^* \end{split}
\end{equation*}
\sphinxAtStartPar
The \(A\), \(B\), and \(C\) coefficients permit a powerful simplification in form. Selecting the appropriate equation from the infinite set contained in the eigenmode expansion {[}\sphinxcite{references:id89}{]}, the following first\sphinxhyphen{}order differential equations can be obtained:
\begin{equation*}
\begin{split} \frac{d A}{d \tau} = Pr \left( B - A \right) \end{split}
\end{equation*}\begin{equation*}
\begin{split} \frac{d B}{d \tau} = rA - B - AC \end{split}
\end{equation*}\begin{equation*}
\begin{split} \frac{d C}{d \tau} = -bc + AB \end{split}
\end{equation*}
\sphinxAtStartPar
Where \(Pr\) is the \sphinxstyleemphasis{Prandtl} number, which must be kept finite for this analysis, though it may still be arbitrarily large; \(b\) represents:
\begin{equation*}
\begin{split} b = \frac{4}{\left[ 1 + \left( \frac{2}{\lambda^*}^2 \right) \right]} \end{split}
\end{equation*}
\sphinxAtStartPar
These three are the Lorenz Equations {[}\sphinxcite{references:id75}{]}, for which solutions represent states of cellular 2D convection. Because they are severely truncated in form, their scope of validity is limited to low values of \(r\). Nevertheless, they are conceptually extremely useful for characterising the macro\sphinxhyphen{}scale character of mantle convection, particularly approaching the point of criticality.

\sphinxAtStartPar
Of the three functions, \(A\) relates to the stream function, \(B\) the resultant temperature variations, and \(C\) a horizontally averaged temperature mode. Three obvious solutions to the system are:
\begin{equation*}
\begin{split} A = B = C = 0 \end{split}
\end{equation*}\begin{equation*}
\begin{split} A = B = \pm \sqrt{b \left( r - 1 \right)} \end{split}
\end{equation*}\begin{equation*}
\begin{split} C = r - 1 \end{split}
\end{equation*}
\sphinxAtStartPar
When \(r<1\), the trivial first solution above describes the only stable steady\sphinxhyphen{}state solution and represents pure conduction, just as we would expect when \(Ra<Ra_{cr}\). When \(r>1\), this solution becomes unstable, and the only stable solutions become the positive and negative valencies of the second expression above, which represent clockwise and counterclockwise unicellular convection. The ‘choice’ of the system to devolve from the unbiased conductive solution to one of either the left\sphinxhyphen{} or right\sphinxhyphen{}biased convective solutions is termed a ‘pitchfork bifurcation’, the first of many we will encounter; its existence proves that mantle convection is chaotic.

\sphinxAtStartPar
The two convective solutions above have been shown to be stable \sphinxhyphen{} but are they necessarily steady? If we take our two primitive convective solutions further, a characteristic equation can be obtained from which we can derive the following special value of \(r\):
\begin{equation*}
\begin{split} r = \frac{Pr \left( Pr + b + 3 \right)}{Pr - b - 1} \end{split}
\end{equation*}
\sphinxAtStartPar
When \(Pr>b+1\), the above expression gives the value of \(r\) above which the two fundamental convective solutions are in fact not stable; in other words, it is the criterion for the instability of steady convection. It is also another kind of bifurcation \sphinxhyphen{} a \sphinxstyleemphasis{Hopf} bifurcation. Around a \sphinxstyleemphasis{Hopf} point, stable solutions are periodic and cyclical; solutions which cross the bifurcation are hence ‘captured’ by it and cycle through a finite set of states ad infinitum, until or unless those oscillations become great enough to tip a system into the zone of attraction of another \sphinxstyleemphasis{Hopf} point. Complex paths through phase space can thus be drawn which represent very high\sphinxhyphen{}order periodic solutions for the system that resist analytical description. Dubbed ‘strange attractors’, they are the iconic property of chaos theory.

\sphinxAtStartPar
So far, for the sake of argument, we have assumed a finite \sphinxstyleemphasis{Prandtl} number. This of course contravenes one of the foundational assumptions of our broader analysis. Before moving on, it behooves us to ask whether the chaotic behaviours observed in the Lorenz equations hold in the limit that \(Pr\to\infty\).

\sphinxAtStartPar
This would imply, first of all, that \(A=B\). Hence:
\begin{equation*}
\begin{split} \frac{d A}{d \tau} = \frac{d B}{d \tau} = \left( r - 1 \right) B - BC \end{split}
\end{equation*}\begin{equation*}
\begin{split} \frac{d C}{d \tau} = -bC + B^2 \end{split}
\end{equation*}
\sphinxAtStartPar
The fixed points of these new equations are the same as for the Lorenz equations, as is the conductive solution when \(A=B=C=0\), which as before is stable only for subcritical \(Ra\); however, the convective solutions can be shown to be stable for all \(r>1\). We might take this to imply that mantle convection cannot be chaotic after all. However, it must be recalled that the Lorenz analysis begins with severe truncation of non\sphinxhyphen{}linear terms. For higher\sphinxhyphen{}order truncations, it is evident that chaotic phases can exist {[}\sphinxcite{references:id89}{]}, particularly at high \sphinxstyleemphasis{Rayleigh} numbers; what is not certain is whether, for a given degree of truncation and a given range of parameters, chaotic behaviours will manifest for a particular system. When we attempt to engage with the problem numerically and empirically through modelling, which is the purpose of this thesis, it will be seen that certain parameter bands are chaotic and time\sphinxhyphen{}dependent while others are not; ultimately it will be argued that such zones of chaos represent boundaries in a very high\sphinxhyphen{}dimensional phase space, and relate fundamentally to the nature and proper characterisation of tectonic modes.


\section{Numerical methods and the Underworld code}
\label{\detokenize{content/chapter_02_methods/section2:numerical-methods-and-the-underworld-code}}\label{\detokenize{content/chapter_02_methods/section2::doc}}
\sphinxAtStartPar
When the font of analysis is exhausted, numerical solutions become indispensable. Though the construction of robust and accurate systems for the computation of mantle convection problems is far from trivial, many successful approaches have been developed over the years, each with its own advantages and limitations. For this thesis we adopt a finite\sphinxhyphen{}element approach, which has been extensively developed by many groups over more than a quarter of a century. In particular, we follow after Professor Moresi and colleagues, the developers of modelling codes including \sphinxstyleemphasis{CITCOM}, \sphinxstyleemphasis{Ellipsis}, and most lately \sphinxstyleemphasis{Underworld} {[}\sphinxcite{references:id71}, \sphinxcite{references:id73}, \sphinxcite{references:id72}, \sphinxcite{references:id552}, \sphinxcite{references:id384}, \sphinxcite{references:id383}, \sphinxcite{references:id381}, \sphinxcite{references:id544}, \sphinxcite{references:id385}{]}.

\sphinxAtStartPar
Here we will discuss in broad terms the principles and infrastructure of our numerical modelling practice; particulars of model design and construction will be discussed in further chapters as relevance dictates.


\subsection{Numerical methods}
\label{\detokenize{content/chapter_02_methods/section2:numerical-methods}}
\sphinxAtStartPar
To take a numerical approach to mantle convection is to endeavour to iteratively solve the following system of equations, which are the Stokes, conservation, and advection\sphinxhyphen{}diffusion equations under the assumptions of incompressibility and infinite Prandtl number:
\begin{equation*}
\begin{split} \nabla p - \nabla \left( \eta D \right) = \Delta \rho \overline{g} \end{split}
\end{equation*}\begin{equation*}
\begin{split} \nabla \cdot \overline{u} = 0 \end{split}
\end{equation*}\begin{equation*}
\begin{split} \frac{\partial T}{\partial t} + \overline{u} \cdot \nabla T = \kappa \nabla^2 T + H \end{split}
\end{equation*}
\sphinxAtStartPar
Where \(\eta\) is dynamic viscosity, \(D\) the strain rate tensor, \(p\) dynamic pressure, \(\Delta\rho\) the density anomaly, \(g\) the gravity vector, \(\overline{u}\) the velocity vector, \(T\) temperature, \(\kappa\) thermal diffusivity, \(t\) time, and \(H\) a thermal source term, i.e. radiogenic heating.

\sphinxAtStartPar
The first two equations together solve for velocity given buoyancy and viscosity without accounting for inertia (hence the lack of time\sphinxhyphen{}dependency), while the third equation governs determines the change in temperature at any given point after an infinitesimal time interval due to diffusion (the first term) and advection (the second). Temperature, being the only time\sphinxhyphen{}dependent quantity, is the only necessary variable of state for this system. It is further possible to construct the viscosity and diffusivity parameters as functions of space, time, temperature, velocity, et cetera, to achieve much more complex rheologies as desired; plastic, elastic, rigid, and insulating materials may all be implemented in this way, each with its own perils and caveats.

\sphinxAtStartPar
We have already established that the equations are generally insoluble; numerical methods are the only recourse. There are unavoidable tradeoffs in terms of time, memory, accuracy, flexibility, extensibility, and robustness associated with every numerical convection scheme. Many methods have been developed, including smooth particle hydrodynamics {[}\sphinxcite{references:id58}{]} and discrete elements {[}\sphinxcite{references:id57}{]}. We employ a finite elements approach {[}\sphinxcite{references:id69}{]} in which the domain is divided into a discrete network of cells (the ‘mesh’): all field values are hosted on the nodes, with integrals across elements approximated using Gauss quadrature {[}\sphinxcite{references:id61}{]}. Such an approach can of course yield only an approximation of what the underlying equations imply, but the approximation can be arbitrarily accurate depending on the fineness of discretisation. The question, then, is how coarse can the approximation be made without losing fidelity to the governing equations?

\sphinxAtStartPar
Computationally, the problem takes the form:
\begin{equation*}
\begin{split} A \overline{u} + B \overline{p} = \overline{f} \end{split}
\end{equation*}\begin{equation*}
\begin{split} B^T \overline{u} = 0 \end{split}
\end{equation*}
\sphinxAtStartPar
Where \(A\) is the known ‘stiffness’ matrix equivalent here to viscosity, \(\overline{u}\) is a vector of unknown velocities, \(B\) is the discrete gradient operator, \(\overline{p}\) contains the pressure unknowns, \(T\) indicates the transpose, and \(\overline{f}\) is the known vector of body and boundary forces acting on the material. The objective is to solve for \(\overline{u}\) as cheaply and reliably as possible; once a velocity solution is obtained, the system can be integrated in time and the cycle begins again. First, however, we require a solution for \(p\). Multiplying both sides by \(B^TA^{-1}\) and substituting for \(B^T\overline{u}=0\) we find:
\begin{equation*}
\begin{split} B^T A^{-1} B \overline{p} = B^T A^{-1} \overline{f} \end{split}
\end{equation*}
\sphinxAtStartPar
Such a form exposes the pressure to solution by a Schur Complement method, at the cost of introducing the non\sphinxhyphen{}trivial \(A^{-1}\) operation, which is both heavy and costly.

\sphinxAtStartPar
To avoid having to generate \(A^{-1}\), we may instead reproduce its effect using a \sphinxstyleemphasis{Krylov Subspace} method \sphinxstyleemphasis{(KSP)}, wherein matrix\sphinxhyphen{}matrix products are resolved by decomposing them into iterative series of matrix\sphinxhyphen{}vector products, with each vector comprising the residual of the previous iteration until the residual is less than some nominated threshold. One very popular implementation of this concept is the \sphinxstyleemphasis{Generalised Minimal Residual} method, or \sphinxstyleemphasis{GMRES} {[}\sphinxcite{references:id62}{]}. Once a solution for \(p\) is obtained, we are free to solve for \(u\) through similar methods; and this solution in turn may be used to advect the temperature field and any other state variables using a standard \sphinxstyleemphasis{Petrov\sphinxhyphen{}Galerkin} scheme; for convection problems an ‘upwind’ variant can be used to avoid ‘wiggles’ between nodes {[}\sphinxcite{references:id59}{]}. The actual time integral is carried out using a \sphinxstyleemphasis{Runge\sphinxhyphen{}Kutta} method for accuracy {[}\sphinxcite{references:id60}{]} and is taken over a time interval chosen to be shorter than either the diffusive or the advective timescales across each element, i.e. the \sphinxstyleemphasis{Courant} condition {[}\sphinxcite{references:id63}{]}. Once advection has been carried out, the system is fully iterated in time and the process may be repeated, with the advantage that the pressure solution for the previous timestep can be retained to quicken the convergence of the subsequent step. Performance can be improved even further by incorporating preconditioners for each solve, which exploit \sphinxstyleemphasis{a priori} analytical insights into certain model configurations that may constrain the solution space {[}\sphinxcite{references:id72}{]}

\sphinxAtStartPar
This approach is accurate, robust and stable {[}\sphinxcite{references:id381}{]}. However, even in the best case scenario, the complexity of such a direct solution scales with the cube of the number of elements \(N^3\). To achieve a scaling behaviour closer to \(N\), it is possible to drastically reduce the workload of the inner \sphinxstyleemphasis{KSP}s using an adaptive multi\sphinxhyphen{}grid approach. With this method, a solution is obtained first for a much coarser discretisation, and then corrected over successively finer meshes until the error is within a provided tolerance. In addition to speed and resilience, the multi\sphinxhyphen{}grid lends itself well to parallelisation, as the first\sphinxhyphen{}order global features of the model can be coordinated at the coarsest levels first, while finer local features captured in heavier arrays can be shared more judiciously, thus reducing communications overheads.

\sphinxAtStartPar
The method we have outlined is the product of decades of meticulous development and is known to be as reliable as it is quick. It has been comprehensively benchmarked across a wide range of rheologies and parameters, including models with extreme viscosity contrasts {[}\sphinxcite{references:id544}{]}, elastic behaviour {[}\sphinxcite{references:id384}{]}, and strain\sphinxhyphen{}localising mechanisms {[}\sphinxcite{references:id383}{]}. It has been tested against physical laboratory experiments {[}\sphinxcite{references:id67}{]} and has been demonstrated to be robust and scalable up to thousands of parallel processes {[}\sphinxcite{references:id73}{]}.

\sphinxAtStartPar
One shortcoming of the method we have described is the inappropriateness of the finite element mesh for preserving the geometries of integer\sphinxhyphen{}valued domains \sphinxhyphen{} for example, deformation history, or the distribution of various material phases. Though there are ways of accommodating such features using mesh\sphinxhyphen{}based approaches, they tend to be cumbersome and inefficient. A superior approach is to incorporate Lagrangian particle swarms {[}\sphinxcite{references:id552}{]}, which are able to carry much higher\sphinxhyphen{}resolution information than the mesh and can transport both real\sphinxhyphen{}valued data (e.g. temperature, viscosity) and integer\sphinxhyphen{}valued data (e.g. history, material identity), which mesh\sphinxhyphen{}based variables necessarily cannot. Swarms and their associated variables are advected according to the Stokes solution at the same time as the underlying mesh\sphinxhyphen{}based state variables. In turn, the Stokes solver calls for the swarms to be interpolated to the mesh if and when they become relevant to the solution. Interpolation can be costly and complicated for unstructured networks like particle swarms \sphinxhyphen{} prohibitively so if the node weightings must be recalculated with each timestep, as is the case for any advecting swarm \sphinxhyphen{} so it is important to carry out such an operation as infrequently and efficiently as possible. Gauss quadrature is the standard method, with simple nearest\sphinxhyphen{}neighbour evaluation to determine which elements own which particles. An alternative is to use a grid\sphinxhyphen{}based Voronoi algorithm {[}\sphinxcite{references:id65}{]} in which domains of control for each node are iteratively built out cell by cell; where two nodes lay claim to the same territory, the interpolation grid is refined, but only inside the conflicted cell, thereby minimising superfluous calculations. Another strategy, particularly suited to the interpolation of very sparse swarms, uses a \sphinxstyleemphasis{k\sphinxhyphen{}d} tree to efficiently seek out the nearest particle from any given node. Regardless of interpolation style, the addition of particle swarms dramatically extends the utility of the finite element method.


\subsection{The Underworld code}
\label{\detokenize{content/chapter_02_methods/section2:the-underworld-code}}
\sphinxAtStartPar
Particular software implementations of the methods outlined above have been developed over many computing generations, and several continue to co\sphinxhyphen{}exist today as part of a broad and branching family. The present state\sphinxhyphen{}of\sphinxhyphen{}the\sphinxhyphen{}art iteration is \sphinxstyleemphasis{Underworld}, which supports 2D, 3D, multigrid, and particle\sphinxhyphen{}in\sphinxhyphen{}cell features while also combining a powerful yet modular C\sphinxhyphen{}level infrastructure {[}\sphinxcite{references:id70}{]} with a user\sphinxhyphen{}friendly, hyper\sphinxhyphen{}extensible \sphinxstyleemphasis{Python}\sphinxhyphen{}based API. Parallelisation is provided through \sphinxstyleemphasis{MPI} while the underlying solvers are implemented with \sphinxstyleemphasis{PETSc}. Though deeper layers of Underworld remain fully transparent and accessible, the \sphinxstyleemphasis{Python} layer is designed to encourage fluidity, creativity, and legibility in model building, providing encapsulated higher\sphinxhyphen{}level proxies for the multifarious underlying \sphinxstyleemphasis{C} assets while subtly encouraging a good modelling idiom in users. This has encouraged bespoke application development {[}\sphinxcite{references:id71}{]} and the integration of geodynamics codes with other modelling packages {[}\sphinxcite{references:id66}{]}.

\sphinxAtStartPar
The higher\sphinxhyphen{}level Underworld syntax fully exploits \sphinxstyleemphasis{Python}’s object\sphinxhyphen{}oriented character, encapsulating standard model features like meshes, swarms, variables, and solvers as independent instances of generalised classes. Each object corresponds to \sphinxstyleemphasis{C}\sphinxhyphen{}level structures which are in turn organised under the \sphinxstyleemphasis{StGermain} interoperability framework for computational modelling {[}\sphinxcite{references:id70}{]}. The algorithmic firepower at the heart of the operation draws on the ubiquitous standard \sphinxstyleemphasis{PETSc} code for partial differential equations. By default, the \sphinxstyleemphasis{PETSc} infrastructure is configured for robustness first, speed second; however, a range of options is exposed at the \sphinxstyleemphasis{Python} level to reconfigure the solvers as desired. The principles of encapsulation and localisation are carefully honoured in \sphinxstyleemphasis{Underworld}’s design, with use of global attributes minimised and namespace pollution strictly avoided. Consequently, deletion of obsolete references or ‘garbage collection’ operates mostly as a \sphinxstyleemphasis{Python} user would intuitively expect, so that in typical use cases it is rarely necessary to do more than the elementary due diligence to limit memory leaks. However, at the scales we have operated at, unavoidable pointer entropy at the \sphinxstyleemphasis{C}\sphinxhyphen{}level has been found to proliferate to problematic levels at times. This has been mitigated by prudent reuse of already instantiated objects and by spawning the heaviest or lengthiest jobs in subprocesses to harness system\sphinxhyphen{}level garbage collection.

\sphinxAtStartPar
As accessible as the new tools are, care must still be taken to ensure an appropriately configured model, beginning with the choice of resolution. While temporal resolution, i.e. timestep size, is determined dynamically based on the prescribed tolerances, spatial resolution is the domain of the user. Overly fine elements are wasteful, while insufficiently fine elements will lossily discretise the underlying physics. As a rule of thumb, the spatial resolution in any given region should be half an order of magnitude finer at least than the smallest relevant model features in that vicinity. Of course, it is not always clear \sphinxstyleemphasis{a priori} what this scale will be. Boundary layer and plume theory can provide some information about featural dimensions at steady\sphinxhyphen{}state for simple rheologies; however, the sensitivity of mantle convection on potentially very small\sphinxhyphen{}scale instabilities means a resolution sufficient for a steady\sphinxhyphen{}state solution may still bias the solution at other stages. The appropriate resolution can be sought empirically, by running a suite of progressively finer models until the point of diminishing returns is reached. In theory, a well\sphinxhyphen{}constructed model should converge with resolution in the limit that a discretised model becomes indistinguishable from a continuous one. If convergence does not occur within a computationally feasible envelope, the model may be presumed to be misconfigured in some deeper sense. Another means of determining the correct resolution is to run a single, very\sphinxhyphen{}high resolution test and conduct a power spectral analysis of the constituent fields; the shortest wavelength that contains information must dictate the spatial resolution. The nexus of \sphinxstyleemphasis{Rayleigh} number, featural thickness, and resolution places an upper bound on what parameters can realistically be tested in the context of a large suite\sphinxhyphen{}modelling experiment: although \(Ra\) values approaching \(10^9\) are likely more appropriate for Earth whole\sphinxhyphen{}mantle convection {[}\sphinxcite{references:id339}{]}, most cases explored here are less than \(Ra=10^7\), which is amenable to resolutions in the order of 128 radial cells. If resources are particularly at a premium, static or even adaptive mesh refinement can be employed to concentrate resolution in areas where it is most needed. Unfortunately, it is a present shortcoming of \sphinxstyleemphasis{Underworld} that only quadrangular elements are supported, starkly limiting the options for grid refinement. Work toward supporting unstructured meshes is, however, underway.

\sphinxAtStartPar
\sphinxstyleemphasis{Underworld} provides an easy interface to the underlying \sphinxstyleemphasis{PETSc} options, most notably the choice of inner solve method and tolerance. The default configuration for the solver is \sphinxcode{\sphinxupquote{mg}} or ‘multigrid’, which is the method we have outlined here: this arguably provides the best balance of speed, robustness, scalability, flexibility, and parallelisability. Alternatives include \sphinxcode{\sphinxupquote{mumps}}, ‘multifrontal massively parallel sparse direct solver’, and the ‘lower\sphinxhyphen{}upper method’ \sphinxcode{\sphinxupquote{LU}}. Careful benchmarking is called for to choose the correct configuration, and optimum results cannot be assumed for the default configuration. Tolerances in particular should always be calibrated manually using convergence and power spectrum tests as outlined above. Excessively fine tolerances will needlessly delay solver convergence, while overly generous tolerances will introduce numerical noise into the solution. The chosen tolerances ultimately determine the uncertainty inherent to the model and should always be chosen with care.


\subsection{Underworld in the annulus}
\label{\detokenize{content/chapter_02_methods/section2:underworld-in-the-annulus}}
\sphinxAtStartPar
Even with the gift of Moore’s law, fully three\sphinxhyphen{}dimensional models remain prohibitively expensive, particularly at sufficient resolutions. A two\sphinxhyphen{}dimensional annulus is a compromise that allows exploration of the influence of curvature without exponentially expanding the degrees of freedom. Although such a geometry cannot claim to reproduce Earth\sphinxhyphen{}like conditions as such, it is nonetheless appropriate for probing a wide range of mantle convection phenomena and continues to be widely used.

\sphinxAtStartPar
In Underworld, which at present uses Cartesian\sphinxhyphen{}type meshes, the annulus is constructed by deforming a rectilinear mesh around the origin such that the ratio of inner and outer radii \(f\) falls in the range \(0\to1\), where \(f\to1\) approaches no curvature and \(f=0\) is a model with no core. When the aspect ratio and curvature are such that the two ends meet, those ends are made periodic and the result is a full annulus. For the whole Earth mantle, a ratio of \(f=0.54\) is appropriate; for the upper mantle only, a ratio of \(f = 0.9\) is more realistic. Naturally, a full annulus of \(f=1\) is not possible, while severe solver complications manifest at \(f<0.2\) due to extreme elongation of the basal cells.

\sphinxAtStartPar
Our approach to the annulus has the advantage of robustness and simplicity. It is more amenable for the solver and allows highly curved geometries without requiring a revision of any of the critical computational systems. However, it does have many shortcomings. Outer cells are stretched as inner cells are shortened, forcing a choice of either under\sphinxhyphen{} or over\sphinxhyphen{}resolution of one or other of the boundary layers. Boundary layer conditions for velocity must be defined according to unit vectors rather than simply Cartesian components; it then becomes necessary to rotate and unrotate the boundary vectors during each solver loop. In periodic cases with zero\sphinxhyphen{}shear upper and lower boundaries, we must also be careful to suppress any solid\sphinxhyphen{}body rotation that might emerge by calculating and subtracting any uniform global angular components from the velocity vector field. All of this takes time and CPU cycles, not to mention an increasingly sprawling code overhead.

\sphinxAtStartPar
For our annulus models, it has proven useful to write a bespoke mapping protocol to go from Cartesian to annular domains. The ‘box’ algorithm projects a standard unit square of \(x,y:(0,1)\) onto each annulus or annular wedge such that Cartesian positions can be smoothly and swiftly evaluated in the box and vice versa. This provides a common frame of reference between rectilinear and curvilinear models of any scale and dimensions, with benefits for interoperability, usability, and visualisation. The procedure is both parallel\sphinxhyphen{}safe and parallel\sphinxhyphen{}efficient, and approaches \(C\)\sphinxhyphen{}level performance through careful, idiomatic usage of the NumPy interface. In theory it could be extended to any geometry which is a continuous mapping of a rectilinear mesh.


\section{Suite\sphinxhyphen{}modelling, PlanetEngine and Everest}
\label{\detokenize{content/chapter_02_methods/section3:suite-modelling-planetengine-and-everest}}\label{\detokenize{content/chapter_02_methods/section3::doc}}
\sphinxAtStartPar
A marked advantage of \sphinxstyleemphasis{Underworld} over other codes is the exposure of the \sphinxstyleemphasis{Python}\sphinxhyphen{}level API, transforming a geodynamic numerical modelling code into a fully\sphinxhyphen{}featured platform for geodynamics\sphinxhyphen{}related application development. One product of this has been the \sphinxstyleemphasis{UWGeodynamics} application, which thoroughly streamlines the design and construction of fully\sphinxhyphen{}dimensionalised lithospheric\sphinxhyphen{}scale models for precise and accurate recreation of real\sphinxhyphen{}world scenarios {[}\sphinxcite{references:id71}{]}, taking the already impressive ease\sphinxhyphen{}of\sphinxhyphen{}use of the standard \sphinxstyleemphasis{Underworld} interface and refining it still further for a particular use case.

\sphinxAtStartPar
Just as the particular challenges of lithospheric\sphinxhyphen{}scale modelling demanded and informed a bespoke tool, so have we found it necessary to invest in a unique software platform to support our idiosyncratic use case. The product has been two new pieces of research software, both open\sphinxhyphen{}source and freely available for the community to use, collaborate on, and iterate: \sphinxstyleemphasis{PlanetEngine}, an \sphinxstyleemphasis{Underworld} application for whole\sphinxhyphen{}mantle modelling in the annulus; and Everest, an \sphinxstyleemphasis{HDF5}\sphinxhyphen{}based data format and associated tools for numerical exploration across potentially any field of science.

\sphinxAtStartPar
Here we will discuss the particular dimensions of the research problem at hand and describe the peculiar methodological framework and vocabulary we have developed in response. We will then provide an overview of the architectural solutions we have ventured and discuss some of the higher\sphinxhyphen{}level ‘big data’ methods which the new format now supports.


\subsection{A conceptual meta\sphinxhyphen{}model for suite\sphinxhyphen{}modelling}
\label{\detokenize{content/chapter_02_methods/section3:a-conceptual-meta-model-for-suite-modelling}}
\sphinxAtStartPar
What kind of thing is a planet? This, broadly stated, is the parent question from which all other research questions presented in this thesis derive. Though our immediate contribution must necessarily be limited and contingent, the methodologies we advance here reflect a genuine attempt to formulate a tractable framework that could eventually embrace a solution to this thorniest of planetary problems.

\sphinxAtStartPar
For this thesis, the term ‘suite\sphinxhyphen{}modelling’ is defined so as to encompass any modelling enterprise in which the primary research output derives from an analysis of the divergent behaviours of a series of models, rather than the particular outcomes of each. A model series of forty or four\sphinxhyphen{}hundred or four\sphinxhyphen{}thousand particular strike\sphinxhyphen{}slip zones done with the sole intent of, say, building a catalogue or database of tectonic scenarios, would not necessarily qualify as a suite\sphinxhyphen{}modelling exercise according to this rubrik. Conversely, a survey of only four strike\sphinxhyphen{}slips might qualify if the purpose was to attain a new general inference about strike\sphinxhyphen{}slips which emerges from and transcends the behaviour of any individual case. Suite\sphinxhyphen{}modelling in this sense is an outgrowth of, and shares common methods and objectives with, the sorts of first\sphinxhyphen{}order analytical approaches discussed earlier in this chapter: it is the pursuit of analytical truth by empirical means.

\sphinxAtStartPar
To understand the scope of this challenge, it is instructive to consider perhaps the simplest possible example of a thermal convection model: a box with two insulated vertical walls and two horizontal walls of fixed temperature \sphinxhyphen{} one cooler, one warmer \sphinxhyphen{} containing some incompressible fluid whose nature we wish to scrutinise.

\sphinxAtStartPar
Let us suppose that insulated walls of our box are such that the temperature very rapidly equilibrates across each of the two walls, but only gradually equilibrates with the fluid, so that the wall temperature at any point is the spatiotemporal average (over certain wavelengths) of the fluid temperature adjacent to the wall. For this system, the only variables that we can control are the absolute temperatures of the two surfaces, or equivalently the gradient between them and a reference temperature, while the only variables we may observe are the temperatures of the two insulating walls.

\sphinxAtStartPar
Such an apparatus is the conceptual equivalent of a parameterised two\sphinxhyphen{}dimensional thermal convection model. From first principles, it can be seen to comprise:
\begin{itemize}
\item {} 
\sphinxAtStartPar
A description or intention, which for this thought experiment is essentially just the previous paragraph. This we will term the ‘schema’ after Kant {[}\sphinxcite{references:id64}{]}. There are infinitely many conceivable schemas which relate to each other through uncountably many dimensions, collectively defining a kind of ‘schematic space’ of which any particular schema is a vector.

\item {} 
\sphinxAtStartPar
The implementation, which involves a number of decisions which imperfectly reify the system into a usable form. These decisions have a first\sphinxhyphen{}order influence on the behaviour of the system \sphinxhyphen{} however, they are not properly a part of the system as such, and whatever influence they do have that does not directly serve the systemic intention is a limitation or shortcoming of the model which must be accounted for. The implementation is made up of some amount of irreducible experimental capital, like the box itself and the room it is held in, as well as some number of variable quantities, such as the thicknesses of the walls. The former we term the model ‘capital’, the latter the model ‘options’. Ideally, both the capital and the options of implementation disappear into the background, as when a sufficiently high\sphinxhyphen{}resolution model becomes indistinguishable from the equivalent analytical treatment. In practice, there is always a bleeding\sphinxhyphen{}over from implementation to conceptualisation; consequently, though the implementation details should ideally not enter into the final analytical formulations, they remain inseparable from those outcomes and \sphinxhyphen{} for the sake of completeness and reproducibility \sphinxhyphen{} must always travel with them as metadata.

\item {} 
\sphinxAtStartPar
It has two ‘system variables’: the temperatures of the opposing thermal walls. These variables can be assigned any of infinitely many values and must ultimately emerge in the final treatment as terms of any complete theorem for the model behaviour. Let us call these the ‘parameters’ of the model. The parameters are the degrees of freedom of the schema, so that the schema itself can be visualised as an infinite Cartesian plane with the parameters as axes: the ‘parameter space’. Upon this space, a specific ‘parameterisation’ can be represented as a vector, while a sequence of parameterisations is figurable as a curve. We will term any complete parameterisation a ‘case’ of the schema: a place in parameter space as indicated by a parameterisation vector. It is equivalent to the box from our thought experiment after the wall temperatures are set but before it is filled with fluid.

\item {} 
\sphinxAtStartPar
It has two ‘state variables’: the temperatures of the insulating walls, which are the finest metrics we are permitted to access for defining the interior behaviour. We will term these two variables the ‘configurables’ of the system, and any given set of values as a ‘configuration’ of the system defining a particular ‘state’ that the system can achieve. Just as the parameterisation was conceptualised as a vector through parameter space, so may we regard the configuration as a vector through ‘state\sphinxhyphen{}space’. Before configuration, a given system is everywhere and nowhere in state\sphinxhyphen{}space. In the example of our box experiment, the apparatus is ‘configured’ by pouring fluid into it; we may do this carefully, and hence know what configuration we are selecting, or carelessly, in which case we have no idea where in state\sphinxhyphen{}space we are placing the system. (One important consequence of the way we have described this concept is that all cases share the same state\sphinxhyphen{}space; a configuration in one case is equally conceivable in any other, allowing models to ‘move’ in case\sphinxhyphen{}space as readily as they move between states.)

\end{itemize}

\sphinxAtStartPar
According to this vocabulary, our box experiment can be described as follows. First we take the concept of the experiment and assemble the appropriate capital, carefully noting several model options whose quantities we have implicitly or explicitly decided along the way, such as the conductivity of the wall plates. We then ‘place’ our apparatus at a particular point in parameter space by setting the temperatures of the two thermal walls, thus choosing a fixed model case to explore. We then fill the box with fluid, thus ‘configuring’ it \sphinxhyphen{} at this stage we are of course careful to note the time. At last, we step back from the apparatus and allow it to ‘iterate’, taking occasional measurements of the two state variables along the way. Eventually, some termination condition is reached, and the experiment is ceased. Being good scientists, we again note the time before breaking for lunch.

\sphinxAtStartPar
What has unfolded here? The model has in effect travelled from one configuration in state\sphinxhyphen{}space to another. Along the way, the model has traced out a curve of intermediary configurations through state\sphinxhyphen{}space \sphinxhyphen{} a curve whose form we are made partially aware of by way of our observations. Specifically, this curve is a parametric curve, i.e. a vector\sphinxhyphen{}valued function of time. We could imagine visualising it by plotting it against the two free wall temperatures, with the time since configuration represented by colour. Unlike any arbitrary curve which we could draw over state\sphinxhyphen{}space, this curve is very special: it is the unique, unbidden tendency of the model, arising solely from its own natural logic. Such a trajectory we will term a ‘traverse’.

\sphinxAtStartPar
What sort of knowledge do we hope to attain with such an exercise? If the model were taken to represent an interesting scenario in itself \sphinxhyphen{} for example, coolant flowing through an engine \sphinxhyphen{} then we have our answer: we wanted to know how such\sphinxhyphen{}and\sphinxhyphen{}such a thing would behave in such\sphinxhyphen{}and\sphinxhyphen{}such a situation, and now we know. But if our purpose is to understand the model itself, the outcome of any given experiment is not important. What we are really interested in is the traverse itself \sphinxhyphen{} and what the traverse tells us about the surface over which it extends.

\sphinxAtStartPar
We beg the indulgence of another thought experiment. This time, imagine we are on a causeway high above a dark and endless sea. We know that the surface of the sea is in motion, but we cannot perceive it directly. All we can say for sure is that the pattern of the currents is always the same. How can we ascertain what forces ultimately motivate the churning of the waters?

\sphinxAtStartPar
One option would be to follow the example of Winnie the Pooh {[}\sphinxcite{references:id56}{]} and pepper the surface with little twigs. As each twig travels with the current, tracing out a single streamline, we carefully plot its position \sphinxhyphen{} northings vs eastings, say. It may be that we find that some twigs eventually come to rest over a downwelling, too buoyant to sink. Others may fall into ever\sphinxhyphen{}revolving loops, or conversely, reveal pathways that seem to continue endlessly. Over time we gradually build a sense of the overall nature of the current that is ever improving \sphinxhyphen{} and ever insufficient. For what is certain is that eventually we must stop casting sticks: it will never be possible to completely sample the infinite sea, nor even any region of that sea, even with infinite time \sphinxhyphen{} nor even infinite sticks.

\sphinxAtStartPar
Let the sea represent state\sphinxhyphen{}space. The twigs are individual instances of our box model, with the eastings and northings representing the values of the two state variables; the paths traced by the twigs are individual traverses of a particular case of the model. We see now that a given case of a model is in fact a function that maps flow vectors to coordinates in state\sphinxhyphen{}space \sphinxhyphen{} what we shall hereafter call the ‘case function’ C:
\begin{equation*}
\begin{split} C \left( c_0, c_1, ... c_n \right) = \left[ u_0, u_1, ... u_n \right] = \overline{u} \end{split}
\end{equation*}
\sphinxAtStartPar
Where \(c\) stands for a ‘configurable’, or a variable of state, and \(u\) is the ‘velocity’ in state\sphinxhyphen{}space. For the dark sea, the configurables are the eastings and northings of every point on the surface of the waters, while \(u\) is the flow vector at that point; for the box convection model, these are instead the temperatures of the non\sphinxhyphen{}thermal walls and the future temperature of those walls after time \(t\).

\sphinxAtStartPar
A written statement of \(C\) would be equivalent to a complete analytical solution for the convection problem set out by this particular case. Obtaining such a statement would obviate any need to ever model that particular case of that particular schema again. Of course, such a scientific triumph is rarely achievable even in that small subset of cases where it is strictly possible. Even a partial solution over a limited interval would be conjectural at best: not only have we taken finite samples, but each sample was taken over a finite time interval \sphinxhyphen{} we would require a space\sphinxhyphen{}filling curve of observations, something only possible at infinity.

\sphinxAtStartPar
A more tractable approach would be to draw a ‘watershed’ diagram, ‘colouring’ \sphinxhyphen{} so to speak \sphinxhyphen{} each point in state\sphinxhyphen{}space according to the common limiting behaviours that models initialised at those points ultimately exhibit. For example, if it is observed that a particular minimum or loop of minima exists in state\sphinxhyphen{}space, one might characterise all instances culminating in that vicinity as belonging to a single ‘basin’ of the model. Somewhere in the unknown region between two basins must lie one or more ‘ridges’ and zero or more unobserved basins. By fitting curves through all such regions, one might produce a family of estimates of approximations of partial solutions to the case function \sphinxhyphen{} a contingent victory, but still a valid one. Better still would be to design a sampling strategy around such an analysis, so that the ridges \sphinxhyphen{} chains of bifurcations in state\sphinxhyphen{}space \sphinxhyphen{} are aggressively sought rather than merely inferred.

\sphinxAtStartPar
Let us return to the dark sea, where we now observe that there has been a change in the wind. We repeat our experiment and find that the motion of the water is quite different from before. We infer that the flow is in fact controlled by two forcings. The first is constant: the shape of the sea bed we cannot perceive. The second is variable: the direction and strength of the wind. We resolve to track the wind with two further variables (either the wind vector components or the wind’s trend and magnitude) and commit ourselves to what now emerges as the true scientific challenge: the charting of the obscured sea floor by proxy of the ocean currents.

\sphinxAtStartPar
The two new ‘wind’ variables are the system variables of the dark sea. In the example of our box convection model, they correspond to the temperatures of the upper and lower thermal walls; while the unseen ocean floor corresponds to the enigmatic nature of the fluid inside the box, which it is our ultimate purpose to discover. The ranges of the system variables, as we have discussed, define ‘case\sphinxhyphen{}space’, the set of all case functions \(C\). In this sense, the schema itself could now be thought of as a higher\sphinxhyphen{}order function \sphinxhyphen{} the ‘schema function’ \(S\), whose outputs are case functions:
\begin{equation*}
\begin{split} S \left( p_0, p_1, ... p_n \right) \to C \end{split}
\end{equation*}
\sphinxAtStartPar
Where \(p\) stands for ‘parameter’, the name we have chosen for system variables.

\sphinxAtStartPar
Given the difficulties already discussed with respect to \(C\), we must expect that a complete written expression for \(S\) will be either infeasible or impossible to obtain for all but the most trivial schema. Nevertheless, we are committed to bettering our knowledge of \(S\) by some means, for to understand it is to understand the schema as a whole. What we require is some means of collapsing the dimensionality of the problem.

\sphinxAtStartPar
One option would be to apply some sort of reduction to \(S\):
\begin{equation*}
\begin{split} S_{\mathbb{R}} = R \circ S \end{split}
\end{equation*}
\sphinxAtStartPar
Where \(R\) is an operator that accepts the case functions \(C\) returned by \(S\) and itself returns some statement or metric of \(C\) which summarises its first\sphinxhyphen{}order features. A familiar example of such a reduction would be the concept of ‘phase’ in the physical sciences. The phases of water \sphinxhyphen{} solid, liquid, gas, supercritical fluid, \sphinxstyleemphasis{et cetera} \sphinxhyphen{} are in effect statements regarding the form that water should be expected to take, at steady\sphinxhyphen{}state, given certain values of the two system variables, temperature and pressure: phase, in other words, is a reduction over the state\sphinxhyphen{}space of water. Extending this notion, a phase diagram can be thought of as a kind of reduced schema function, \(S_{\mathbb{R}}\), which maps the degrees of freedom of \(S\) to the outputs of \(R\).

\sphinxAtStartPar
A completely different approach to the same problem is represented by \sphinxstyleemphasis{Hertzsprung\sphinxhyphen{}Russel} (HR) diagram, which organises stars according to their colour and brightness. This is equivalent to taking randomly selected states from randomly selected traverses of randomly selected cases of the schema and plotting them according to some consistent reduction over each state \sphinxhyphen{} in this example, the average surface temperature (colour) and the stellar disc area (brightness) of an arbitrary sample of observed stars. In effect, the HR plot is a kind of multi\sphinxhyphen{}case, discontinuously and stochastically sampled, shared state\sphinxhyphen{}space. The parameters of the HR schema are stellar mass, total internal energy, and relative elemental abundances, which \sphinxhyphen{} being \sphinxstyleemphasis{a priori} unknown \sphinxhyphen{} were not available as axes to organise the observations of stars. Instead, these parameters, along with the missing time dimension and many other symmetries, became apparent \sphinxstyleemphasis{a posteriori} as clusters in shared state\sphinxhyphen{}space. (\sphinxstyleemphasis{Principal Component Analysis} is an example of how such clustering can be detected algorithmically, and for arbitrarily many dimensions to boot.) The HR diagram has understandably become a bellwether of astrophysics, as canonical in its field as phase diagrams are in chemistry.

\sphinxAtStartPar
A third method has recently become available with the advent of modern machine learning techniques. ‘Page\sphinxhyphen{}ranking’ algorithms commonly work by representing each web page as a \textasciitilde{}100,000\sphinxhyphen{}dimensional vector in lexical space, where the vector components are relative word frequencies. Network diagrams of semantic relatedness are constructed by determining distances through the resulting hypercube. These networks can then be tagged with metadata, equivalent to system variables or ‘parameters’ in our treatment, and used to study the evolution of online knowledge systems. Image recognition algorithms work on a similar principle but with the addition of intermediary layers. The ‘configurables’ as such are the RGB values of each pixel, while the ‘parameters’ are the categories or ‘tags’ associated with each image (usually by a human); e.g. ‘dog’, ‘cat’, or ‘truck’. Each case implies infinitely many configurations, yet there is evidently some symmetry in state\sphinxhyphen{}space for each case that uniquely implies its parameters. To detect these symmetries, the state\sphinxhyphen{} and case\sphinxhyphen{}spaces are interleaved with one or more ‘latent spaces’ of much lower dimension than either, revealing latent variables that map similar cases to potentially wildly unalike configurations. One shortcoming of this approach is that it is difficult to peer into these hidden layers to learn exactly how the algorithm knows what it appears to know. Lately, however, it has even become possible to ‘traverse’ these latent spaces \sphinxhyphen{} or, rather, to traverse shared state\sphinxhyphen{}space according to trajectories in latent space {[}\sphinxcite{references:id55}{]}. Such an approach can elucidate machine intuitions and potentially allow them to be captured in human\sphinxhyphen{}cognisable form.

\sphinxAtStartPar
Regardless of how it is achieved, the fundamental modus operandi of all these methods is the same: to uncover hidden symmetries that relate a multitude of superficially dissimilar effects to a small number of inferred causes. Minting an original vocabulary for this family of problems may help provoke novel collaborations between apparently disparate fields which share similar problems. We cannot hope to make much progress in isolation.


\subsection{Suite\sphinxhyphen{}modelling for geodynamics: PlanetEngine and Everest}
\label{\detokenize{content/chapter_02_methods/section3:suite-modelling-for-geodynamics-planetengine-and-everest}}
\sphinxAtStartPar
Having developed an expressive vocabulary for discussing suite\sphinxhyphen{}modelling problems, we now apply it to geodynamics. Our purpose is to constrain the nature of our problem, and determine what tools we require to successfully engage with it.

\sphinxAtStartPar
We begin with our intention, which is to ascertain the consequences of some rheological formulation through numerical modelling of mantle convection. This gives us our schema, which will be implemented through Underworld code. We have several options to consider: resolution, timestep size, tolerance. These we must carefully select according to the considerations discussed earlier. The parameters of our schema are manifold, and might include the Rayleigh number \(Ra\), the heating term \(H\), the degree of curvature \(f\), the model aspect ratio \(A\), the maximum viscosity contrast \(\eta_0\), the yielding coefficient \(\tau\), the thermal diffusivity \(\kappa\), the thermal expansivity \(\alpha\), the reference density \(\rho\), and more.

\sphinxAtStartPar
Let us estimate the total number of parameters to be in the order of \(10\) \sphinxhyphen{} this compares to only two in the examples presented thus far. At the configuration level, we have some number of state variables, beginning with the temperature field and possibly including the distribution of material phases, the stress history, and others. In two dimensions, each field has in the order of \(N^2\) degrees of freedom, where \(N\) is the resolution; if our resolution is in the order of \(100\), the total number of configurables will be in the order \(10^4\); compare this to the mere two configurables in both the box convection and ‘dark sea’ thought experiments before. Overall, the dimensions of the problem are clearly very much greater than before. Instead of two\sphinxhyphen{}dimensional planes describing the case\sphinxhyphen{} and state\sphinxhyphen{}spaces, these must now be represented by 10\sphinxhyphen{}dimensional and 100,000\sphinxhyphen{}dimensional hypercubes.

\sphinxAtStartPar
We are already aware of several methods with which to attack such a problem in post\sphinxhyphen{}analysis. What we have not discussed is the formidable theoretical, logistical, and organisational challenge of designing, commissioning, overseeing, and aggregating a suite\sphinxhyphen{}modelling campaign on this scale. Bespoke tools are called for; tools specific to geodynamics, as well as tools germane to suite\sphinxhyphen{}modelling in general. To answer these needs, we introduce \sphinxstyleemphasis{PlanetEngine} and \sphinxstyleemphasis{Everest}.


\subsubsection{\sphinxstyleemphasis{PlanetEngine}}
\label{\detokenize{content/chapter_02_methods/section3:planetengine}}
\sphinxAtStartPar
\sphinxstyleemphasis{PlanetEngine} is a \sphinxstyleemphasis{Python}\sphinxhyphen{}based wrapper for the geodynamics code Underworld which specialises in dimensionless large\sphinxhyphen{}scale convection problems in Cartesian and cylindrical geometries. Its primary purpose is to provide successively higher\sphinxhyphen{}order objects that implement in code the suite\sphinxhyphen{}modelling idiom described above. Accordingly, \sphinxstyleemphasis{PlanetEngine}’s top\sphinxhyphen{}level classes include:
\begin{itemize}
\item {} 
\sphinxAtStartPar
The ‘System’ class, which non\sphinxhyphen{}invasively and transparently wraps native Underworld models to provide an easy\sphinxhyphen{}to\sphinxhyphen{}use higher\sphinxhyphen{}level framework for essential model\sphinxhyphen{}wrangling operations, including methods for visualisation, initialisation, iteration, observation, checkpointing, loading, and more. The user is invited to flag system inputs as options, parameters, or configurables, which are then separately ‘stamped’ as metadata on all model outputs; apart from this, users are only expected to define ‘update’ and ‘iterate’ functions to enable all higher\sphinxhyphen{}level behaviours of the System interface. \sphinxstyleemphasis{PlanetEngine} systems correspond semantically to the ‘system’ notion introduced in our earlier conceptual model \sphinxhyphen{} that is, a system is a particular instance of a particular case, initialised at a particular point in state\sphinxhyphen{}space, with the equipped with the ability to integrate through time in accordance with the ‘flow’ of the case. \sphinxstyleemphasis{PlanetEngine} systems inherit from \sphinxstyleemphasis{Everest} ‘Iterables’ (see next section), whence they derive most of their useful properties.

\item {} 
\sphinxAtStartPar
A family of abstractly\sphinxhyphen{}defined ‘conditions’, which may be either unique \sphinxhyphen{} suitable for initialising a model (e.g. a sinusoidal initial condition) \sphinxhyphen{} or non\sphinxhyphen{}unique \sphinxhyphen{} suitable for terminating them (e.g. a steady\sphinxhyphen{}state criterion). Conditions can be merged in various ways to create more complex conditions, for example overlaying noise over a conductive geotherm, or applying a horizontal perturbation after applying a vertical one. A special class of conditions allows model state variables to be initialised from the outputs of other models, even across different geometries. \sphinxstyleemphasis{PlanetEngine} conditions should be thought of as representing localities or regions in state\sphinxhyphen{}space, which may either be defined absolutely (i.e. with respect to the origin) or relatively (i.e. with respect to internal metrics, typically timestep or time index, of other systems). \sphinxstyleemphasis{PlanetEngine} conditions inherit from \sphinxstyleemphasis{Everest} ‘Booleans’ and hence are valid inputs for \sphinxstyleemphasis{Python} ‘if’ statements and other typical true\sphinxhyphen{}false operations.

\item {} 
\sphinxAtStartPar
An ‘Observer’ class that makes and stores observations about Systems as they iterate through time. Observers are initialised with an observee System and a Condition object that determines when observations should be taken, e.g. every tenth step, or whenever average temperature exceeds a certain threshold. While it is a functionality of Systems to store their own state variables in full at certain user\sphinxhyphen{}defined intervals (i.e. checkpointing), it is the role of Observers to gather various derived datas, often of lower dimension than state\sphinxhyphen{}space, and typically at a much greater frequency than the checkpointing interval. \sphinxstyleemphasis{PlanetEngine} is designed to support extremely thorough runtime analyses while minimising any resulting impacts to iteration speed and memory usage. This is invaluable for very large model suites, where disk efficiency and sampling rate are priorities. \sphinxstyleemphasis{PlanetEngine} observers are \sphinxstyleemphasis{Everest} ‘Producers’ but not Iterables in their own right; their internal iteration count is incremented only by their associated system.

\item {} 
\sphinxAtStartPar
A ‘Traverse’ class which, when activated, builds a System from provided ingredients (schema, options, parameters, and configurables) and iterates it until a provided terminal condition is recognised. Observers may optionally be attached, which are ‘prompted’ to observe with every iteration (whether they do or not depends on their internal Condition). Traverses in \sphinxstyleemphasis{PlanetEngine} correspond directly to traverses in the conceptual model: they are journeys from a unique starting point to a non\sphinxhyphen{}unique end point according to the ineluctable logic of each case and schema. \sphinxstyleemphasis{PlanetEngine} traverses are instances of the Everest Task class, hence their basic functionality is to do some ‘indelible work’ (work which is written to disk), which here involves taking at least one initial and one terminal checkpoint. This allows other traverses (or tasks generally) to pick up where the previous traverse left off.

\item {} 
\sphinxAtStartPar
The ‘Campaign’ class: the capstone functionality of \sphinxstyleemphasis{PlanetEngine}. Campaigns are a kind of \sphinxstyleemphasis{Everest} Container type which, when iterated, continually produce, execute, and destroy feasibly endless sequences of Traverses. Campaigns are configured to be operated by multiple unaffiliated processes, and even multiple separate devices, simultaneously. Hosted tasks are tracked using a ‘library card’ system which monitors which tasks are currently checked out by other processes, which have been returned complete, which incomplete, and which have been returned after having failed. Campaigns distinguish between failures due to systematic errors (e.g. unacceptable input parameters or misconfigured models) and failures due to extraneous circumstances (e.g. power outages); extensive use of context managers ensures resilience and minimises corruption, which can otherwise become critically problematic when suite populations grow beyond a user’s ability to individually curate them. Most importantly, Campaigns are designed to accept stopping and starting without complaint; resources may be attached and detached at will, or the campaign as a whole suspended for any length of time, without any logistical consequences: when the campaign is renewed, or new resources added, it is guaranteed to continue precisely from where it was before. Campaigns can be provided with static lists of jobs to loop through, but can also be given ‘smart’ assignments that react to model outcomes and change tack accordingly. For instance, a Campaign might be instructed to randomly sample case space until a bifurcation is encountered, then spawn new jobs parallel to that bifurcation to explore its extent. The potential of such a tool may be formidable.

\end{itemize}

\sphinxAtStartPar
In addition to its headline features, \sphinxstyleemphasis{PlanetEngine} provides a wide range of extended functionalities for basic \sphinxstyleemphasis{Underworld} objects. Some are being considered for incorporation into the main \sphinxstyleemphasis{Underworld} branch; others are more specialised for the particular use case of \sphinxstyleemphasis{PlanetEngine}. Highlights include:
\begin{itemize}
\item {} 
\sphinxAtStartPar
Specialised visualisation options within and beyond \sphinxstyleemphasis{Underworld}’s built\sphinxhyphen{}in \sphinxstyleemphasis{gLucifer} interface. In particular, \sphinxstyleemphasis{PlanetEngine} provides a class that produces cheap, light, and consistent raster images and animations for any given \sphinxstyleemphasis{Underworld} data object at any given degree of detail \sphinxhyphen{} ideal for machine learning applications. PlanetEngine also offers a \sphinxcode{\sphinxupquote{quickShow}} feature that aggressively searches for an appropriate visualisation strategy for any given input or inputs, automatically constructing projections and dummy variables as required, allowing for very rapid generation of figures. All \sphinxstyleemphasis{PlanetEngine} visualisations are wrapped in the ‘Fig’ class, which handles disk operations, filenames, and other house\sphinxhyphen{}keeping matters.

\item {} 
\sphinxAtStartPar
A module that handles mappings between coordinate systems, especially between different annular domains, using a ‘box’ abstraction that normalises coordinates to a unit height/width/length volume, allowing \sphinxhyphen{} for instance \sphinxhyphen{} the evaluation of \sphinxstyleemphasis{Underworld} functions or data objects between multiple different meshes. This powers a workhorse \sphinxcode{\sphinxupquote{copyField}} function that warps data between meshes to within a given tolerance; \sphinxcode{\sphinxupquote{copyField}} can also be configured to tile, fade, or mirror spatially referenced data, allowing \sphinxhyphen{} for instance \sphinxhyphen{} a steady\sphinxhyphen{}state condition derived cheaply on a small\sphinxhyphen{}aspect coarse mesh to be tiled over a higher\sphinxhyphen{}aspect fine mesh.

\item {} 
\sphinxAtStartPar
An extensive toolbox of new \sphinxstyleemphasis{Underworld} ‘function’\sphinxhyphen{}type classes with a number of useful features, including lazy evaluation, optional lazy initialisation, optional incremental computation, optional memoisation, automated projection, automatic labelling, hash identification, and extended operator overloading. The new functions include dynamic one\sphinxhyphen{}dimensional variables (e.g. minima/maxima, integrals, average), region\sphinxhyphen{}limited operations, arbitrarily nested derivatives, stream functions, quantiles, vector operations, masks, normalisation functions, conditionals, filters, clips, and much more. Most of the new functions are built from standard \sphinxstyleemphasis{Underworld} functions and data types and are consequently robust and \sphinxstyleemphasis{C}\sphinxhyphen{}level efficient; all in turn inherit from the built\sphinxhyphen{}in \sphinxstyleemphasis{Underworld} function class and so may be used anywhere the native inventory is used; however, their design is particularly oriented towards and optimised for run\sphinxhyphen{}time analysis. All are tested parallel\sphinxhyphen{}safe and memory leak\sphinxhyphen{}free.

\item {} 
\sphinxAtStartPar
A series of analysis classes providing single\sphinxhyphen{}line invocation for common analytical operations, including \sphinxstyleemphasis{Nusselt} number, velocity root\sphinxhyphen{}mean\sphinxhyphen{}square, and vertical stress profile. Like \sphinxstyleemphasis{PlanetEngine} function objects, analysis objects are hashed and logged at initialisation to avoid needless duplication; they also store their observations locally and are configured to avoid redundant invocation. They export the \sphinxstyleemphasis{Underworld} function interface and so are safe to use inside Underworld systems: for example, one could define the yield stress in terms of some time\sphinxhyphen{}averaged function of the stress history by simply incorporating the appropriate \sphinxstyleemphasis{PlanetEngine} analyser.

\item {} 
\sphinxAtStartPar
A family of abstractly\sphinxhyphen{}defined ‘conditions’, both unique, suitable for initialising a model (e.g. a sinusoidal initial condition), and non\sphinxhyphen{}unique, suitable for terminating them (e.g. a steady\sphinxhyphen{}state criterion). Conditions can be merged in various ways to create more complex conditions, for example overlaying noise over a conductive geotherm, or applying a horizontal perturbation after applying a vertical one. A special class of conditions allows model state variables to be initialised from the outputs of other models, even across different geometries.

\item {} 
\sphinxAtStartPar
A system for attaching boundary and range assumptions directly to Underworld data objects, enabling automatic clipping and normalisation of hosted data whenever changes are detected.

\end{itemize}

\sphinxAtStartPar
Together, these features constitute a meaningful advance in the state\sphinxhyphen{}of\sphinxhyphen{}the\sphinxhyphen{}art, not only for \sphinxstyleemphasis{Underworld}, but for computational geodynamics generally.


\subsubsection{\sphinxstyleemphasis{Everest}}
\label{\detokenize{content/chapter_02_methods/section3:everest}}
\sphinxAtStartPar
\sphinxstyleemphasis{Everest} is a free, open\sphinxhyphen{}source \sphinxstyleemphasis{Python} package developed as a part of this thesis for the express purpose of organising and analysing massive suite\sphinxhyphen{}modelling campaigns. Fundamentally, it comprises two systems:
\begin{itemize}
\item {} 
\sphinxAtStartPar
A library of Python classes which may be wrapped around user code to provide a high\sphinxhyphen{}level standardised interface for common suite\sphinxhyphen{}modelling operations.

\item {} 
\sphinxAtStartPar
A file format under \sphinxstyleemphasis{HDF5} which provides disk correlates for objects inheriting from \sphinxstyleemphasis{Everest} and supports simultaneous and parallel access for any number of processes at once.

\end{itemize}

\sphinxAtStartPar
The core \sphinxstyleemphasis{Everest} class is the so\sphinxhyphen{}called ‘Built’. The general principle of the Built is to provide a framework that allows a complete working copy of the environment in which some data was produced to be recreated, live, just as it originally was, using only some attached metadata. Modules which define a class that inherits from Built can be ‘anchored’ to an \sphinxstyleemphasis{HDF5} file on disk, referred to as a ‘frame’; the class can then be loaded directly from the frame even if the original module is no longer present. Each Built is assigned a unique hashed identity based on the complete code of the module defining it, which in turn provides the name of the HDF ‘group’ that the object is filed under in the archive: a simple ‘word hash’ facility is provided that generates memorable English\sphinxhyphen{}pronouncable names based on each hash, for human convenience. If an instantiated Built is anchored, the parameters that initialised the instance are also saved and hashed: supported parameters for saving include all the standard \sphinxstyleemphasis{Python} and \sphinxstyleemphasis{NumPy} data types and containers, all objects that provide a ‘Pickling’ interface (user\sphinxhyphen{}defined or otherwise), as well as any class or class instance that inherits from Built. The use of hashes ensures that even the slightest difference between two classes or instances will be recorded for posterity, so that large projects can never be corrupted by the introduction of small errors and new objects never overwrite, but instead coordinate with, any previously anchored objects which are identical; while the storing of ample metadata with each Built ensures complete reproducibility within the scope of the environment that \sphinxstyleemphasis{Everest} is made aware of. The use of hash identities means that all builts can be stored equally as top\sphinxhyphen{}level objects in the host frame without any negative organisational consequences: this is virtuous because an appropriate hierarchical structure to explore one particular scientific query may be entirely counterproductive even for an only slightly different query. Flat structures allow the most efficient organisational structure for each use\sphinxhyphen{}case to be mapped to the database on a case\sphinxhyphen{}by\sphinxhyphen{}case basis, without needlessly and perhaps irreversibly inflecting the database for one purpose alone.

\sphinxAtStartPar
Various subclasses to Built are provided which add further functionality. The Producer class adds support for producing and storing arbitrary data: outputs can be stored in memory, saved to the frame, or automatically configured to save at certain intervals depending on memory usage, clock time, or other factors; data can be sorted either in the order it is saved or in order of some provided index, in which case inadvertent duplication of entries is detected and prevented. The Iterator class, a kind of Cycler or callable Built, is designed to be inherited by intrinsically iterable models \sphinxhyphen{} for example, the mantle convection models discussed here which iterate through discrete time steps \sphinxhyphen{} and provides for saving and loading of states, ‘bouncing’ between states, iterating between step counts, and many other useful features. (An anchored Iterator is always aware of what states have previously been saved to the frame, even if they have been saved by completely disconnected processes, such that multiple devices can do work on the same model without clashing.) The State class provides for the definition of abstract conditions which can be evaluated with respect to other Builts; States can then be used to trigger Enactors, Conditions, Inquirers, and other objects, progressively supporting higher levels of coordination. The highest level of abstraction currently provided by Everest is the Task class, which permutes a given Cycler until a stipulated Condition is met; because the Task is itself both a Cycler and a Condition, it can be used as either input of another Task, allowing the construction of arbitrarily complex trees of tasks. Tasks also provide the very useful feature of being able to instantiate themselves as subprocesses, so that one process can serve as a master for many others: any errors are logged and returned to the master task where they are handled as per the user’s requirements. Carrying out a task in a subprocess also ensures that the slate is wiped clean for any future task: this is particularly important for memory management in large model suites, as the system\sphinxhyphen{}level garbage collectors are much more thorough than those provided at the \sphinxstyleemphasis{Python} or \sphinxstyleemphasis{Cython} levels.

\sphinxAtStartPar
\sphinxstyleemphasis{Everest} also provides bespoke Writer and Reader classes for the \sphinxstyleemphasis{HDF5}\sphinxhyphen{}based ‘frame’ to which Builts may optionally be anchored. The Writer automatically decides what sort of \sphinxstyleemphasis{HDF5} object to create, and where, based on intuitive \sphinxstyleemphasis{Python}\sphinxhyphen{}level syntax: dictionaries are mapped to groups, for instance, while \sphinxstyleemphasis{NumPy} arrays are mapped to datasets or attributes as appropriate. The Writer also manages access to the frame using a lock file and a prime\sphinxhyphen{}number based ‘window of access’ protocol, so that simultaneous write operations never occur regardless of how many processes or devices attempt access. The Reader, conversely, supports multiple simultaneous access through the new \sphinxstyleemphasis{HDF5} single\sphinxhyphen{}write\sphinxhyphen{}simultaneous\sphinxhyphen{}read protocol. The Reader exports the \sphinxstyleemphasis{Python} ‘index’ and ‘slice’ syntaxes and is designed to provide fluid access to archived data no matter how hefty the frame becomes. It comes with its own search algorithm with wildcard support, allowing quick and easy filtering according to data type, Built type, hash identity, value, or any other data: search is recursive to a prescribed depth and is careful to appropriately manage loops where \sphinxstyleemphasis{HDF5} links are present. It operates in two modes: ‘soft’, which returns summaries of query results, and ‘hard’, which returns all results and loads any pickled objects, Built classes, and Built instances which are included. This latter feature makes it very quick and easy to load, for example, one particular model out of thousands by referencing merely one or two of its input parameters, some faintly recalled property of its output data, and the hash ID of the associated type. Algorithmic access is of course also supported \sphinxhyphen{} iterating through models to produce a figure, for instance, or to conduct some test whose outcome could then be tagged to each model as metadata.

\sphinxAtStartPar
Finally, \sphinxstyleemphasis{Everest} also describes a ‘Scope’ class which is in effect an abstract statement about the range of ‘builts’, and the range of indices within the outputs of each built, for which some boolean statement evaluates true. Scopes are produced by indexing Readers with ‘Fetch’ requests, somewhat similar to \sphinxstyleemphasis{SQL} queries; the resultant ‘scope’ is instantiated with the metadata of the request that produced it, ensuring that all data travels with its full and proper context. Once instantiated, however, a scope is free to travel independent of the reader or indeed the frame over which it was defined: this is possible thanks to the use of unique hash identities for all Everest builts of equivalent type and initialisation profile. Once the desired Scope has been acquired, it can be used to ‘slice’ into a frame (once again through the Reader interface) to pull the indicated data in full. The Scope protocol allows easy and rapid aggregation of very large datasets across arbitrarily broad model suites. In tandem with standard disk\sphinxhyphen{}based analysis packages like Dask, they are intended to provide a pain\sphinxhyphen{}free solution for wrangling model data all the way up to the terabyte scale.


\section{Conclusion}
\label{\detokenize{content/chapter_02_methods/conclusion:conclusion}}\label{\detokenize{content/chapter_02_methods/conclusion::doc}}
\sphinxAtStartPar
In this chapter we have discussed the theory of mantle convection; explained and justified our preferred numerical approach; introduced the software tools requisite for the job; and presented a detailed conceptual framework for our research methodology. Further and more detailed methods will be explained in future chapters as they become relevant. Although the challenge at hand is formidable \sphinxhyphen{} ever more evidently so as we progress \sphinxhyphen{} we hope we have demonstrated that the methodologies thus far outlined are tried, true, and fit for purpose. With equal parts ambition and caution, let us proceed.


\chapter{Everest}
\label{\detokenize{content/chapter_03_everest/main:everest}}\label{\detokenize{content/chapter_03_everest/main::doc}}

\chapter{Scaling Behaviours of Linear Rheologies}
\label{\detokenize{content/chapter_04_linear/abstract:scaling-behaviours-of-linear-rheologies}}\label{\detokenize{content/chapter_04_linear/abstract::doc}}

\section{Introduction}
\label{\detokenize{content/chapter_04_linear/section01_introduction:introduction}}\label{\detokenize{content/chapter_04_linear/section01_introduction::doc}}

\section{Background}
\label{\detokenize{content/chapter_04_linear/section02_background:background}}\label{\detokenize{content/chapter_04_linear/section02_background::doc}}
\begin{sphinxVerbatim}[commandchars=\\\{\}]
\PYG{k+kn}{from} \PYG{n+nn}{referencing} \PYG{k+kn}{import} \PYG{n}{search}
\end{sphinxVerbatim}


\section{Methods}
\label{\detokenize{content/chapter_04_linear/section03_methods:methods}}\label{\detokenize{content/chapter_04_linear/section03_methods::doc}}

\section{Results}
\label{\detokenize{content/chapter_04_linear/section04_results:results}}\label{\detokenize{content/chapter_04_linear/section04_results::doc}}

\section{Discussion}
\label{\detokenize{content/chapter_04_linear/section05_discussion:discussion}}\label{\detokenize{content/chapter_04_linear/section05_discussion::doc}}

\section{Linear}
\label{\detokenize{content/chapter_04_linear/conclusion:linear}}\label{\detokenize{content/chapter_04_linear/conclusion::doc}}

\chapter{Mode Boundaries in Viscoplastic Rheology}
\label{\detokenize{content/chapter_05_viscoplastic/abstract:mode-boundaries-in-viscoplastic-rheology}}\label{\detokenize{content/chapter_05_viscoplastic/abstract::doc}}

\section{Introduction}
\label{\detokenize{content/chapter_05_viscoplastic/section01_introduction:introduction}}\label{\detokenize{content/chapter_05_viscoplastic/section01_introduction::doc}}

\section{Background}
\label{\detokenize{content/chapter_05_viscoplastic/section02_background:background}}\label{\detokenize{content/chapter_05_viscoplastic/section02_background::doc}}

\section{Method}
\label{\detokenize{content/chapter_05_viscoplastic/section03_methods:method}}\label{\detokenize{content/chapter_05_viscoplastic/section03_methods::doc}}

\section{Results}
\label{\detokenize{content/chapter_05_viscoplastic/section04_results:results}}\label{\detokenize{content/chapter_05_viscoplastic/section04_results::doc}}

\section{Discussion}
\label{\detokenize{content/chapter_05_viscoplastic/section05_discussion:discussion}}\label{\detokenize{content/chapter_05_viscoplastic/section05_discussion::doc}}

\section{Linear}
\label{\detokenize{content/chapter_05_viscoplastic/conclusion:linear}}\label{\detokenize{content/chapter_05_viscoplastic/conclusion::doc}}

\chapter{Big Data Methods for Computational Geodynamics}
\label{\detokenize{content/chapter_06_advanced/abstract:big-data-methods-for-computational-geodynamics}}\label{\detokenize{content/chapter_06_advanced/abstract::doc}}

\section{Introduction}
\label{\detokenize{content/chapter_06_advanced/section01_introduction:introduction}}\label{\detokenize{content/chapter_06_advanced/section01_introduction::doc}}

\section{Background}
\label{\detokenize{content/chapter_06_advanced/section02_background:background}}\label{\detokenize{content/chapter_06_advanced/section02_background::doc}}

\section{Methods}
\label{\detokenize{content/chapter_06_advanced/section03_methods:methods}}\label{\detokenize{content/chapter_06_advanced/section03_methods::doc}}

\section{Results}
\label{\detokenize{content/chapter_06_advanced/section04_results:results}}\label{\detokenize{content/chapter_06_advanced/section04_results::doc}}

\section{Discussion}
\label{\detokenize{content/chapter_06_advanced/section05_discussion:discussion}}\label{\detokenize{content/chapter_06_advanced/section05_discussion::doc}}

\section{Linear}
\label{\detokenize{content/chapter_06_advanced/conclusion:linear}}\label{\detokenize{content/chapter_06_advanced/conclusion::doc}}

\chapter{Discussion}
\label{\detokenize{content/chapter_07_discussion/main:discussion}}\label{\detokenize{content/chapter_07_discussion/main::doc}}

\chapter{Conclusion}
\label{\detokenize{content/chapter_08_conclusion/main:conclusion}}\label{\detokenize{content/chapter_08_conclusion/main::doc}}

\chapter{References}
\label{\detokenize{references:references}}\label{\detokenize{references::doc}}
\sphinxAtStartPar



\chapter{Tables}
\label{\detokenize{appendices/tables/main:tables}}\label{\detokenize{appendices/tables/main::doc}}
\begin{sphinxthebibliography}{Internat}
\bibitem[Abe93]{references:id696}
\sphinxAtStartPar
Yutaka Abe. Physical state of the very early earth. \sphinxstyleemphasis{Lithos}, 30(3):223–235, 8 1993.
\bibitem[ACD+12]{references:id212}
\sphinxAtStartPar
Elisabeth R Adams, David R Ciardi, Andrea K Dupree, T Nick Gautier, III, Craig Kulesa, and Don McCarthy. Adaptive optics images of kepler objects of interest. \sphinxstyleemphasis{Astron. J.}, 144(2):42, 2012.
\bibitem[Alb98]{references:id490}
\sphinxAtStartPar
Francis Albarede. The growth of continental crust. \sphinxstyleemphasis{ELSEVIER Tectonophysics}, 296:1–14, 1998.
\bibitem[And82]{references:id261}
\sphinxAtStartPar
Don L Anderson. Hotspots, polar wander, mesozoic convection and the geoid. \sphinxstyleemphasis{Nature}, 297(5865):391–393, 5 1982.
\bibitem[ABCN+11]{references:id334}
\sphinxAtStartPar
Denis Andrault, Nathalie Bolfan\sphinxhyphen{}Casanova, Giacomo Lo Nigro, Mohamed A Bouhifd, Gaston Garbarino, and Mohamed Mezouar. Solidus and liquidus profiles of chondritic mantle: implication for melting of the earth across its history. \sphinxstyleemphasis{Earth Planet. Sci. Lett.}, 304(1):251–259, 3 2011.
\bibitem[AGOsamuIshizuka+19]{references:id277}
\sphinxAtStartPar
Richard J Arculus, Michael Gurnis, Osamu Ishizuka, Mark K Reagan, Julian A Pearce, and Rupert Sutherland. HOW TO CREATE NEW SUBDUCTION ZONES: a global perspective. \sphinxstyleemphasis{Oceanography}, 32(1):160, 2 2019.
\bibitem[ACF+18]{references:id365}
\sphinxAtStartPar
M Arnould, N Coltice, N Flament, V Seigneur, and R D Müller. On the scales of dynamic topography in Whole\sphinxhyphen{}Mantle convection models. \sphinxstyleemphasis{Geochem. Geophys. Geosyst.}, 37:883, 8 2018.
\bibitem[Ast18]{references:id66}
\sphinxAtStartPar
Michael Asten. Education matters: the ARC basin GENESIS hub–connecting solid earth evolution to sedimentary basins. \sphinxstyleemphasis{Preview}, 2018(195):27–32, 2018.
\bibitem[Aul12]{references:id705}
\sphinxAtStartPar
Sonja Aulbach. Craton nucleation and formation of thick lithospheric roots. \sphinxstyleemphasis{Lithos}, 149:16–30, 8 2012.
\bibitem[Ayd06]{references:id617}
\sphinxAtStartPar
Atilla Aydin. Failure modes of the lineaments on jupiter's moon, europa: implications for the evolution of its icy crust. \sphinxstyleemphasis{J. Struct. Geol.}, 28:2222, 2006.
\bibitem[BC18]{references:id198}
\sphinxAtStartPar
Daniel N Baker and Amal Chandran. Space, still the final frontier. \sphinxstyleemphasis{Science}, 361(6399):207, 6 2018.
\bibitem[BMD07]{references:id220}
\sphinxAtStartPar
Victor R Baker, Shigenori Maruyama, and James M Dohm. Tharsis superplume and the geological evolution of early mars. In David A Yuen, Shigenori Maruyama, Shun\sphinxhyphen{}Ichiro Karato, and Brian F Windley, editors, \sphinxstyleemphasis{Superplumes: Beyond Plate Tectonics}, pages 507–522. Springer Netherlands, Dordrecht, 2007.
\bibitem[BGT92]{references:id237}
\sphinxAtStartPar
W B Banerdt, Matthew P Golombek, and Kenneth L Tanaka. Stress and tectonics on mars. In \sphinxstyleemphasis{Mars}, pages 249–297. 0 1992.
\bibitem[Bar08]{references:id158}
\sphinxAtStartPar
Amy C Barr. Mobile lid convection beneath enceladus' south polar terrain. \sphinxstyleemphasis{J. Geophys. Res.}, 113(E7):L09202, 6 2008.
\bibitem[BTJ+04]{references:id591}
\sphinxAtStartPar
Gwendolyn D Bart, Elizabeth P Turtle, Windy L Jaeger, Laszlo P Keszthelyi, and Richard Greenberg. Ridges and tidal stress on io. \sphinxstyleemphasis{Icarus}, 169:111–126, 2004.
\bibitem[BB06]{references:id118}
\sphinxAtStartPar
Gibor Basri and Michael E Brown. PLANETESIMALS TO BROWN DWARFS: what is a planet? \sphinxstyleemphasis{Annu. Rev. Earth Planet. Sci.}, 34(1):193–216, 4 2006.
\bibitem[BRB+13]{references:id210}
\sphinxAtStartPar
Natalie M Batalha, Jason F Rowe, Stephen T Bryson, Thomas Barclay, Christopher J Burke, Douglas A Caldwell, Jessie L Christiansen, Fergal Mullally, Susan E Thompson, Timothy M Brown, Andrea K Dupree, Daniel C Fabrycky, Eric B Ford, Jonathan J Fortney, Ronald L Gilliland, Howard Isaacson, David W Latham, Geoffrey W Marcy, Samuel N Quinn, Darin Ragozzine, Avi Shporer, William J Borucki, David R Ciardi, I I I Thomas N. Gautier, Michael R Haas, Jon M Jenkins, David G Koch, Jack J Lissauer, William Rapin, Gibor S Basri, Alan P Boss, Lars A Buchhave, Joshua A Carter, David Charbonneau, Joergen Christensen\sphinxhyphen{}Dalsgaard, Bruce D Clarke, William D Cochran, Brice\sphinxhyphen{}Olivier Demory, Jean\sphinxhyphen{}Michel Desert, Edna Devore, Laurance R Doyle, Gilbert A Esquerdo, Mark Everett, Francois Fressin, John C Geary, Forrest R Girouard, Alan Gould, Jennifer R Hall, Matthew J Holman, Andrew W Howard, Steve B Howell, Khadeejah A Ibrahim, Karen Kinemuchi, Hans Kjeldsen, Todd C Klaus, Jie Li, Philip W Lucas, Søren Meibom, Robert L Morris, Andrej Prša, Elisa Quintana, Dwight T Sanderfer, Dimitar Sasselov, Shawn E Seader, Jeffrey C Smith, Jason H Steffen, Martin Still, Martin C Stumpe, Jill C Tarter, Peter Tenenbaum, Guillermo Torres, Joseph D Twicken, Kamal Uddin, Jeffrey Van Cleve, Lucianne Walkowicz, and William F Welsh. PLANETARY CANDIDATES OBSERVED BY KEPLER. III. ANALYSIS OF THE FIRST 16 MONTHS OF DATA. \sphinxstyleemphasis{ApJS}, 204(2):24, 1 2013.
\bibitem[BMS17]{references:id269}
\sphinxAtStartPar
Adam P Beall, Louis Moresi, and Tim Stern. Dripping or delamination? a range of mechanisms for removing the lower crust or lithosphere. 7 2017.
\bibitem[BB18]{references:id250}
\sphinxAtStartPar
Whitney M Behr and Thorsten W Becker. Sediment control on subduction plate speeds. \sphinxstyleemphasis{Earth Planet. Sci. Lett.}, 502:166–173, 10 2018.
\bibitem[BBK+16]{references:id213}
\sphinxAtStartPar
Charles Beichman, Bjoern Benneke, Heather Knutson, Roger Smith, Pierre\sphinxhyphen{}Olivier Lagage, Courtney Dressing, David Latham, Jonathan Lunine, Stephan Birkmann, Pierre Ferruit, Giovanna Giardino, Eliza Kempton, Sean Carey, Jessica Krick, Pieter D Deroo, Avi Mandell, Michael E Ressler, Avi Shporer, Mark Swain, Gautam Vasisht, George Ricker, Jeroen Bouwman, Ian Crossfield, Tom Greene, Steve Howell, Jessie Christiansen, David Ciardi, Mark Clampin, Matt Greenhouse, Alessandro Sozzetti, Paul Goudfrooij, Dean Hines, Tony Keyes, Janice Lee, Peter McCullough, Massimo Robberto, John Stansberry, Jeff Valenti, Marcia Rieke, George Rieke, Jonathan Fortney, Jacob Bean, Laura Kreidberg, David Ehrenreich, Drake Deming, Loïc Albert, René Doyon, and David Sing. Observations of transiting exoplanets with the james webb space telescope (JWST). \sphinxstyleemphasis{PASP}, 126(946):1134, 0 2016.
\bibitem[BWB08]{references:id276}
\sphinxAtStartPar
William B Bennett, Jingfeng Wang, and Rafael L Bras. Estimation of global ground heat flux. \sphinxstyleemphasis{J. Hydrometeorol.}, 9(4):744–759, 7 2008.
\bibitem[BTR15]{references:id331}
\sphinxAtStartPar
D Bercovici, P J Tackley, and Y Ricard. 7.07 \sphinxhyphen{} the generation of plate tectonics from mantle dynamics. In Gerald Schubert, editor, \sphinxstyleemphasis{Treatise on Geophysics (Second Edition)}, pages 271–318. Elsevier, Oxford, 0 2015.
\bibitem[Ber93]{references:id247}
\sphinxAtStartPar
David Bercovici. A simple model of plate generation from mantle flow. \sphinxstyleemphasis{Geophys. J. Int.}, 114(3):635–650, 8 1993.
\bibitem[Ber95]{references:id246}
\sphinxAtStartPar
David Bercovici. A source\sphinxhyphen{}sink model of the generation of plate tectonics from non\sphinxhyphen{}newtonian mantle flow. \sphinxstyleemphasis{J. Geophys. Res. {[}Solid Earth{]}}, 100(B2):2013–2030, 1995.
\bibitem[BRR00]{references:id456}
\sphinxAtStartPar
David Bercovici, Yanick Ricard, and Mark A Richards. The relation between mantle dynamics and plate tectonics: a primer. In \sphinxstyleemphasis{The History and Dynamics of Global Plate Motions}, pages 5–46. American Geophysical Union, 2000.
\bibitem[BB97]{references:id252}
\sphinxAtStartPar
Robert A Berner and Elizabeth K Berner. Silicate weathering and climate. In William F Ruddiman, editor, \sphinxstyleemphasis{Tectonic Uplift and Climate Change}, pages 353–365. Springer US, Boston, MA, 1997.
\bibitem[BF16]{references:id147}
\sphinxAtStartPar
Tanguy Bertrand and François Forget. Observed glacier and volatile distribution on pluto from atmosphere–topography processes. \sphinxstyleemphasis{Nature}, 540:86, 8 2016.
\bibitem[BMG+19]{references:id71}
\sphinxAtStartPar
Romain Beucher, Louis Moresi, Julian Giordani, John Mansour, Dan Sandiford, Rebecca Farrington, Luke Mondy, Claire Mallard, Patrice Rey, Guillaume Duclaux, and Others. UWGeodynamics: a teaching and research tool for numerical geodynamic modelling. \sphinxstyleemphasis{Journal of Open Source Software}, 4(36):1136, 2019.
\bibitem[BW83]{references:id254}
\sphinxAtStartPar
G E Birchfield and J Wertman. Topography, albedo\sphinxhyphen{}temperature feedback, and climate sensitivity. \sphinxstyleemphasis{Science}, 219(4582):284–285, 0 1983.
\bibitem[BBC+89]{references:id392}
\sphinxAtStartPar
B Blankenbach, F Busse, U Christensen, L Cserepes, D Gunkel, U Hansen, H Harder, G Jarvis, M Koch, G Marquart, D Moore, P Olson, H Schmeling, and T Schnaubelt. A benchmark comparison for mantle convection codes. \sphinxstyleemphasis{Geophys. J. Int.}, 98(1):23–38, 6 1989.
\bibitem[BRBH13]{references:id258}
\sphinxAtStartPar
Philipp A Brandl, Marcel Regelous, Christoph Beier, and Karsten M Haase. High mantle temperatures following rifting caused by continental insulation. \sphinxstyleemphasis{Nat. Geosci.}, 6:391, 2 2013.
\bibitem[BH82]{references:id59}
\sphinxAtStartPar
Alexander N Brooks and Thomas J R Hughes. Streamline upwind/Petrov\sphinxhyphen{}Galerkin formulations for convection dominated flows with particular emphasis on the incompressible Navier\sphinxhyphen{}Stokes equations. \sphinxstyleemphasis{Comput. Methods Appl. Mech. Eng.}, 32(1):199–259, 8 1982.
\bibitem[BSN13]{references:id122}
\sphinxAtStartPar
Suzanne H Broughton, Gale M Sinatra, and E Michael Nussbaum. “pluto has been a planet my whole life!” emotions, attitudes, and conceptual change in elementary students' learning about pluto's reclassification. \sphinxstyleemphasis{Research in Science Education}, 43(2):529–550, 2013.
\bibitem[BD73]{references:id235}
\sphinxAtStartPar
Kevin Burke and J F Dewey. Plume\sphinxhyphen{}Generated triple junctions: key indicators in applying plate tectonics to old rocks. \sphinxstyleemphasis{J. Geol.}, 81(4):406–433, 6 1973.
\bibitem[BM14]{references:id215}
\sphinxAtStartPar
Adam S Burrows and Geoffrey W Marcy. Exoplanets. introduction. \sphinxstyleemphasis{Proc. Natl. Acad. Sci. U. S. A.}, 111(35):12599–12600, 8 2014.
\bibitem[Can12]{references:id117}
\sphinxAtStartPar
Robin M Canup. Forming a moon with an earth\sphinxhyphen{}like composition via a giant impact. \sphinxstyleemphasis{Science}, 338(6110):1052–1055, 10 2012.
\bibitem[CMST79]{references:id191}
\sphinxAtStartPar
M H Carr, H Masursky, R G Strom, and R J Terrile. Volcanic features of io. \sphinxstyleemphasis{Nature}, 280(5725):729–733, 7 1979.
\bibitem[Car06]{references:id127}
\sphinxAtStartPar
Edwin Cartlidge. Pluto demotion divides astronomers. \sphinxstyleemphasis{Phys. World}, 19(10):11, 2006.
\bibitem[CKB+12]{references:id467}
\sphinxAtStartPar
A Cassan, D Kubas, J\sphinxhyphen{}P Beaulieu, M Dominik, K Horne, J Greenhill, J Wambsganss, J Menzies, A Williams, U G Jørgensen, A Udalski, D P Bennett, M D Albrow, V Batista, S Brillant, J A R Caldwell, A Cole, Ch Coutures, K H Cook, S Dieters, D Dominis Prester, J Donatowicz, P Fouqué, K Hill, N Kains, S Kane, J\sphinxhyphen{}B Marquette, R Martin, K R Pollard, K C Sahu, C Vinter, D Warren, B Watson, M Zub, T Sumi, M K Szymański, M Kubiak, R Poleski, I Soszynski, K Ulaczyk, G Pietrzyński, and L Wyrzykowski. One or more bound planets per milky way star from microlensing observations. \sphinxstyleemphasis{Nature}, 481(7380):167–169, 0 2012.
\bibitem[CHP+18]{references:id350}
\sphinxAtStartPar
Peter A Cawood, Chris J Hawkesworth, Sergei A Pisarevsky, Bruno Dhuime, Fabio A Capitanio, and Oliver Nebel. Geological archive of the onset of plate tectonics. \sphinxstyleemphasis{Philos. Trans. A Math. Phys. Eng. Sci.}, 9 2018.
\bibitem[Chr07]{references:id121}
\sphinxAtStartPar
L L Christensen. The pluto affair: when professionals talk to professionals with the public watching. \sphinxstyleemphasis{from Future Professional Communication in Astronomy}, 2007.
\bibitem[CTV15]{references:id205}
\sphinxAtStartPar
Ryan Cloutier, Daniel Tamayo, and Diana Valencia. COULD JUPITER OR SATURN HAVE EJECTED a FIFTH GIANT PLANET? \sphinxstyleemphasis{ApJ}, 813(1):8, 9 2015.
\bibitem[CRTL12]{references:id279}
\sphinxAtStartPar
N Coltice, T Rolf, P J Tackley, and S Labrosse. Dynamic causes of the relation between area and age of the ocean floor. \sphinxstyleemphasis{Science}, 336(6079):335–338, 3 2012.
\bibitem[CGeraultUlvrova17]{references:id461}
\sphinxAtStartPar
Nicolas Coltice, Mélanie Gérault, and Martina Ulvrová. A mantle convection perspective on global tectonics. \sphinxstyleemphasis{Earth\sphinxhyphen{}Sci. Rev.}, 165:120–150, 2017.
\bibitem[CGrigneH+12]{references:id409}
\sphinxAtStartPar
M Combes, C Grigné, L Husson, C P Conrad, S Le Yaouanq, M Parenthoën, C Tisseau, and J Tisseau. Multiagent simulation of evolutive plate tectonics applied to the thermal evolution of the earth. \sphinxstyleemphasis{Geochem. Geophys. Geosyst.}, 13(5):Q05006, 4 2012.
\bibitem[Con00]{references:id706}
\sphinxAtStartPar
Kent C Condie. Episodic continental growth models: afterthoughts and extensions. \sphinxstyleemphasis{Tectonophysics}, 322(1–2):153–162, 6 2000.
\bibitem[Con18]{references:id305}
\sphinxAtStartPar
Kent C Condie. A planet in transition: the onset of plate tectonics on earth between 3 and 2 ga? \sphinxstyleemphasis{Geoscience Frontiers}, 9(1):51–60, 0 2018.
\bibitem[CH01]{references:id255}
\sphinxAtStartPar
Clinton P Conrad and Bradford H Hager. Mantle convection with strong subduction zones. \sphinxstyleemphasis{Geophys. J. Int.}, 144(2):271–288, 1 2001.
\bibitem[CST13]{references:id265}
\sphinxAtStartPar
Clinton P Conrad, Bernhard Steinberger, and Trond H Torsvik. Stability of active mantle upwelling revealed by net characteristics of plate tectonics. \sphinxstyleemphasis{Nature}, 498(7455):479–482, 5 2013.
\bibitem[CLM06]{references:id262}
\sphinxAtStartPar
C M Cooper, A Lenardic, and L Moresi. Effects of continental insulation and the partitioning of heat producing elements on the earth's heat loss. \sphinxstyleemphasis{Geophys. Res. Lett.}, 33(13):4741, 2006.
\bibitem[CFL67]{references:id63}
\sphinxAtStartPar
R Courant, K Friedrichs, and H Lewy. On the partial difference equations of mathematical physics. \sphinxstyleemphasis{International Business Machines Corporation. Journal of Research and Development}, 11:215, 0 1967.
\bibitem[CUB+19]{references:id140}
\sphinxAtStartPar
Dale P Cruikshank, Orkan M Umurhan, Ross A Beyer, Bernard Schmitt, James T Keane, Kirby D Runyon, Dimitra Atri, Oliver L White, Isamu Matsuyama, Jeffrey M Moore, William B McKinnon, Scott A Sandford, Kelsi N Singer, William M Grundy, Cristina M Dalle Ore, Jason C Cook, Tanguy Bertrand, S Alan Stern, Catherine B Olkin, Harold A Weaver, Leslie A Young, John R Spencer, Carey M Lisse, Richard P Binzel, Alissa M Earle, Stuart J Robbins, G Randall Gladstone, Richard J Cartwright, and Kimberly Ennico. Recent cryovolcanism in virgil fossae on pluto. \sphinxstyleemphasis{Icarus}, 330:155–168, 8 2019.
\bibitem[CS80]{references:id57}
\sphinxAtStartPar
P A Cundall and O D L Strack. Discussion: a discrete numerical model for granular assemblies. \sphinxstyleemphasis{Géotechnique}, 30(3):331–336, 8 1980.
\bibitem[Cze17]{references:id159}
\sphinxAtStartPar
Leszek Czechowski. Present and future tectonics of enceladus. In \sphinxstyleemphasis{European Planetary Science Congress}, volume 11. meetingorganizer.copernicus.org, 2017.
\bibitem[DOCP+19]{references:id299}
\sphinxAtStartPar
C M Dalle Ore, D P Cruikshank, S Protopapa, F Scipioni, W B McKinnon, J C Cook, W M Grundy, B Schmitt, S A Stern, J M Moore, A Verbiscer, A H Parker, K N Singer, O M Umurhan, H A Weaver, C B Olkin, L A Young, and K Ennico. Detection of ammonia on pluto's surface in a region of geologically recent tectonism. \sphinxstyleemphasis{Science Advances}, 5(5):eaav5731, 4 2019.
\bibitem[DJ93]{references:id380}
\sphinxAtStartPar
Anne Davaille and Claude Jaupart. Transient high\sphinxhyphen{}rayleigh\sphinxhyphen{}number thermal convection with large viscosity variations. \sphinxstyleemphasis{J. Fluid Mech.}, 253:141–166, 7 1993.
\bibitem[DCS+08]{references:id188}
\sphinxAtStartPar
Ashley Gerard Davies, Julie Calkins, Lucas Scharenbroich, R Greg Vaughan, Robert Wright, Philip Kyle, Rebecca Castańo, Steve Chien, and Daniel Tran. Multi\sphinxhyphen{}instrument remote and in situ observations of the erebus volcano (antarctica) lava lake in 2005: a comparison with the pele lava lake on the jovian moon io. \sphinxstyleemphasis{J. Volcanol. Geotherm. Res.}, 177(3):705–724, 10 2008.
\bibitem[Dav06]{references:id722}
\sphinxAtStartPar
Geoffrey F Davies. Gravitational depletion of the early earth's upper mantle and the viability of early plate tectonics. \sphinxstyleemphasis{Earth Planet. Sci. Lett.}, 243(3–4):376–382, 2 2006.
\bibitem[DD10]{references:id108}
\sphinxAtStartPar
J H Davies and D R Davies. Earth's surface heat flux. \sphinxstyleemphasis{Solid Earth}, 1(1):5–24, 1 2010.
\bibitem[Dav13]{references:id430}
\sphinxAtStartPar
J Huw Davies. Global map of solid earth surface heat flow. \sphinxstyleemphasis{Geochem. Geophys. Geosyst.}, 14(10):4608–4622, 9 2013.
\bibitem[dPL10]{references:id474}
\sphinxAtStartPar
Imke de Pater and Jack J Lissauer. \sphinxstyleemphasis{Planetary Sciences}. Cambridge University Press, 2 edition edition, 7 2010.
\bibitem[DBONeill+09]{references:id239}
\sphinxAtStartPar
V Debaille, A D Brandon, C O'Neill, Q\sphinxhyphen{}Z Yin, and B Jacobsen. Early martian mantle overturn inferred from isotopic composition of nakhlite meteorites. \sphinxstyleemphasis{Nat. Geosci.}, 2(8):548–552, 7 2009.
\bibitem[DONeillB+13]{references:id717}
\sphinxAtStartPar
Vinciane Debaille, Craig O'Neill, Alan D Brandon, Pierre Haenecour, Qing\sphinxhyphen{}Zhu Yin, Nadine Mattielli, and Allan H Treiman. Stagnant\sphinxhyphen{}lid tectonics in early earth revealed by 142nd variations in late archean rocks. \sphinxstyleemphasis{Earth Planet. Sci. Lett.}, 373:83–92, 6 2013.
\bibitem[DBMA07]{references:id219}
\sphinxAtStartPar
James M Dohm, Victor R Baker, Shigenori Maruyama, and Robert C Anderson. Traits and evolution of the tharsis superplume, mars. In David A Yuen, Shigenori Maruyama, Shun\sphinxhyphen{}Ichiro Karato, and Brian F Windley, editors, \sphinxstyleemphasis{Superplumes: Beyond Plate Tectonics}, pages 523–536. Springer Netherlands, Dordrecht, 2007.
\bibitem[DHV17]{references:id420}
\sphinxAtStartPar
C Dorn, N R Hinkel, and J Venturini. Bayesian analysis of interiors of HD 219134b, kepler\sphinxhyphen{}10b, kepler\sphinxhyphen{}93b, CoRoT\sphinxhyphen{}7b, 55 cnc e, and HD 97658b using stellar abundance proxies. \sphinxstyleemphasis{N/A}, 0 2017. \sphinxhref{https://arxiv.org/abs/1609.03909}{arXiv:1609.03909}.
\bibitem[DVK+17]{references:id421}
\sphinxAtStartPar
C Dorn, J Venturini, A Khan, K Heng, Y Alibert, R Helled, A Rivoldini, and W Benz. A generalized bayesian inference method for constraining the interiors of super earths and sub\sphinxhyphen{}neptunes. \sphinxstyleemphasis{N/A}, 0 2017. \sphinxhref{https://arxiv.org/abs/1609.03908}{arXiv:1609.03908}.
\bibitem[DBR17]{references:id2}
\sphinxAtStartPar
Caroline Dorn, Dan J Bower, and Antoine Rozel. Assessing the interior structure of terrestrial exoplanets with implications for habitability. In Hans J Deeg and Juan Antonio Belmonte, editors, \sphinxstyleemphasis{Handbook of Exoplanets}, pages 1–25. Springer International Publishing, Cham, 2017.
\bibitem[DKN+06]{references:id173}
\sphinxAtStartPar
M K Dougherty, K K Khurana, F M Neubauer, C T Russell, J Saur, J S Leisner, and M E Burton. Identification of a dynamic atmosphere at enceladus with the cassini magnetometer. \sphinxstyleemphasis{Science}, 311(5766):1406–1409, 2 2006.
\bibitem[DB13]{references:id718}
\sphinxAtStartPar
P Driscoll and D Bercovici. Divergent evolution of earth and venus: influence of degassing, tectonics, and magnetic fields. \sphinxstyleemphasis{Icarus}, 226(2):1447–1464, 10 2013.
\bibitem[Dye12]{references:id85}
\sphinxAtStartPar
S T Dye. Geoneutrinos and the radioactive power of the earth. \sphinxstyleemphasis{Rev. Geophys.}, 50(3):1, 8 2012.
\bibitem[DA81]{references:id106}
\sphinxAtStartPar
Adam M Dziewonski and Don L Anderson. Preliminary reference earth model. \sphinxstyleemphasis{Phys. Earth Planet. Inter.}, 25(4):297–356, 5 1981.
\bibitem[EDS11]{references:id242}
\sphinxAtStartPar
Stephen M Elardo, David S Draper, and Charles K Shearer. Lunar magma ocean crystallization revisited: bulk composition, early cumulate mineralogy, and the source regions of the highlands mg\sphinxhyphen{}suite. \sphinxstyleemphasis{Geochim. Cosmochim. Acta}, 75(11):3024–3045, 5 2011.
\bibitem[ETVOHG02]{references:id241}
\sphinxAtStartPar
Linda T Elkins Tanton, James A Van Orman, Bradford H Hager, and Timothy L Grove. Re\sphinxhyphen{}examination of the lunar magma ocean cumulate overturn hypothesis: melting or mixing is required. \sphinxstyleemphasis{Earth Planet. Sci. Lett.}, 196(3):239–249, 2 2002.
\bibitem[ETBY11]{references:id243}
\sphinxAtStartPar
Linda T Elkins\sphinxhyphen{}Tanton, Seth Burgess, and Qing\sphinxhyphen{}Zhu Yin. The lunar magma ocean: reconciling the solidification process with lunar petrology and geochronology. \sphinxstyleemphasis{Earth Planet. Sci. Lett.}, 304(3):326–336, 3 2011.
\bibitem[ED04]{references:id703}
\sphinxAtStartPar
R E Ernst and D W Desnoyers. Lessons from venus for understanding mantle plumes on earth. \sphinxstyleemphasis{Phys. Earth Planet. Inter.}, 146:195–229, 2004.
\bibitem[EB01]{references:id226}
\sphinxAtStartPar
Richard E Ernst and Kenneth L Buchan. \sphinxstyleemphasis{Mantle Plumes: Their Identification Through Time}. Geological Society of America, 0 2001.
\bibitem[FMQ+05]{references:id73}
\sphinxAtStartPar
Rebecca Farrington, Louis Moresi, Steve Quenette, Robert Turnbull, and Patrick Sunter. Geodynamic benchmarking tests in HPC. In \sphinxstyleemphasis{APAC Conference, Gold Coast, Australia}. researchgate.net, 2005.
\bibitem[FG16]{references:id743}
\sphinxAtStartPar
R Fischer and T Gerya. Early earth plume\sphinxhyphen{}lid tectonics: a high\sphinxhyphen{}resolution 3D numerical modelling approach. \sphinxstyleemphasis{J. Geodyn.}, 100:198–214, 9 2016.
\bibitem[Fol15]{references:id416}
\sphinxAtStartPar
Bradford J Foley. THE ROLE OF PLATE TECTONIC–CLIMATE COUPLING AND EXPOSED LAND AREA IN THE DEVELOPMENT OF HABITABLE CLIMATES ON ROCKY PLANETS. \sphinxstyleemphasis{ApJ}, 812(1):36, 9 2015.
\bibitem[Fol18]{references:id348}
\sphinxAtStartPar
Bradford J Foley. The dependence of planetary tectonics on mantle thermal state: applications to early earth evolution. \sphinxstyleemphasis{Philos. Trans. A Math. Phys. Eng. Sci.}, 9 2018.
\bibitem[FBL12]{references:id202}
\sphinxAtStartPar
Bradford J Foley, David Bercovici, and William Landuyt. The conditions for plate tectonics on super\sphinxhyphen{}earths: inferences from convection models with damage. \sphinxstyleemphasis{Earth Planet. Sci. Lett.}, 331\sphinxhyphen{}332:281–290, 4 2012.
\bibitem[For14]{references:id201}
\sphinxAtStartPar
Eric B Ford. Architectures of planetary systems and implications for their formation. \sphinxstyleemphasis{Proc. Natl. Acad. Sci. U. S. A.}, 111(35):12616–12621, 8 2014.
\bibitem[FU75]{references:id272}
\sphinxAtStartPar
Donald Forsyth and Seiya Uyeda. On the relative importance of the driving forces of plate motion. \sphinxstyleemphasis{Geophys. J. Int.}, 43(1):163–200, 9 1975.
\bibitem[GGI+11]{references:id264}
\sphinxAtStartPar
A Gando, Y Gando, K Ichimura, H Ikeda, K Inoue, Y Kibe, Y Kishimoto, M Koga, Y Minekawa, T Mitsui, T Morikawa, N Nagai, K Nakajima, K Nakamura, K Narita, I Shimizu, Y Shimizu, J Shirai, F Suekane, A Suzuki, H Takahashi, N Takahashi, Y Takemoto, K Tamae, H Watanabe, B D Xu, H Yabumoto, H Yoshida, S Yoshida, S Enomoto, A Kozlov, H Murayama, C Grant, G Keefer, A Piepke, T I Banks, T Bloxham, J A Detwiler, S J Freedman, B K Fujikawa, K Han, R Kadel, T O'Donnell, H M Steiner, D A Dwyer, R D McKeown, C Zhang, B E Berger, C E Lane, J Maricic, T Miletic, M Batygov, J G Learned, S Matsuno, M Sakai, G A Horton\sphinxhyphen{}Smith, K E Downum, G Gratta, K Tolich, Y Efremenko, O Perevozchikov, H J Karwowski, D M Markoff, W Tornow, K M Heeger, M P Decowski, and The KamLAND Collaboration. Partial radiogenic heat model for earth revealed by geoneutrino measurements. \sphinxstyleemphasis{Nat. Geosci.}, 4(9):647–651, 8 2011.
\bibitem[GSB+15]{references:id233}
\sphinxAtStartPar
T V Gerya, R J Stern, M Baes, S V Sobolev, and S A Whattam. Plate tectonics on the earth triggered by plume\sphinxhyphen{}induced subduction initiation. \sphinxstyleemphasis{Nature}, 527(7577):221–225, 10 2015.
\bibitem[GTD+17]{references:id466}
\sphinxAtStartPar
Michaël Gillon, Amaury H M J Triaud, Brice\sphinxhyphen{}Olivier Demory, Emmanuël Jehin, Eric Agol, Katherine M Deck, Susan M Lederer, Julien de Wit, Artem Burdanov, James G Ingalls, Emeline Bolmont, Jeremy Leconte, Sean N Raymond, Franck Selsis, Martin Turbet, Khalid Barkaoui, Adam Burgasser, Matthew R Burleigh, Sean J Carey, Aleksander Chaushev, Chris M Copperwheat, Laetitia Delrez, Catarina S Fernandes, Daniel L Holdsworth, Enrico J Kotze, Valérie Van Grootel, Yaseen Almleaky, Zouhair Benkhaldoun, Pierre Magain, and Didier Queloz. Seven temperate terrestrial planets around the nearby ultracool dwarf star TRAPPIST\sphinxhyphen{}1. \sphinxstyleemphasis{Nature}, 542(7642):456–460, 1 2017.
\bibitem[GSE+16]{references:id150}
\sphinxAtStartPar
G Randall Gladstone, S Alan Stern, Kimberly Ennico, Catherine B Olkin, Harold A Weaver, Leslie A Young, Michael E Summers, Darrell F Strobel, David P Hinson, Joshua A Kammer, Alex H Parker, Andrew J Steffl, Ivan R Linscott, Joel Wm Parker, Andrew F Cheng, David C Slater, Maarten H Versteeg, Thomas K Greathouse, Kurt D Retherford, Henry Throop, Nathaniel J Cunningham, William W Woods, Kelsi N Singer, Constantine C C Tsang, Eric Schindhelm, Carey M Lisse, Michael L Wong, Yuk L Yung, Xun Zhu, Werner Curdt, Panayotis Lavvas, Eliot F Young, G Leonard Tyler, and New Horizons Science Team. The atmosphere of pluto as observed by new horizons. \sphinxstyleemphasis{Science}, 351(6279):aad8866, 2 2016.
\bibitem[GKuzmin15]{references:id99}
\sphinxAtStartPar
M Z Glukhovskii and M I Kuz'min. Extraterrestrial factors and their role in the earth's tectonic evolution in the early precambrian. \sphinxstyleemphasis{Russ. Geol. Geophys.}, 56(7):959–977, 2015.
\bibitem[GBONeill+14]{references:id404}
\sphinxAtStartPar
W L Griffin, E A Belousova, C O'Neill, Suzanne Y O'Reilly, V Malkovets, N J Pearson, S Spetsius, and S A Wilde. The world turns over: Hadean–Archean crust–mantle evolution. \sphinxstyleemphasis{Lithos}, 189:2–15, 1 2014.
\bibitem[GrigneLT05]{references:id459}
\sphinxAtStartPar
C Grigné, S Labrosse, and P J Tackley. Convective heat transfer as a function of wavelength: implications for the cooling of the earth. \sphinxstyleemphasis{J. Geophys. Res.}, 110(B3):B03409, 2 2005.
\bibitem[GBB+16]{references:id149}
\sphinxAtStartPar
W M Grundy, R P Binzel, B J Buratti, J C Cook, D P Cruikshank, C M Dalle Ore, A M Earle, K Ennico, C J A Howett, A W Lunsford, C B Olkin, A H Parker, S Philippe, S Protopapa, E Quirico, D C Reuter, B Schmitt, K N Singer, A J Verbiscer, R A Beyer, M W Buie, A F Cheng, D E Jennings, I R Linscott, J Wm Parker, P M Schenk, J R Spencer, J A Stansberry, S A Stern, H B Throop, C C C Tsang, H A Weaver, G E Weigle, 2nd, L A Young, and New Horizons Science Team. Surface compositions across pluto and charon. \sphinxstyleemphasis{Science}, 351(6279):aad9189, 2 2016.
\bibitem[Gur88]{references:id260}
\sphinxAtStartPar
Michael Gurnis. Large\sphinxhyphen{}scale mantle convection and the aggregation and dispersal of supercontinents. \sphinxstyleemphasis{Nature}, 332(6166):695–699, 3 1988.
\bibitem[GuntherPD+19]{references:id206}
\sphinxAtStartPar
Maximilian N Günther, Francisco J Pozuelos, Jason A Dittmann, Diana Dragomir, Stephen R Kane, Tansu Daylan, Adina D Feinstein, Chelsea X Huang, Timothy D Morton, Andrea Bonfanti, L G Bouma, Jennifer Burt, Karen A Collins, Jack J Lissauer, Elisabeth Matthews, Benjamin T Montet, Andrew Vanderburg, Songhu Wang, Jennifer G Winters, George R Ricker, Roland K Vanderspek, David W Latham, Sara Seager, Joshua N Winn, Jon M Jenkins, James D Armstrong, Khalid Barkaoui, Natalie Batalha, Jacob L Bean, Douglas A Caldwell, David R Ciardi, Kevin I Collins, Ian Crossfield, Michael Fausnaugh, Gabor Furesz, Tianjun Gan, Michaël Gillon, Natalia Guerrero, Keith Horne, Steve B Howell, Michael Ireland, Giovanni Isopi, Emmanuël Jehin, John F Kielkopf, Sebastien Lepine, Franco Mallia, Rachel A Matson, Gordon Myers, Enric Palle, Samuel N Quinn, Howard M Relles, Bárbara Rojas\sphinxhyphen{}Ayala, Joshua Schlieder, Ramotholo Sefako, Avi Shporer, Juan C Suárez, Thiam\sphinxhyphen{}Guan Tan, Eric B Ting, Jose Twicken, and Ian A Waite. A super\sphinxhyphen{}earth and two sub\sphinxhyphen{}neptunes transiting the nearby and quiet M dwarf TOI\sphinxhyphen{}270. \sphinxstyleemphasis{Nature Astronomy}, 6 2019.
\bibitem[HLR06]{references:id60}
\sphinxAtStartPar
Ernst Hairer, Christian Lubich, and Michel Roche. \sphinxstyleemphasis{The Numerical Solution of Differential\sphinxhyphen{}Algebraic Systems by Runge\sphinxhyphen{}Kutta Methods}. Springer, 10 2006.
\bibitem[HBP16]{references:id155}
\sphinxAtStartPar
Noah P Hammond, Amy C Barr, and Edgar M Parmentier. Recent tectonic activity on pluto driven by phase changes in the ice shell: THERMAL EVOLUTION OF PLUTO. \sphinxstyleemphasis{Geophys. Res. Lett.}, 43(13):6775–6782, 6 2016.
\bibitem[HWW+14]{references:id471}
\sphinxAtStartPar
Eunkyu Han, Sharon X Wang, Jason T Wright, Y Katherina Feng, Ming Zhao, Onsi Fakhouri, Jacob I Brown, and Colin Hancock. Exoplanet orbit database. II. updates to exoplanets.org. \sphinxstyleemphasis{Publ. Astro. Soc. Pac.}, 126:827, 2014.
\bibitem[Han15]{references:id141}
\sphinxAtStartPar
Eric Hand. Scientists ponder an improbably active pluto. \sphinxstyleemphasis{Science}, 349(6246):352–353, 6 2015.
\bibitem[HCF+82]{references:id178}
\sphinxAtStartPar
R Hanel, B Conrath, F M Flasar, V Kunde, W Maguire, J Pearl, J Pirraglia, R Samuelson, D Cruikshank, D Gautier, P Gierasch, L Horn, and C Ponnamperuma. Infrared observations of the saturnian system from voyager 2. \sphinxstyleemphasis{Science}, 215(4532):544–548, 0 1982.
\bibitem[HL11]{references:id263}
\sphinxAtStartPar
Philip J Heron and Julian P Lowman. The effects of supercontinent size and thermal insulation on the formation of mantle plumes. \sphinxstyleemphasis{Tectonophysics}, 510(1):28–38, 0 2011.
\bibitem[HL14]{references:id259}
\sphinxAtStartPar
Philip J Heron and Julian P Lowman. The impact of rayleigh number on assessing the significance of supercontinent insulation: HERON AND LOWMAN. \sphinxstyleemphasis{J. Geophys. Res. {[}Solid Earth{]}}, 119(1):711–733, 0 2014.
\bibitem[HCDG92]{references:id228}
\sphinxAtStartPar
R I Hill, I H Campbell, G F Davies, and R W Griffiths. Mantle plumes and continental tectonics. \sphinxstyleemphasis{Science}, 256(5054):186–193, 3 1992.
\bibitem[HCG91]{references:id227}
\sphinxAtStartPar
R I Hill, I H Campbell, and R W Griffiths. Plume tectonics and the development of stable continental crust. \sphinxstyleemphasis{Explor. Geophys.}, 22(1):185–188, 1991.
\bibitem[Hog06]{references:id132}
\sphinxAtStartPar
Jenny Hogan. Pluto: the backlash begins. \sphinxstyleemphasis{Nature}, 442(7106):965–966, 7 2006.
\bibitem[HYZ13]{references:id685}
\sphinxAtStartPar
Jinshui Huang, An Yang, and Shijie Zhong. Constraints of the topography, gravity and volcanism on venusian mantle dynamics and generation of plate tectonics. \sphinxstyleemphasis{Earth Planet. Sci. Lett.}, 362(C):207–214, 0 2013.
\bibitem[HCM+13]{references:id386}
\sphinxAtStartPar
Yu Huang, Viacheslav Chubakov, Fabio Mantovani, Roberta L Rudnick, and William F McDonough. A reference earth model for the heat\sphinxhyphen{}producing elements and associated geoneutrino flux: EARTH MODEL HPE GEONEUTRINO. \sphinxstyleemphasis{Geochem. Geophys. Geosyst.}, 14(6):2003–2029, 5 2013.
\bibitem[Hug12]{references:id69}
\sphinxAtStartPar
Thomas J R Hughes. \sphinxstyleemphasis{The Finite Element Method: Linear Static and Dynamic Finite Element Analysis}. Courier Corporation, 4 2012.
\bibitem[HBS+05]{references:id560}
\sphinxAtStartPar
T A Hurford, R A Beyer, B Schmidt, B Preblich, A R Sarid, and R Greenberg. Flexure of europa's lithosphere due to ridge\sphinxhyphen{}loading. \sphinxstyleemphasis{Icarus}, 177:380–396, 2005.
\bibitem[HS02]{references:id583}
\sphinxAtStartPar
Hauke Hussmann and Tilman Spohn. Thermal equilibrium states of europa's ice shell: implications for internal ocean thickness and surface heat flow. \sphinxstyleemphasis{Icarus}, 156:143–151, 2002.
\bibitem[HKCN90]{references:id249}
\sphinxAtStartPar
William T Hyde, Kwang\sphinxhyphen{}Yul Kim, Thomas J Crowley, and Gerald R North. On the relation between polar continentality and climate: studies with a nonlinear seasonal energy balance model. \sphinxstyleemphasis{J. Geophys. Res.}, 95(D11):18653, 1990.
\bibitem[HRvSramekZ11]{references:id218}
\sphinxAtStartPar
Brian M Hynek, Stuart J Robbins, Ondřej Šrámek, and Shijie J Zhong. Geological evidence for a migrating tharsis plume on early mars. \sphinxstyleemphasis{Earth Planet. Sci. Lett.}, 310(3):327–333, 9 2011.
\bibitem[HoinkLJ13]{references:id439}
\sphinxAtStartPar
Tobias Höink, Adrian Lenardic, and A Mark Jellinek. Earth's thermal evolution with multiple convection modes: a Monte\sphinxhyphen{}Carlo approach. \sphinxstyleemphasis{Phys. Earth Planet. Inter.}, 221:22–26, 2013.
\bibitem[ISP+14]{references:id164}
\sphinxAtStartPar
L Iess, D J Stevenson, M Parisi, D Hemingway, R A Jacobson, J I Lunine, F Nimmo, J W Armstrong, S W Asmar, M Ducci, and P Tortora. The gravity field and interior structure of enceladus. \sphinxstyleemphasis{Science}, 344(6179):78–80, 3 2014.
\bibitem[JM09]{references:id128}
\sphinxAtStartPar
Ruth Jarman and Billy McClune. `a planet of confusion and debate': children's and young people's response to the news coverage of pluto's loss of planetary status. \sphinxstyleemphasis{Research in Science \& Technological Education}, 27(3):309–325, 2009.
\bibitem[JM10]{references:id104}
\sphinxAtStartPar
Claude Jaupart and Jean\sphinxhyphen{}Claude Mareschal. \sphinxstyleemphasis{Heat Generation and Transport in the Earth}. Cambridge University Press, 10 2010.
\bibitem[JP85]{references:id78}
\sphinxAtStartPar
Claude Jaupart and Barry Parsons. Convective instabilities in a variable viscosity fluid cooled from above. \sphinxstyleemphasis{Phys. Earth Planet. Inter.}, 39(1):14–32, 5 1985.
\bibitem[JR03]{references:id222}
\sphinxAtStartPar
Catherine L Johnson and Mark A Richards. A conceptual model for the relationship between coronae and large\sphinxhyphen{}scale mantle dynamics on venus. \sphinxstyleemphasis{Journal of Geophysical Research: Planets}, 2003.
\bibitem[KNS+19]{references:id142}
\sphinxAtStartPar
Shunichi Kamata, Francis Nimmo, Yasuhito Sekine, Kiyoshi Kuramoto, Naoki Noguchi, Jun Kimura, and Atsushi Tani. Pluto's ocean is capped and insulated by gas hydrates. \sphinxstyleemphasis{Nat. Geosci.}, 12(6):407, 2019.
\bibitem[KY18]{references:id312}
\sphinxAtStartPar
Masanori Kameyama and Mayumi Yamamoto. Numerical experiments on thermal convection of highly compressible fluids with variable viscosity and thermal conductivity: implications for mantle convection of super\sphinxhyphen{}earths. \sphinxstyleemphasis{Phys. Earth Planet. Inter.}, 274:23–36, 0 2018.
\bibitem[KM16]{references:id183}
\sphinxAtStartPar
Duminda G J Kankanamge and William B Moore. Heat transport in the hadean mantle: from heat pipes to plates. \sphinxstyleemphasis{Geophys. Res. Lett.}, 43(7):3208–3214, 3 2016.
\bibitem[Kan81]{references:id64}
\sphinxAtStartPar
Immanuel Kant. \sphinxstyleemphasis{Critique of Pure Reason}. Cambridge University Press, 1781.
\bibitem[Kar14]{references:id735}
\sphinxAtStartPar
Shun\sphinxhyphen{}Ichiro Karato. Some remarks on the models of plate tectonics on terrestrial planets: from the view\sphinxhyphen{}point of mineral physics. \sphinxstyleemphasis{Tectonophysics}, 631:4–13, 8 2014.
\bibitem[Kar83]{references:id167}
\sphinxAtStartPar
J S Kargel. Enceladus: an analog of terrestrial plate tectonism? \sphinxstyleemphasis{Lunar and Planetary Science Conference}, 1983.
\bibitem[KP96]{references:id170}
\sphinxAtStartPar
Jeffrey S Kargel and Stefania Pozio. The volcanic and tectonic history of enceladus. \sphinxstyleemphasis{Icarus}, 119(2):385–404, 1 1996.
\bibitem[KM06]{references:id586}
\sphinxAtStartPar
Simon A Kattenhorn and Scott T Marshall. Fault\sphinxhyphen{}induced perturbed stress fields and associated tensile and compressive deformation at fault tips in the ice shell of europa: implications for fault mechanics. \sphinxstyleemphasis{J. Struct. Geol.}, 28:2204, 2006.
\bibitem[KKV09]{references:id284}
\sphinxAtStartPar
P Kearey, Keith A Klepeis, and F J Vine. \sphinxstyleemphasis{Global tectonics}. Wiley\sphinxhyphen{}Blackwell, 3rd ed. edition, 2009.
\bibitem[KWK16]{references:id244}
\sphinxAtStartPar
Gavin G Kenny, Martin J Whitehouse, and Balz S Kamber. Differentiated impact melt sheets may be a potential source of hadean detrital zircon. \sphinxstyleemphasis{Geology}, 44(6):435–438, 5 2016.
\bibitem[Kie03]{references:id238}
\sphinxAtStartPar
Walter S Kiefer. Melting in the martian mantle: shergottite formation and implications for present\sphinxhyphen{}day mantle convection on mars. \sphinxstyleemphasis{Meteorit. Planet. Sci.}, 38(12):1815–1832, 11 2003.
\bibitem[KRS+14]{references:id296}
\sphinxAtStartPar
Ravi Kumar Kopparapu, Ramses M Ramirez, James SchottelKotte, James F Kasting, Shawn Domagal\sphinxhyphen{}Goldman, and Vincent Eymet. Habitable zones around Main\sphinxhyphen{}Sequence stars: dependence on planetary mass. \sphinxstyleemphasis{N/A}, 3 2014. \sphinxhref{https://arxiv.org/abs/1404.5292}{arXiv:1404.5292}.
\bibitem[Kor03]{references:id457}
\sphinxAtStartPar
Jun Korenaga. Energetics of mantle convection and the fate of fossil heat. \sphinxstyleemphasis{Geophys. Res. Lett.}, 30(8):1437, 3 2003.
\bibitem[Kor08]{references:id292}
\sphinxAtStartPar
Jun Korenaga. Urey ratio and the structure and evolution of earth's mantle. \sphinxstyleemphasis{Rev. Geophys.}, 46(2):11, 5 2008.
\bibitem[Kor10]{references:id431}
\sphinxAtStartPar
Jun Korenaga. ON THE LIKELIHOOD OF PLATE TECTONICS ON SUPER\sphinxhyphen{}EARTHS: DOES SIZE MATTER? \sphinxstyleemphasis{ApJL}, 725(1):L43, 10 2010.
\bibitem[Kor11]{references:id402}
\sphinxAtStartPar
Jun Korenaga. Clairvoyant geoneutrinos. \sphinxstyleemphasis{Nat. Geosci.}, 4:581, 7 2011.
\bibitem[KLL+06]{references:id112}
\sphinxAtStartPar
Yu N Kulikov, H Lammer, H I M Lichtenegger, N Terada, I Ribas, C Kolb, D Langmayr, R Lundin, E F Guinan, S Barabash, and H K Biernat. Atmospheric and water loss from early venus. \sphinxstyleemphasis{Planet. Space Sci.}, 54(13):1425–1444, 10 2006.
\bibitem[KM94]{references:id231}
\sphinxAtStartPar
Mineo Kumazawa and Shigenori Maruyama. Whole earth tectonics. In \sphinxstyleemphasis{Journal\sphinxhyphen{}Geological Society of Japan}, pages 81–102. 1994.
\bibitem[KK00]{references:id193}
\sphinxAtStartPar
Oleg L Kuskov and Victor A Kronrod. Resemblance and difference between the constitution of the moon and io. \sphinxstyleemphasis{Planet. Space Sci.}, 48(7):717–726, 5 2000.
\bibitem[LHB08]{references:id762}
\sphinxAtStartPar
Thorne Lay, John Hernlund, and Bruce A Buffett. Core–mantle boundary heat flow. \sphinxstyleemphasis{Nat. Geosci.}, 1(1):25–32, 0 2008.
\bibitem[Len18]{references:id351}
\sphinxAtStartPar
A Lenardic. The diversity of tectonic modes and thoughts about transitions between them. \sphinxstyleemphasis{Philosophical Transactions of the Royal Society A: Mathematical, Physical and Engineering Sciences}, 10 2018.
\bibitem[LCM11]{references:id400}
\sphinxAtStartPar
A Lenardic, C M Cooper, and L Moresi. A note on continents and the earth's urey ratio. \sphinxstyleemphasis{Phys. Earth Planet. Inter.}, 188(1):127–130, 8 2011.
\bibitem[LC12]{references:id435}
\sphinxAtStartPar
A Lenardic and J W Crowley. ON THE NOTION OF WELL\sphinxhyphen{}DEFINED TECTONIC REGIMES FOR TERRESTRIAL PLANETS IN THIS SOLAR SYSTEM AND OTHERS. \sphinxstyleemphasis{ApJ}, 755(2):132, 7 2012.
\bibitem[LCJW16]{references:id94}
\sphinxAtStartPar
A Lenardic, J W Crowley, A M Jellinek, and M Weller. The solar system of forking paths: bifurcations in planetary evolution and the search for Life\sphinxhyphen{}Bearing planets in our galaxy. \sphinxstyleemphasis{Astrobiology}, 16(7):551–559, 6 2016.
\bibitem[LJF+16]{references:id367}
\sphinxAtStartPar
A Lenardic, A M Jellinek, B Foley, C O'Neill, and W B Moore. Climate\sphinxhyphen{}tectonic coupling: variations in the mean, variations about the mean, and variations in mode: TECTONIC\sphinxhyphen{}CLIMATE COUPLING. \sphinxstyleemphasis{J. Geophys. Res. Planets}, 121(10):1831–1864, 9 2016.
\bibitem[LJM08]{references:id674}
\sphinxAtStartPar
A Lenardic, A M Jellinek, and L Moresi. A climate induced transition in the tectonic style of a terrestrial planet. \sphinxstyleemphasis{Earth Planet. Sci. Lett.}, 271:34–42, 2008.
\bibitem[LMJM05]{references:id676}
\sphinxAtStartPar
A Lenardic, L\sphinxhyphen{}N Moresi, A M Jellinek, and M Manga. Continental insulation, mantle cooling, and the surface area of oceans and continents. \sphinxstyleemphasis{Earth Planet. Sci. Lett.}, 234(3–4):317–333, 5 2005.
\bibitem[LR05]{references:id555}
\sphinxAtStartPar
Jere H Lipps and Sarah Rieboldt. Habitats and taphonomy of europa. \sphinxstyleemphasis{Icarus}, 177:515–527, 2005.
\bibitem[LBS98]{references:id271}
\sphinxAtStartPar
Carolina Lithgow\sphinxhyphen{}Bertelloni and Paul G Silver. Dynamic topography, plate driving forces and the african superswell. \sphinxstyleemphasis{Nature}, 395:269–272, 8 1998.
\bibitem[LRH+16]{references:id414}
\sphinxAtStartPar
Zac Yung\sphinxhyphen{}Chun Liu, Jani Radebaugh, Ron A Harris, Eric H Christiansen, Catherine D Neish, Randolph L Kirk, and Ral Lorenz. The tectonics of titan: global structural mapping from cassini RADAR. \sphinxstyleemphasis{Icarus}, 270:14–29, 4 2016.
\bibitem[LF98]{references:id473}
\sphinxAtStartPar
Katharina Lodders and Bruce Fegley. \sphinxstyleemphasis{The Planetary Scientist's Companion}. Oxford University Press, 1 edition edition, 11 1998.
\bibitem[Lor63]{references:id75}
\sphinxAtStartPar
Edward N Lorenz. Deterministic nonperiodic flow. \sphinxstyleemphasis{J. Atmos. Sci.}, 20(2):130–141, 2 1963.
\bibitem[LourenccoRT16]{references:id338}
\sphinxAtStartPar
Diogo L Lourenço, Antoine Rozel, and Paul J Tackley. Melting\sphinxhyphen{}induced crustal production helps plate tectonics on earth\sphinxhyphen{}like planets. \sphinxstyleemphasis{Earth Planet. Sci. Lett.}, 439:18–28, 3 2016.
\bibitem[LT12]{references:id282}
\sphinxAtStartPar
Frederick K Lutgens and Edward J Tarbuck. \sphinxstyleemphasis{Essentials of geology}. Prentice Hall, 11th ed. edition, 2012.
\bibitem[Mac98]{references:id278}
\sphinxAtStartPar
Ken C Macdonald. Exploring the global mid\sphinxhyphen{}ocean ridge: a quarter\sphinxhyphen{}century of discovery. \sphinxstyleemphasis{Oceanus}, pages 2, 1998.
\bibitem[MC15]{references:id768}
\sphinxAtStartPar
I N Machulin and Borexino Collaboration. Geoneutrinos. \sphinxstyleemphasis{Phys. At. Nucl.}, 78(14):1613–1616, 11 2015.
\bibitem[MV15]{references:id55}
\sphinxAtStartPar
A Mahendran and A Vedaldi. Understanding deep image representations by inverting them. In \sphinxstyleemphasis{2015 IEEE Conference on Computer Vision and Pattern Recognition (CVPR)}, 5188–5196. 5 2015.
\bibitem[MJPP12]{references:id401}
\sphinxAtStartPar
Jean\sphinxhyphen{}Claude Mareschal, Claude Jaupart, Catherine Phaneuf, and Claire Perry. Geoneutrinos and the energy budget of the earth. \sphinxstyleemphasis{J. Geodyn.}, 54:43–54, 2 2012.
\bibitem[MM09]{references:id148}
\sphinxAtStartPar
Laurence A Marschall and Stephen P Maran. \sphinxstyleemphasis{Pluto Confidential: An Insider Account of the Ongoing Battles Over the Status of Pluto}. BenBella Books, 2009.
\bibitem[MWK88]{references:id251}
\sphinxAtStartPar
H G Marshall, J C Walker, and W R Kuhn. Long\sphinxhyphen{}term climate change and the geochemical cycle of carbon. \sphinxstyleemphasis{J. Geophys. Res.}, 93(D1):791–801, 0 1988.
\bibitem[ML16]{references:id454}
\sphinxAtStartPar
Rebecca G Martin and Mario Livio. On the formation of Super\sphinxhyphen{}Earths with implications for the solar system. \sphinxstyleemphasis{N/A}, 2 2016. \sphinxhref{https://arxiv.org/abs/1603.08145}{arXiv:1603.08145}.
\bibitem[Mar94]{references:id232}
\sphinxAtStartPar
Shigenori Maruyama. Plume tectonics. In \sphinxstyleemphasis{Journal\sphinxhyphen{}Geological Survey of Japan}, pages 24–49. 1994.
\bibitem[MMBM10]{references:id581}
\sphinxAtStartPar
W G Mason, L Moresi, P G Betts, and M S Miller. Three\sphinxhyphen{}dimensional numerical models of the influence of a buoyant oceanic plateau on subduction zones. \sphinxstyleemphasis{Tectonophysics}, 483(1–2):71–79, 2 2010.
\bibitem[MBJ+98]{references:id186}
\sphinxAtStartPar
D Matson, D Blaney, T Johnson, G Veeder, and A Davis. Io and the early earth. \sphinxstyleemphasis{N/A}, 1998.
\bibitem[MM08]{references:id72}
\sphinxAtStartPar
Dave A May and Louis Moresi. Preconditioned iterative methods for stokes flow problems arising in computational geodynamics. \sphinxstyleemphasis{Phys. Earth Planet. Inter.}, 171(1):33–47, 11 2008.
\bibitem[MQ95]{references:id156}
\sphinxAtStartPar
Michel Mayor and Didier Queloz. A jupiter\sphinxhyphen{}mass companion to a solar\sphinxhyphen{}type star. \sphinxstyleemphasis{Nature}, 378(6555):355–359, 10 1995.
\bibitem[MKS+98]{references:id187}
\sphinxAtStartPar
A S McEwen, L Keszthelyi, J R Spencer, G Schubert, D L Matson, R Lopes\sphinxhyphen{}Gautier, K P Klaasen, T V Johnson, J W Head, P Geissler, S Fagents, A G Davies, M H Carr, H H Breneman, and M J Belton. High\sphinxhyphen{}temperature silicate volcanism on jupiter's moon io. \sphinxstyleemphasis{Science}, 281(5373):87–90, 6 1998.
\bibitem[MLGKK00]{references:id195}
\sphinxAtStartPar
Alfred S McEwen, Rosaly Lopes\sphinxhyphen{}Gautier, Laszlo Keszthelyi, and Susan W Kieffer. Extreme volcanism on jupiter's moon io. In James R Zimbelman and Tracy K P Gregg, editors, \sphinxstyleemphasis{Environmental Effects on Volcanic Eruptions: From Deep Oceans to Deep Space}, pages 179–205. Springer US, Boston, MA, 2000.
\bibitem[MRW74]{references:id80}
\sphinxAtStartPar
D P Mckenzie, J M Roberts, and N O Weiss. Convection in the earth's mantle: towards a numerical simulation. \sphinxstyleemphasis{J. Fluid Mech.}, 62(3):465–538, 1 1974.
\bibitem[McK87]{references:id171}
\sphinxAtStartPar
William B McKinnon. Jovian and saturnian satellites. \sphinxstyleemphasis{Rev. Geophys.}, 25(2):260, 1987.
\bibitem[MT93]{references:id115}
\sphinxAtStartPar
H J Melosh and W B Tonks. Swapping rocks: ejection and exchange of surface material among the terrestrial planets. \sphinxstyleemphasis{Meteoritics}, 1993.
\bibitem[Mes10]{references:id131}
\sphinxAtStartPar
Lisa R Messeri. The problem with pluto: conflicting cosmologies and the classification of planets. \sphinxstyleemphasis{Soc. Stud. Sci.}, 40(2):187–214, 3 2010.
\bibitem[Mil88]{references:id56}
\sphinxAtStartPar
Alan Alexander Milne. \sphinxstyleemphasis{The House at Pooh Corner}. Dutton, 1988.
\bibitem[Mon05]{references:id58}
\sphinxAtStartPar
J J Monaghan. Smoothed particle hydrodynamics. \sphinxstyleemphasis{Rep. Prog. Phys.}, 68(8):1703, 6 2005.
\bibitem[MMS+16]{references:id152}
\sphinxAtStartPar
Jeffrey M Moore, William B McKinnon, John R Spencer, Alan D Howard, Paul M Schenk, Ross A Beyer, Francis Nimmo, Kelsi N Singer, Orkan M Umurhan, Oliver L White, S Alan Stern, Kimberly Ennico, Cathy B Olkin, Harold A Weaver, Leslie A Young, Richard P Binzel, Marc W Buie, Bonnie J Buratti, Andrew F Cheng, Dale P Cruikshank, Will M Grundy, Ivan R Linscott, Harold J Reitsema, Dennis C Reuter, Mark R Showalter, Veronica J Bray, Carrie L Chavez, Carly J A Howett, Tod R Lauer, Carey M Lisse, Alex Harrison Parker, S B Porter, Stuart J Robbins, Kirby Runyon, Ted Stryk, Henry B Throop, Constantine C C Tsang, Anne J Verbiscer, Amanda M Zangari, Andrew L Chaikin, Don E Wilhelms, and New Horizons Science Team. The geology of pluto and charon through the eyes of new horizons. \sphinxstyleemphasis{Science}, 351(6279):1284–1293, 2 2016.
\bibitem[Moo01]{references:id181}
\sphinxAtStartPar
William B Moore. The thermal state of io. \sphinxstyleemphasis{Icarus}, 154(2):548–550, 11 2001.
\bibitem[Moo08]{references:id81}
\sphinxAtStartPar
William B Moore. Heat transport in a convecting layer heated from within and below. \sphinxstyleemphasis{J. Geophys. Res.}, 113(B11):93, 10 2008.
\bibitem[MSAS07]{references:id194}
\sphinxAtStartPar
William B Moore, Gerald Schubert, John D Anderson, and John R Spencer. The interior of io. In Rosaly M C Lopes and John R Spencer, editors, \sphinxstyleemphasis{Io After Galileo: A New View of Jupiter's Volcanic Moon}, pages 89–108. Springer Berlin Heidelberg, Berlin, Heidelberg, 2007.
\bibitem[MSW17]{references:id391}
\sphinxAtStartPar
William B Moore, Justin I Simon, and A Alexander G Webb. Heat\sphinxhyphen{}pipe planets. \sphinxstyleemphasis{Earth Planet. Sci. Lett.}, 474:13–19, 8 2017.
\bibitem[MW13]{references:id767}
\sphinxAtStartPar
William B Moore and A Alexander G Webb. Heat\sphinxhyphen{}pipe earth. \sphinxstyleemphasis{Nature}, 501(7468):501–505, 8 2013.
\bibitem[MDBM03]{references:id552}
\sphinxAtStartPar
L Moresi, F Dufour, and H \sphinxhyphen{}B. Muuhlhaus. A lagrangian integration point finite element method for large deformation modeling of viscoelastic geomaterials. \sphinxstyleemphasis{J. Comput. Phys.}, 184:476–497, 2003.
\bibitem[MDM02]{references:id384}
\sphinxAtStartPar
L Moresi, F Dufour, and H\sphinxhyphen{}B Muhlhaus. Mantle convection modeling with Viscoelastic/Brittle lithosphere: numerical methodology and plate tectonic modeling. \sphinxstyleemphasis{Pure Appl. Geophys.}, pages 2335, 2002.
\bibitem[MQL+07]{references:id383}
\sphinxAtStartPar
L Moresi, S Quenette, V Lemiale, C Mériaux, B Appelbe, and H\sphinxhyphen{}B Mühlhaus. Computational approaches to studying non\sphinxhyphen{}linear dynamics of the crust and mantle. \sphinxstyleemphasis{Phys. Earth Planet. Inter.}, 163(1):69–82, 7 2007.
\bibitem[MS95]{references:id381}
\sphinxAtStartPar
L N Moresi and V S Solomatov. Numerical investigation of 2D convection with extremely large viscosity variations. \sphinxstyleemphasis{Phys. Fluids}, 8 1995.
\bibitem[MS98]{references:id759}
\sphinxAtStartPar
Louis Moresi and Viatcheslav Solomatov. Mantle convection with a brittle lithosphere: thoughts on the global tectonic styles of the earth and venus. \sphinxstyleemphasis{Geophys. J. Int.}, 133:669–682, 1998.
\bibitem[MZG96]{references:id544}
\sphinxAtStartPar
Louis Moresi, Shijie Zhong, and Michael Gurnis. The accuracy of finite element solutions of stokes' flow with strongly varying viscosity. \sphinxstyleemphasis{ELSEVIER Physics of the Earth and Planetary Interiors}, 97:83–94, 1996.
\bibitem[MeriauxMM+18]{references:id67}
\sphinxAtStartPar
Catherine A Mériaux, Dave A May, John Mansour, Zhihao Chen, and Owen Kaluza. Benchmark of three\sphinxhyphen{}dimensional numerical models of subduction against a laboratory experiment. \sphinxstyleemphasis{Phys. Earth Planet. Inter.}, 283:110–121, 9 2018.
\bibitem[MegeM96]{references:id236}
\sphinxAtStartPar
Daniel Mège and Philippe Masson. A plume tectonics model for the tharsis province, mars. \sphinxstyleemphasis{Planet. Space Sci.}, 44(12):1499–1546, 11 1996.
\bibitem[NHM+16]{references:id145}
\sphinxAtStartPar
F Nimmo, D P Hamilton, W B McKinnon, P M Schenk, R P Binzel, C J Bierson, R A Beyer, J M Moore, S A Stern, H A Weaver, C B Olkin, L A Young, K E Smith, and New Horizons Geology, Geophysics \&Imaging Theme Team. Reorientation of sputnik planitia implies a subsurface ocean on pluto. \sphinxstyleemphasis{Nature}, 540(7631):94–96, 11 2016.
\bibitem[NPM14]{references:id176}
\sphinxAtStartPar
Francis Nimmo, Carolyn Porco, and Colin Mitchell. TIDALLY MODULATED ERUPTIONS ON ENCELADUS: CASSINI ISS OBSERVATIONS AND MODELS. \sphinxstyleemphasis{AJS}, 148(3):46, 6 2014.
\bibitem[NMMF07]{references:id184}
\sphinxAtStartPar
Delphine Nna Mvondo and Jesus Martinez\sphinxhyphen{}Frias. Review komatiites: from earth's geological settings to planetary and astrobiological contexts. \sphinxstyleemphasis{Earth Moon Planets}, 100(3\sphinxhyphen{}4):157–179, 2007.
\bibitem[OBrienGG02]{references:id524}
\sphinxAtStartPar
David P O'Brien, Paul Geissler, and Richard Greenberg. A melt\sphinxhyphen{}through model for chaos formation on europa. \sphinxstyleemphasis{Icarus}, 156(1):152–161, 2 2002.
\bibitem[ONeillD14]{references:id441}
\sphinxAtStartPar
C O'Neill and V Debaille. The evolution of Hadean–Eoarchaean geodynamics. \sphinxstyleemphasis{Earth Planet. Sci. Lett.}, 406:49–58, 10 2014.
\bibitem[ONeillJL07]{references:id675}
\sphinxAtStartPar
C O'Neill, A M Jellinek, and A Lenardic. Conditions for the onset of plate tectonics on terrestrial planets and moons. \sphinxstyleemphasis{Earth Planet. Sci. Lett.}, 261(1–2):20–32, 8 2007.
\bibitem[ONeillL07]{references:id204}
\sphinxAtStartPar
C O'Neill and A Lenardic. Geological consequences of super\sphinxhyphen{}sized earths. \sphinxstyleemphasis{Geophys. Res. Lett.}, 34(19):437, 9 2007.
\bibitem[ONeillLC15]{references:id100}
\sphinxAtStartPar
C O'Neill, A Lenardic, and K C Condie. Earth's punctuated tectonic evolution: cause and effect. \sphinxstyleemphasis{Geological Society, London, Special Publications}, 389(1):17–40, 2015.
\bibitem[ONeillLJM09]{references:id257}
\sphinxAtStartPar
C O'Neill, A Lenardic, A M Jellinek, and L Moresi. Influence of supercontinents on deep mantle flow. \sphinxstyleemphasis{Gondwana Res.}, 15(3):276–287, 5 2009.
\bibitem[ONeillMZB17]{references:id101}
\sphinxAtStartPar
C O'Neill, S Marchi, S Zhang, and W Bottke. Impact\sphinxhyphen{}driven subduction on the hadean earth. \sphinxstyleemphasis{Nat. Geosci.}, 10:793, 8 2017.
\bibitem[ONeillLW+16]{references:id644}
\sphinxAtStartPar
Craig O'Neill, Adrian Lenardic, Matthew Weller, Louis Moresi, Steve Quenette, and Siqi Zhang. A window for plate tectonics in terrestrial planet evolution? \sphinxstyleemphasis{Phys. Earth Planet. Inter.}, 255(C):80–92, 2016.
\bibitem[ONeillN10]{references:id160}
\sphinxAtStartPar
Craig O'Neill and Francis Nimmo. The role of episodic overturn in generating the surface geology and heat flow on enceladus. \sphinxstyleemphasis{Nat. Geosci.}, 3(2):88–91, 1 2010.
\bibitem[ONeillTR18]{references:id346}
\sphinxAtStartPar
Craig O'Neill, Simon Turner, and Tracy Rushmer. The inception of plate tectonics: a record of failure. \sphinxstyleemphasis{Philos. Trans. A Math. Phys. Eng. Sci.}, 9 2018.
\bibitem[OReillyD81]{references:id407}
\sphinxAtStartPar
T C O'Reilly and G F Davies. Magma transport of heat on io \sphinxhyphen{} a mechanism allowing a thick lithosphere. \sphinxstyleemphasis{Geophys. Res. Lett.}, 8:313–316, 1981.
\bibitem[ORourkeK12]{references:id733}
\sphinxAtStartPar
Joseph G O'Rourke and Jun Korenaga. Terrestrial planet evolution in the stagnant\sphinxhyphen{}lid regime: size effects and the formation of self\sphinxhyphen{}destabilizing crust. \sphinxstyleemphasis{Icarus}, 221(2):1043–1060, 10 2012.
\bibitem[Oga18]{references:id311}
\sphinxAtStartPar
Masaki Ogawa. The effects of magmatic redistribution of heat producing elements on the lunar mantle evolution inferred from numerical models that start from various initial states. \sphinxstyleemphasis{Planet. Space Sci.}, 151:43–55, 1 2018.
\bibitem[OWJ+06]{references:id542}
\sphinxAtStartPar
Steven J Ostro, Richard D West, Michael A Janssen, Ralph D Lorenz, Howard A Zebker, Gregory J Black, Jonathan I Lunine, Lauren C Wye, Rosaly M Lopes, Stephen D Wall, Charles Elachi, Laci Roth, Scott Hensley, Kathleen Kelleher, Gary A Hamilton, Yonggyu Gim, Yanhua Z Anderson, Rudy A Boehmer, and William T K Johnson. Cassini RADAR observations of enceladus, tethys, dione, rhea, iapetus, hyperion, and phoebe. \sphinxstyleemphasis{Icarus}, 183(2):479–490, 7 2006.
\bibitem[OBN00]{references:id116}
\sphinxAtStartPar
T C Owen and A Bar\sphinxhyphen{}Nun. Volatile contributions from icy planetesimals. In \sphinxstyleemphasis{Origin of the Earth and Moon}, pages 459–471. adsabs.harvard.edu, 0 2000.
\bibitem[PHP06]{references:id549}
\sphinxAtStartPar
G W Patterson, J W Head, and R T Pappalardo. Plate motion on europa and nonrigid behavior of the icy lithosphere: the castalia macula region. \sphinxstyleemphasis{J. Struct. Geol.}, 28(12):2237–2258, 11 2006.
\bibitem[PdT92]{references:id113}
\sphinxAtStartPar
Hector Perez de Tejada. Solar wind erosion of the mars early atmosphere. \sphinxstyleemphasis{J. Geophys. Res. {[}Space Phys{]}}, 97(A3):3159–3167, 1992.
\bibitem[PHM13]{references:id463}
\sphinxAtStartPar
Erik A Petigura, Andrew W Howard, and Geoffrey W Marcy. Prevalence of earth\sphinxhyphen{}size planets orbiting sun\sphinxhyphen{}like stars. \sphinxstyleemphasis{Proc. Natl. Acad. Sci. U. S. A.}, 110(48):19273–19278, 10 2013.
\bibitem[PMH09]{references:id93}
\sphinxAtStartPar
F Pla, A M Mancho, and H Herrero. Bifurcation phenomena in a convection problem with temperature dependent viscosity at low aspect ratio. \sphinxstyleemphasis{Physica D}, 238(5):572–580, 2 2009.
\bibitem[PHT+06]{references:id172}
\sphinxAtStartPar
C C Porco, P Helfenstein, P C Thomas, A P Ingersoll, J Wisdom, R West, G Neukum, T Denk, R Wagner, T Roatsch, S Kieffer, E Turtle, A McEwen, T V Johnson, J Rathbun, J Veverka, D Wilson, J Perry, J Spitale, A Brahic, J A Burns, A D Delgenio, L Dones, C D Murray, and S Squyres. Cassini observes the active south pole of enceladus. \sphinxstyleemphasis{Science}, 311(5766):1393–1401, 2 2006.
\bibitem[QMSA07]{references:id70}
\sphinxAtStartPar
Steve Quenette, Louis Moresi, P D Sunter, and Bill F Appelbe. Explaining StGermain: an aspect oriented environment for building extensible computational mechanics modeling software. In \sphinxstyleemphasis{Parallel and Distributed Processing Symposium, IPDPS 2007}, 1–8. 2 2007.
\bibitem[Ray16]{references:id76}
\sphinxAtStartPar
Lord Rayleigh. LIX. on convection currents in a horizontal layer of fluid, when the higher temperature is on the under side. \sphinxstyleemphasis{The London, Edinburgh, and Dublin Philosophical Magazine and Journal of Science}, 32(192):529–546, 11 1916.
\bibitem[RR92]{references:id253}
\sphinxAtStartPar
M E Raymo and W F Ruddiman. Tectonic forcing of late cenozoic climate. \sphinxstyleemphasis{Nature}, 359(6391):117–122, 8 1992.
\bibitem[Ree82]{references:id114}
\sphinxAtStartPar
M H Rees. On the interaction of auroral protons with the earth's atmosphere. \sphinxstyleemphasis{Planet. Space Sci.}, 30(5):463–472, 4 1982.
\bibitem[RCF14]{references:id366}
\sphinxAtStartPar
Patrice F Rey, Nicolas Coltice, and Nicolas Flament. Spreading continents kick\sphinxhyphen{}started plate tectonics. \sphinxstyleemphasis{Nature}, 513(7518):405–408, 8 2014.
\bibitem[RGS06]{references:id741}
\sphinxAtStartPar
Jeannie Riley, Richard Greenberg, and Alyssa Sarid. Europa's south pole region: a sequential reconstruction of surface modification processes. \sphinxstyleemphasis{Earth Planet. Sci. Lett.}, 248(3–4):808–821, 7 2006.
\bibitem[RWahlischG+08]{references:id175}
\sphinxAtStartPar
Th Roatsch, M Wählisch, B Giese, A Hoffmeister, K\sphinxhyphen{}D Matz, F Scholten, A Kuhn, R Wagner, G Neukum, P Helfenstein, and C Porco. High\sphinxhyphen{}resolution enceladus atlas derived from Cassini\sphinxhyphen{}ISS images. \sphinxstyleemphasis{Planet. Space Sci.}, 56(1):109–116, 0 2008.
\bibitem[RN08]{references:id165}
\sphinxAtStartPar
James H Roberts and Francis Nimmo. Tidal heating and the long\sphinxhyphen{}term stability of a subsurface ocean on enceladus. \sphinxstyleemphasis{Icarus}, 194(2):675–689, 3 2008.
\bibitem[Rob67]{references:id77}
\sphinxAtStartPar
P H Roberts. Convection in horizontal layers with internal heat generation. theory. \sphinxstyleemphasis{J. Fluid Mech.}, 30(1):33–49, 9 1967.
\bibitem[RS09]{references:id450}
\sphinxAtStartPar
L A Rogers and S Seager. A framework for quantifying the degeneracies of exoplanet interior compositions. \sphinxstyleemphasis{N/A}, 11 2009. \sphinxhref{https://arxiv.org/abs/0912.3288}{arXiv:0912.3288}.
\bibitem[RLopezD10]{references:id518}
\sphinxAtStartPar
Javier Ruiz, Valle López, and James M Dohm. The present\sphinxhyphen{}day thermal state of mars. \sphinxstyleemphasis{Icarus}, 207(2):631–637, 2010.
\bibitem[RSL+17]{references:id119}
\sphinxAtStartPar
K D Runyon, S A Stern, T R Lauer, W Grundy, M E Summers, and K N Singer. A geophysical planet definition. In \sphinxstyleemphasis{48th Lunar and Planetary Science 2017}. adsabs.harvard.edu, 2 2017.
\bibitem[SS86]{references:id62}
\sphinxAtStartPar
Youcef Saad and Martin H Schultz. GMRES: a generalized minimal residual algorithm for solving nonsymmetric linear systems. \sphinxstyleemphasis{SIAM J. Sci. and Stat. Comput.}, 7(3):856–869, 6 1986.
\bibitem[SBS08]{references:id126}
\sphinxAtStartPar
R Sarma, K Baruah, and J K Sarma. IAU planet definition: some confusions and their modifications. \sphinxstyleemphasis{N/A}, 9 2008. \sphinxhref{https://arxiv.org/abs/0810.0993}{arXiv:0810.0993}.
\bibitem[SET18]{references:id354}
\sphinxAtStartPar
Laura Schaefer and Linda T Elkins\sphinxhyphen{}Tanton. Magma oceans as a critical stage in the tectonic development of rocky planets. \sphinxstyleemphasis{Philos. Trans. A Math. Phys. Eng. Sci.}, 9 2018.
\bibitem[SBM+18]{references:id134}
\sphinxAtStartPar
Paul Michael Schenk, Ross A Beyer, William B McKinnon, Jeffrey M Moore, John R Spencer, Oliver L White, Kelsi Singer, Francis Nimmo, Carver Thomason, Tod R Lauer, Stuart Robbins, Orkan M Umurhan, William M Grundy, S Alan Stern, Harold A Weaver, Leslie A Young, K Ennico Smith, and Cathy Olkin. Basins, fractures and volcanoes: global cartography and topography of pluto from new horizons. \sphinxstyleemphasis{Icarus}, 314:400–433, 10 2018.
\bibitem[SBTothers89]{references:id223}
\sphinxAtStartPar
G Schubert, D Bercovici, P J Thomas, and others. Venus coronae: formation by mantle plumes. \sphinxstyleemphasis{Lunar and Planetary}, 1989.
\bibitem[SATP07]{references:id163}
\sphinxAtStartPar
Gerald Schubert, John D Anderson, Bryan J Travis, and Jennifer Palguta. Enceladus: present internal structure and differentiation by early and long\sphinxhyphen{}term radiogenic heating. \sphinxstyleemphasis{Icarus}, 188(2):345–355, 5 2007.
\bibitem[STO01]{references:id89}
\sphinxAtStartPar
Gerald Schubert, Donald L Turcotte, and Peter Olson. \sphinxstyleemphasis{Mantle Convection in the Earth and Planets}. Cambridge University Press, 8 2001.
\bibitem[SSW+17]{references:id146}
\sphinxAtStartPar
K N Singer, P Schenk, O L White, J M Moore, W B McKinnon, W M Grundy, J R Spencer, A Stern, J C Cook, F Nimmo, A D Howard, D P Cruikshank, R A Beyer, O M Umurhan, T Lauer, H A Weaver, Jr, L A Young, and K Ennico Smith. Cryovolcanic resurfacing on pluto. In \sphinxstyleemphasis{AGU Fall Meeting Abstracts}, volume 2017. adsabs.harvard.edu, 11 2017.
\bibitem[SSM+18]{references:id133}
\sphinxAtStartPar
Kelsi N Singer, Paul M Schenk, William B McKinnon, Ross A Beyer, Bernard Schmitt, Oliver L White, Jeffrey M Moore, William Grundy, John Spencer, S A Stern, Tod R Lauer, Catherine B Olkin, Harold A Weaver, Leslie A Young, Kimberly Ennico, Horizons Geology, New, and Geophysics Team. Cryovolcanic constructs on pluto. In \sphinxstyleemphasis{AAS/Division for Planetary Sciences Meeting Abstracts \#50}. 9 2018.
\bibitem[SS07]{references:id102}
\sphinxAtStartPar
N H Sleep and G Schubert. Plate tectonics through time. \sphinxstyleemphasis{Evolution of the Earth}, 9:145–169, 2007.
\bibitem[Sle00]{references:id498}
\sphinxAtStartPar
Norman H Sleep. Evolution of the mode of convection within terrestrial planets. \sphinxstyleemphasis{J. Geophys. Res.}, 105(E7):17563–17578, 6 2000.
\bibitem[SSKC79]{references:id189}
\sphinxAtStartPar
B A Smith, E M Shoemaker, S W Kieffer, and A F Cook. The role of SO 2 in volcanism on io. \sphinxstyleemphasis{Nature}, 1979.
\bibitem[SSB+82]{references:id179}
\sphinxAtStartPar
B A Smith, L Soderblom, R Batson, P Bridges, J Inge, H Masursky, E Shoemaker, R Beebe, J Boyce, G Briggs, A Bunker, S A Collins, C J Hansen, T V Johnson, J L Mitchell, R J Terrile, A F Cook, 2nd, J Cuzzi, J B Pollack, G E Danielson, A P Ingersoll, M E Davies, G E Hunt, D Morrison, T Owen, C Sagan, J Veverka, R Strom, and V E Suomi. A new look at the saturn system: the voyager 2 images. \sphinxstyleemphasis{Science}, 215(4532):504–537, 0 1982.
\bibitem[SS97]{references:id221}
\sphinxAtStartPar
Suzanne E Smrekar and Ellen R Stofan. Corona formation and heat loss on venus by coupled upwelling and delamination. \sphinxstyleemphasis{Science}, 277(5330):1289–1294, 7 1997.
\bibitem[SB19]{references:id298}
\sphinxAtStartPar
Stephan V Sobolev and Michael Brown. Surface erosion events controlled the evolution of plate tectonics on earth. \sphinxstyleemphasis{Nature}, 570(7759):52–57, 5 2019.
\bibitem[Sol95]{references:id408}
\sphinxAtStartPar
V S Solomatov. Scaling of temperature‐ and stress‐dependent viscosity convection. \sphinxstyleemphasis{Phys. Fluids}, 7(2):266–274, 1 1995.
\bibitem[SSC+07]{references:id196}
\sphinxAtStartPar
J R Spencer, S A Stern, A F Cheng, H A Weaver, D C Reuter, K Retherford, A Lunsford, J M Moore, O Abramov, R M C Lopes, J E Perry, L Kamp, M Showalter, K L Jessup, F Marchis, P M Schenk, and C Dumas. Io volcanism seen by new horizons: a major eruption of the tvashtar volcano. \sphinxstyleemphasis{Science}, 318(5848):240–243, 9 2007.
\bibitem[SBE+09]{references:id168}
\sphinxAtStartPar
John R Spencer, Amy C Barr, Larry W Esposito, Paul Helfenstein, Andrew P Ingersoll, Ralf Jaumann, Christopher P McKay, Francis Nimmo, and J Hunter Waite. Enceladus: an active cryovolcanic satellite. In Michele K Dougherty, Larry W Esposito, and Stamatios M Krimigis, editors, \sphinxstyleemphasis{Saturn from Cassini\sphinxhyphen{}Huygens}, pages 683–724. Springer Netherlands, Dordrecht, 2009.
\bibitem[SRCP82]{references:id169}
\sphinxAtStartPar
S W Squyres, R T Reynolds, P M Cassen, and S J Peale. The tectonics of enceladus. \sphinxstyleemphasis{Lunar and Planetary Science}, 12:762–763, 1982.
\bibitem[SRCP83]{references:id180}
\sphinxAtStartPar
Steven W Squyres, Ray T Reynolds, Patrick M Cassen, and Stanton J Peale. The evolution of enceladus. \sphinxstyleemphasis{Icarus}, 53(2):319–331, 1 1983.
\bibitem[StamenkovicNBS12]{references:id433}
\sphinxAtStartPar
Vlada Stamenković, Lena Noack, Doris Breuer, and Tilman Spohn. THE INFLUENCE OF PRESSURE\sphinxhyphen{}DEPENDENT VISCOSITY ON THE THERMAL EVOLUTION OF SUPER\sphinxhyphen{}EARTHS. \sphinxstyleemphasis{ApJ}, 748(1):41, 2 2012.
\bibitem[StamenkovicS16]{references:id434}
\sphinxAtStartPar
Vlada Stamenković and Sara Seager. EMERGING POSSIBILITIES AND INSUPERABLE LIMITATIONS OF EXOGEOPHYSICS: THE EXAMPLE OF PLATE TECTONICS. \sphinxstyleemphasis{ApJ}, 825(1):78, 6 2016.
\bibitem[SLH13]{references:id747}
\sphinxAtStartPar
C Stein, J P Lowman, and U Hansen. The influence of mantle internal heating on lithospheric mobility: implications for super\sphinxhyphen{}earths. \sphinxstyleemphasis{Earth Planet. Sci. Lett.}, 361:448–459, 0 2013.
\bibitem[Ste16]{references:id417}
\sphinxAtStartPar
Robert J Stern. Is plate tectonics needed to evolve technological species on exoplanets? \sphinxstyleemphasis{Geoscience Frontiers}, 7(4):573–580, 6 2016.
\bibitem[Ste18]{references:id349}
\sphinxAtStartPar
Robert J Stern. The evolution of plate tectonics. \sphinxstyleemphasis{Philos. Trans. A Math. Phys. Eng. Sci.}, 9 2018.
\bibitem[SBE+15]{references:id154}
\sphinxAtStartPar
S A Stern, F Bagenal, K Ennico, G R Gladstone, W M Grundy, W B McKinnon, J M Moore, C B Olkin, J R Spencer, H A Weaver, L A Young, T Andert, J Andrews, M Banks, B Bauer, J Bauman, O S Barnouin, P Bedini, K Beisser, R A Beyer, S Bhaskaran, R P Binzel, E Birath, M Bird, D J Bogan, A Bowman, V J Bray, M Brozovic, C Bryan, M R Buckley, M W Buie, B J Buratti, S S Bushman, A Calloway, B Carcich, A F Cheng, S Conard, C A Conrad, J C Cook, D P Cruikshank, O S Custodio, C M Dalle Ore, C Deboy, Z J B Dischner, P Dumont, A M Earle, H A Elliott, J Ercol, C M Ernst, T Finley, S H Flanigan, G Fountain, M J Freeze, T Greathouse, J L Green, Y Guo, M Hahn, D P Hamilton, S A Hamilton, J Hanley, A Harch, H M Hart, C B Hersman, A Hill, M E Hill, D P Hinson, M E Holdridge, M Horanyi, A D Howard, C J A Howett, C Jackman, R A Jacobson, D E Jennings, J A Kammer, H K Kang, D E Kaufmann, P Kollmann, S M Krimigis, D Kusnierkiewicz, T R Lauer, J E Lee, K L Lindstrom, I R Linscott, C M Lisse, A W Lunsford, V A Mallder, N Martin, D J McComas, R L McNutt, Jr, D Mehoke, T Mehoke, E D Melin, M Mutchler, D Nelson, F Nimmo, J I Nunez, A Ocampo, W M Owen, M Paetzold, B Page, A H Parker, J W Parker, F Pelletier, J Peterson, N Pinkine, M Piquette, S B Porter, S Protopapa, J Redfern, H J Reitsema, D C Reuter, J H Roberts, S J Robbins, G Rogers, D Rose, K Runyon, K D Retherford, M G Ryschkewitsch, P Schenk, E Schindhelm, B Sepan, M R Showalter, K N Singer, M Soluri, D Stanbridge, A J Steffl, D F Strobel, T Stryk, M E Summers, J R Szalay, M Tapley, A Taylor, H Taylor, H B Throop, C C C Tsang, G L Tyler, O M Umurhan, A J Verbiscer, M H Versteeg, M Vincent, R Webbert, S Weidner, G E Weigle, 2nd, O L White, K Whittenburg, B G Williams, K Williams, S Williams, W W Woods, A M Zangari, and E Zirnstein. The pluto system: initial results from its exploration by new horizons. \sphinxstyleemphasis{Science}, 350(6258):aad1815, 9 2015.
\bibitem[SGM+18]{references:id153}
\sphinxAtStartPar
S Alan Stern, William M Grundy, William B McKinnon, Harold A Weaver, and Leslie A Young. The pluto system after new horizons. \sphinxstyleemphasis{Annu. Rev. Astron. Astrophys.}, 56(1):357–392, 8 2018.
\bibitem[Ste03]{references:id647}
\sphinxAtStartPar
David J Stevenson. Styles of mantle convection and their influence on planetary evolution. \sphinxstyleemphasis{C. R. Geosci.}, 335(1):99–111, 2003.
\bibitem[SSS+92]{references:id224}
\sphinxAtStartPar
Ellen R Stofan, Virgil L Sharpton, Gerald Schubert, Gidon Baer, Duane L Bindschadler, Daniel M Janes, and Steven W Squyres. Global distribution and characteristics of coronae and related features on venus: implications for origin and relation to mantle processes. \sphinxstyleemphasis{J. Geophys. Res.}, 97(E8):13347, 1992.
\bibitem[SS05]{references:id225}
\sphinxAtStartPar
Ellen R Stofan and Suzanne E Smrekar. Large topographic rises, coronae, large flow fields, and large volcanoes on venus: evidence for mantle plumes? \sphinxstyleemphasis{SPECIAL PAPERS\sphinxhyphen{}GEOLOGICAL SOCIETY OF AMERICA}, 388:841, 2005.
\bibitem[SWR06]{references:id248}
\sphinxAtStartPar
Drew B Stolar, Sean D Willett, and Gerard H Roe. Climatic and tectonic forcing of a critical orogen. \sphinxstyleemphasis{SPECIAL PAPERS\sphinxhyphen{}GEOLOGICAL SOCIETY OF AMERICA}, 398:241, 2006.
\bibitem[Swa03]{references:id61}
\sphinxAtStartPar
Paul N Swarztrauber. On computing the points and weights for Gauss–Legendre quadrature. \sphinxstyleemphasis{SIAM J. Sci. Comput.}, 24(3):945–954, 0 2003.
\bibitem[Tac96]{references:id84}
\sphinxAtStartPar
Paul J Tackley. Effects of strongly variable viscosity on three\sphinxhyphen{}dimensional compressible convection in planetary mantles. \sphinxstyleemphasis{J. Geophys. Res.}, 101(B2):3311–3332, 1 1996.
\bibitem[TL00]{references:id274}
\sphinxAtStartPar
T Tanimoto and T Lay. Mantle dynamics and seismic tomography. \sphinxstyleemphasis{Proc. Natl. Acad. Sci. U. S. A.}, 97(23):12409–12410, 10 2000.
\bibitem[TPR+18]{references:id151}
\sphinxAtStartPar
Matt W Telfer, Eric J R Parteli, Jani Radebaugh, Ross A Beyer, Tanguy Bertrand, François Forget, Francis Nimmo, Will M Grundy, Jeffrey M Moore, S Alan Stern, John Spencer, Tod R Lauer, Alissa M Earle, Richard P Binzel, Hal A Weaver, Cathy B Olkin, Leslie A Young, Kimberly Ennico, Kirby Runyon, New Horizons Geology, Geophysics and Imaging Science Theme Team, Marc Buie, Bonnie Buratti, Andy Cheng, J J Kavelaars, Ivan Linscott, William B McKinnon, Harold Reitsema, Dennis Reuter, Paul Schenk, Mark Showalter, and Len Tyler. Dunes on pluto. \sphinxstyleemphasis{Science}, 360(6392):992–997, 5 2018.
\bibitem[TS11]{references:id111}
\sphinxAtStartPar
Michael Thoennessen and Bradley Sherrill. From isotopes to the stars. \sphinxstyleemphasis{Nature}, 473(7345):25, 2011.
\bibitem[Tho62]{references:id103}
\sphinxAtStartPar
William Thomson. XV.— on the secular cooling of the earth. \sphinxstyleemphasis{Earth Environ. Sci. Trans. R. Soc. Edinb.}, 23(1):157–169, 1862.
\bibitem[TvCadekS08]{references:id166}
\sphinxAtStartPar
G Tobie, O Čadek, and C Sotin. Solid tidal friction above a liquid water reservoir as the origin of the south pole hotspot on enceladus. \sphinxstyleemphasis{Icarus}, 196(2):642–652, 7 2008.
\bibitem[TSBS06]{references:id270}
\sphinxAtStartPar
Trond H Torsvik, Mark A Smethurst, Kevin Burke, and Bernhard Steinberger. Large igneous provinces generated from the margins of the large low\sphinxhyphen{}velocity provinces in the deep mantle. \sphinxstyleemphasis{Geophys. J. Int.}, 167(3):1447–1460, 11 2006.
\bibitem[TWT+19]{references:id199}
\sphinxAtStartPar
Angelos Tsiaras, Ingo P Waldmann, Giovanna Tinetti, Jonathan Tennyson, and Sergey N Yurchenko. Water vapour in the atmosphere of the habitable\sphinxhyphen{}zone eight\sphinxhyphen{}earth\sphinxhyphen{}mass planet K2\sphinxhyphen{}18 b. \sphinxstyleemphasis{Nature Astronomy}, 8 2019.
\bibitem[TKM13]{references:id501}
\sphinxAtStartPar
Taku Tsuchiya, Kenji Kawai, and Shigenori Maruyama. Expanding\sphinxhyphen{}contracting earth. \sphinxstyleemphasis{Geoscience Frontiers}, 4(3):341–347, 2013.
\bibitem[TMRM99]{references:id639}
\sphinxAtStartPar
D L Turcotte, G Morein, D Roberts, and B D Malamud. Catastrophic resurfacing and episodic subduction on venus. \sphinxstyleemphasis{Icarus}, 139:49–54, 1999.
\bibitem[TO69]{references:id79}
\sphinxAtStartPar
D L Turcotte and E R Oxburgh. Convection in a mantle with variable physical properties. \sphinxstyleemphasis{J. Geophys. Res.}, 74(6):1458–1474, 2 1969.
\bibitem[TS14]{references:id88}
\sphinxAtStartPar
Donald Turcotte and Gerald Schubert. \sphinxstyleemphasis{Geodynamics}. Cambridge University Press, 3 2014.
\bibitem[THH15]{references:id190}
\sphinxAtStartPar
Robert H Tyler, Wade G Henning, and Christopher W Hamilton. TIDAL HEATING IN a MAGMA OCEAN WITHIN JUPITER'S MOON io. \sphinxstyleemphasis{ApJS}, 218(2):22, 5 2015.
\bibitem[Tys09]{references:id123}
\sphinxAtStartPar
Neil Degrasse Tyson. \sphinxstyleemphasis{The Pluto Files: The Rise and Fall of America's Favorite Planet}. W. W. Norton \& Company, 11 2009.
\bibitem[TothN14]{references:id465}
\sphinxAtStartPar
Zs Tóth and I Nagy. Dynamical stability of the gliese 581 exoplanetary system. \sphinxstyleemphasis{Mon. Not. R. Astron. Soc.}, 442(1):454–461, 6 2014.
\bibitem[Ure55]{references:id399}
\sphinxAtStartPar
H C Urey. THE COSMIC ABUNDANCES OF POTASSIUM, URANIUM, AND THORIUM AND THE HEAT BALANCES OF THE EARTH, THE MOON, AND MARS. \sphinxstyleemphasis{Proc. Natl. Acad. Sci. U. S. A.}, 41(3):127–144, 2 1955.
\bibitem[VGPF13]{references:id207}
\sphinxAtStartPar
Diana Valencia, Tristan Guillot, Vivien Parmentier, and Richard S Freedman. BULK COMPOSITION OF GJ 1214b AND OTHER SUB\sphinxhyphen{}NEPTUNE EXOPLANETS. \sphinxstyleemphasis{ApJ}, 775(1):10, 7 2013.
\bibitem[VOConnell09]{references:id643}
\sphinxAtStartPar
Diana Valencia and Richard J O'Connell. Convection scaling and subduction on earth and super\sphinxhyphen{}earths. \sphinxstyleemphasis{Earth Planet. Sci. Lett.}, 286(3–4):492–502, 8 2009.
\bibitem[VOConnellS07]{references:id426}
\sphinxAtStartPar
Diana Valencia, Richard J O'Connell, and Dimitar D Sasselov. Inevitability of plate tectonics on Super\sphinxhyphen{}Earths. \sphinxstyleemphasis{ApJ}, 670(1):L45, 9 2007.
\bibitem[Val05]{references:id245}
\sphinxAtStartPar
John W Valley. A cool early earth? \sphinxstyleemphasis{Sci. Am.}, 293(4):58–65, 9 2005.
\bibitem[VB76]{references:id230}
\sphinxAtStartPar
R W Van Bemmelen. Plate tectonics and the undation model: a comparison. \sphinxstyleemphasis{Tectonophysics}, 32(3):145–182, 5 1976.
\bibitem[VDL85]{references:id281}
\sphinxAtStartPar
Willem J M Van Der Linden. Orogeny: J.G. dennis (editor). benchmark papers in geology 62. hutchinson ross publ. cy, stroudsburg, pa., 1982, XV + 379 pp., US \textbackslash{}\$ 46.00, hardcover. \sphinxstyleemphasis{Tectonophysics}, 111(1):167–170, 0 1985.
\bibitem[vHT11]{references:id744}
\sphinxAtStartPar
H J van Heck and P J Tackley. Plate tectonics on super\sphinxhyphen{}earths: equally or more likely than on earth. \sphinxstyleemphasis{Earth Planet. Sci. Lett.}, 310(3–4):252–261, 9 2011.
\bibitem[VK10]{references:id234}
\sphinxAtStartPar
Martin J Van Kranendonk. Two types of archean continental crust: plume and plate tectonics on early earth. \sphinxstyleemphasis{Am. J. Sci.}, 310(10):1187–1209, 11 2010.
\bibitem[vTVvdB05]{references:id625}
\sphinxAtStartPar
P van Thienen, N J Vlaar, and A P van den Berg. Assessment of the cooling capacity of plate tectonics and flood volcanism in the evolution of earth, mars and venus. \sphinxstyleemphasis{Phys. Earth Planet. Inter.}, 150:287–315, 2005.
\bibitem[VDM+12]{references:id505}
\sphinxAtStartPar
Glenn J Veeder, Ashley Gerard Davies, Dennis L Matson, Torrence V Johnson, David A Williams, and Jani Radebaugh. Io: volcanic thermal sources and global heat flow. \sphinxstyleemphasis{Icarus}, 219(2):701–722, 2012.
\bibitem[VelicMM09]{references:id65}
\sphinxAtStartPar
Mirko Velić, Dave May, and Louis Moresi. A fast robust algorithm for computing discrete voronoi diagrams. \sphinxstyleemphasis{J. Math. Model. Algorithms}, 8(3):343–355, 7 2009.
\bibitem[Ver80]{references:id107}
\sphinxAtStartPar
John Verhoogen. \sphinxstyleemphasis{Energetics of the Earth}. National Academies, 1980.
\bibitem[WCI+06]{references:id177}
\sphinxAtStartPar
J Hunter Waite, Jr, Michael R Combi, Wing\sphinxhyphen{}Huen Ip, Thomas E Cravens, Ralph L McNutt, Jr, Wayne Kasprzak, Roger Yelle, Janet Luhmann, Hasso Niemann, David Gell, Brian Magee, Greg Fletcher, Jonathan Lunine, and Wei\sphinxhyphen{}Ling Tseng. Cassini ion and neutral mass spectrometer: enceladus plume composition and structure. \sphinxstyleemphasis{Science}, 311(5766):1419–1422, 2 2006.
\bibitem[WH17]{references:id429}
\sphinxAtStartPar
U Walzer and R Hendel. Continental crust formation: numerical modelling of chemical evolution and geological implications. \sphinxstyleemphasis{Lithos}, 278–281:215–228, 2017.
\bibitem[War85]{references:id240}
\sphinxAtStartPar
P H Warren. The magma ocean concept and lunar evolution. \sphinxstyleemphasis{Annu. Rev. Earth Planet. Sci.}, 1985.
\bibitem[Weg24]{references:id280}
\sphinxAtStartPar
Alfred Wegener. \sphinxstyleemphasis{The origin of continents and oceans}. Methuen, 1924.
\bibitem[WM14]{references:id214}
\sphinxAtStartPar
Lauren M Weiss and Geoffrey W Marcy. THE MASS\sphinxhyphen{}RADIUS RELATION FOR 65 EXOPLANETS SMALLER THAN 4 EARTH RADII. \sphinxstyleemphasis{ApJL}, 783(1):L6, 1 2014.
\bibitem[WL12]{references:id96}
\sphinxAtStartPar
M B Weller and A Lenardic. Hysteresis in mantle convection: plate tectonics systems. \sphinxstyleemphasis{Geophys. Res. Lett.}, 4 2012.
\bibitem[WLONeill15]{references:id513}
\sphinxAtStartPar
M B Weller, A Lenardic, and C O'Neill. The effects of internal heating and large scale climate variations on tectonic bi\sphinxhyphen{}stability in terrestrial planets. \sphinxstyleemphasis{Earth Planet. Sci. Lett.}, 420(C):85–94, 5 2015.
\bibitem[WLM16]{references:id291}
\sphinxAtStartPar
Matthew B Weller, Adrian Lenardic, and William B Moore. Scaling relationships and physics for mixed heating convection in planetary interiors: isoviscous spherical shells: scaling and physics for mixed heating. \sphinxstyleemphasis{J. Geophys. Res. {[}Solid Earth{]}}, 121(10):7598–7617, 9 2016.
\bibitem[Wer09]{references:id504}
\sphinxAtStartPar
Stephanie C Werner. The global martian volcanic evolutionary history. \sphinxstyleemphasis{Icarus}, 201(1):44–68, 2009.
\bibitem[Whi84]{references:id83}
\sphinxAtStartPar
Frank M White. \sphinxstyleemphasis{Heat transfer}. Addison\sphinxhyphen{}Wesley Longman, Incorporated, 1984.
\bibitem[WWG00]{references:id185}
\sphinxAtStartPar
David A Williams, Allan H Wilson, and Ronald Greeley. A komatiite analog to potential ultramafic materials on io. \sphinxstyleemphasis{J. Geophys. Res.}, 105(E1):1671–1684, 0 2000.
\bibitem[WL15]{references:id208}
\sphinxAtStartPar
Angie Wolfgang and Eric Lopez. How rocky are they? the composition distribution of kepler's sub\sphinxhyphen{}neptune planet candidates within 0.15 AU. \sphinxstyleemphasis{Astrophys. J.}, 806(2):183, 2015.
\bibitem[WDD09]{references:id339}
\sphinxAtStartPar
M Wolstencroft, J H Davies, and D R Davies. Nusselt–Rayleigh number scaling for spherical shell earth mantle simulation up to a rayleigh number of 109. \sphinxstyleemphasis{Phys. Earth Planet. Inter.}, 176(1):132–141, 8 2009.
\bibitem[WF92]{references:id469}
\sphinxAtStartPar
A Wolszczan and D A Frail. A planetary system around the millisecond pulsar PSR1257 + 12. \sphinxstyleemphasis{Nature}, 355:145–147, 1992.
\bibitem[YS11]{references:id631}
\sphinxAtStartPar
Masaki Yoshida and M Santosh. Supercontinents, mantle dynamics and plate tectonics: a perspective based on conceptual vs. numerical models. \sphinxstyleemphasis{Earth\sphinxhyphen{}Sci. Rev.}, 105(1–2):1–24, 2011.
\bibitem[YS18]{references:id256}
\sphinxAtStartPar
Masaki Yoshida and M Santosh. Voyage of the indian subcontinent since pangea breakup and driving force of supercontinent cycles: insights on dynamics from numerical modeling. \sphinxstyleemphasis{Geoscience Frontiers}, 9(5):1279–1292, 8 2018.
\bibitem[ZSJ16]{references:id105}
\sphinxAtStartPar
Li Zeng, Dimitar D Sasselov, and Stein B Jacobsen. Mass–radius relation for rocky planets based on PREM. \sphinxstyleemphasis{Astrophys. J.}, 819(2):127, 2016.
\bibitem[Zer18]{references:id342}
\sphinxAtStartPar
Aubrey L Zerkle. Biogeodynamics: bridging the gap between surface and deep earth processes. \sphinxstyleemphasis{Philos. Trans. A Math. Phys. Eng. Sci.}, 9 2018.
\bibitem[Zha09]{references:id266}
\sphinxAtStartPar
Dapeng Zhao. Multiscale seismic tomography and mantle dynamics. \sphinxstyleemphasis{Gondwana Res.}, 15(3):297–323, 0 2009.
\bibitem[ZGM98]{references:id385}
\sphinxAtStartPar
Shijie Zhong, Michael Gurnis, and Louis Moresi. Role of faults, nonlinear rheology, and viscosity structure in generating plates from instantaneous mantle flow models. \sphinxstyleemphasis{J. Geophys. Res.}, 103(B7):15255–15268, 6 1998.
\bibitem[ZL16]{references:id438}
\sphinxAtStartPar
Shijie Zhong and Xi Liu. The long\sphinxhyphen{}wavelength mantle structure and dynamics and implications for large\sphinxhyphen{}scale tectonics and volcanism in the phanerozoic. \sphinxstyleemphasis{Gondwana Res.}, 29(1):83–104, 2016.
\bibitem[ZZLR07]{references:id629}
\sphinxAtStartPar
Shijie Zhong, Nan Zhang, Zheng\sphinxhyphen{}Xiang Li, and James H Roberts. Supercontinent cycles, true polar wander, and very long\sphinxhyphen{}wavelength mantle convection. \sphinxstyleemphasis{Earth Planet. Sci. Lett.}, 261:551, 2007.
\bibitem[Zie07]{references:id124}
\sphinxAtStartPar
Sarah Zielinski. In brief: new mexico declares pluto a planet. \sphinxstyleemphasis{Eos Trans. AGU}, 88(11):133, 2 2007.
\bibitem[vCivzkovaB19]{references:id273}
\sphinxAtStartPar
Hana Čížková and Craig R Bina. Linked influences on slab stagnation: interplay between lower mantle viscosity structure, phase transitions, and plate coupling. \sphinxstyleemphasis{Earth Planet. Sci. Lett.}, 509:88–99, 2 2019.
\bibitem[AristotleG39]{references:id285}
\sphinxAtStartPar
Aristotle and W K C Guthrie. \sphinxstyleemphasis{On the heavens}. Harvard University Press ; Heinemann, 1939.
\bibitem[HEASARC17]{references:id464}
\sphinxAtStartPar
HEASARC. SAO star catalog. 2017.
\bibitem[IEA07]{references:id109}
\sphinxAtStartPar
IEA. \sphinxstyleemphasis{Key world energy statistics}. International Energy Agency Paris, 2007.
\bibitem[InternationalAUnion06]{references:id472}
\sphinxAtStartPar
International Astronomical Union. IAU 2006 general assembly: resolutions 5 and 6. In \sphinxstyleemphasis{2006 International Astronomical Union (IAU) General Assembly}. 7 2006.
\bibitem[MELT98]{references:id268}
\sphinxAtStartPar
MELT. Imaging the deep seismic structure beneath a mid\sphinxhyphen{}ocean ridge: the MELT experiment. \sphinxstyleemphasis{Science}, 280(5367):1215–1218, 4 1998.
\bibitem[NASAESInstitute19]{references:id462}
\sphinxAtStartPar
NASA Exoplanet Science Institute. NASA exoplanet archive. 7 2019.
\end{sphinxthebibliography}







\renewcommand{\indexname}{Index}
\printindex
\end{document}